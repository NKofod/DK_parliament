%!TEX TS-program = xelatex
%!TEX encoding = UTF-8 Unicode
\documentclass[11pt, a4paper]{awesome-cv}
\geometry{left=1.4cm, top=.8cm, right=1.4cm, bottom=1.8cm, footskip=.5cm}
\fontdir[fonts/]
\colorlet{awesome}{RV-colour}
\setbool{acvSectionColorHighlight}{true}
\renewcommand{\acvHeaderSocialSep}{\quad\textbar\quad}
\recipient{}{}
\name{Marianne}{Jelved}
\mobile{+45 3337 4703}
\email{marianne.jelvedft.dk}
\position{Fhv. minister{\enskip\cdotp\enskip}Radikale Venstre}
\address{}
\photo[circle,noedge,left]{"./party_RV/Marianne_Jelved_profile.jpg"}
\letterdate{\today}
\lettertitle{Marianne Jelved - Blå Bog}
\letteropening{}
\letterclosing{}
\letterenclosure[Attached]{Stemme Statistik}
\begin{document}
\makecvheader[R]
\makecvfooter{\today}{\lettertitle{Marianne Jelved - Blå Bog}}{}
\makelettertitle
\begin{cvletter}
\lettersection{Baggrund}
Marianne Bruus Jelved, foslashdt 5. september 1943 i Charlottenlund, datter af arkitekt Sigurd Hirsbro og Else Hirsbro. Gift med Jan Jelved.nbsp

\lettersection{Uddannelse}
\begin{itemize}
\item Cand.pæd. i dansk, Danmarks Lærerhøjskole,19791979.
\item Lærereksamen, Hellerup Seminarium,19671967.
\item Student, Statens Studenterkursus, Frederiksberg,19641964.
\item Maglegårdsskolen, Charlottenlund,19501960.
\end{itemize}
\lettersection{Parlamentarisk Karriere}
\subsection*{Ministerposter}
\begin{itemize}
\item Kulturminister og kirkeminister3. februar 2014  28. juni 2015.
\item Kulturminister6. december 2012  28. juni 2015.
\item Økonomiminister og minister for nordisk samarbejde27. september 1994  27. november 2001.
\item Økonomiminister25. januar 1993  27. september 1994.
\end{itemize}
\subsection*{Ordførerskaber}
\begin{itemize}
\item Ordfører for frie skoleformer
\item Skoleordfører
\item Folkeoplysningsordfører
\item Kirkeordfører
\end{itemize}
\subsection*{Parlamentariske Tillidsposter}
\begin{itemize}
\item Formand for Radikale Venstres folketingsgruppe20112012.
\item Formand for Udvalget for Videnskab og Teknologi20072011.
\item Formand for Radikale Venstres folketingsgruppe20012007.
\item Medlem af Nordisk Råd20012003.
\item Formand for Radikale Venstres folketingsgruppe19881993.
\item Medlem af Nordisk Råd19881993.
\end{itemize}
\subsection*{Folketinget}
\subsubsection*{Medlemsperioder}
\begin{itemize}
\item Folketingsmedlem for Radikale Venstre i Nordjyllands Storkreds fra 13. november 2007.
\item Folketingsmedlem for Radikale Venstre i Nordjyllands Amtskreds21. september 1994  13. november 2007.
\item Folketingsmedlem for Radikale Venstre i Nordjyllands Amtskreds12. december 1990  1. december 1993.
\item Folketingsmedlem for Radikale Venstre i Roskilde Amtskreds8. september 1987  12. december 1990.
\end{itemize}
\subsubsection*{Kandidaturer}
\begin{itemize}
\item Kandidat for Radikale Venstre i Hjørringkredsenfra 1990.
\item Kandidat for Radikale Venstre i Roskildekredsen19821990.
\end{itemize}
\subsection*{Folketingets Præsidium}
\begin{itemize}
\item Medlem af Folketingets Præsidium30. september 2011  5. december 2012.
\end{itemize}
\lettersection{Erhvervserfaring}
\begin{itemize}
\item Timelærer ved Danmarks Lærerhøjskole,19791987.
\item Lærer ved Jyllinge Skole, Gundsø,19751989.
\item Lærer ved Ny Østensgård Skole, Valby,19671975.
\end{itemize}
\lettersection{Publikationer}
Har skrevet Alt har sin pris, 1999, og undervisningsmateriale, bl.a. Danskbogen 3, 1988, og Hvad er meningen  Folketinget i arbejde i verden, 2005. Redaktør af Meddelelser fra Dansklærerforeningen, 19751982, og af Det er dansk  læseplan og hverdag, 1984.

\end{cvletter}
\end{document}