%!TEX TS-program = xelatex
%!TEX encoding = UTF-8 Unicode
\documentclass[11pt, a4paper]{awesome-cv}
\geometry{left=1.4cm, top=.8cm, right=1.4cm, bottom=1.8cm, footskip=.5cm}
\fontdir[fonts/]
\colorlet{awesome}{RV-colour}
\setbool{acvSectionColorHighlight}{true}
\renewcommand{\acvHeaderSocialSep}{\quad\textbar\quad}
\recipient{}{}
\name{Sofie Carsten}{Nielsen}
\mobile{+45 3337 5500}
\email{sofie.carsten.nielsenft.dk}
\position{Medlem af Folketinget{\enskip\cdotp\enskip}Radikale Venstre}
\address{}
\photo[circle,noedge,left]{"./party_RV/Sofie Carsten_Nielsen_profile.jpg"}
\letterdate{\today}
\lettertitle{Sofie Carsten Nielsen - Blå Bog}
\letteropening{}
\letterclosing{}
\letterenclosure[Attached]{Stemme Statistik}
\begin{document}
\makecvheader[R]
\makecvfooter{\today}{\lettertitle{Sofie Carsten Nielsen - Blå Bog}}{}
\makelettertitle
\begin{cvletter}
\lettersection{Baggrund}
Sofie Carsten Nielsen, foslashdt 24. maj 1975 i Birkeroslashd, datter af pensionist Jens Carsten Nielsen og pensionist Kirsten Carsten Nielsen.nbspGift med Rasmus Dalsgaard. Parret har boslashrnene Gustav og Villads.

\lettersection{Uddannelse}
\begin{itemize}
\item Cand.scient.pol., Københavns Universitet,19952002.
\item Master i europæisk politik og administration, Europakollegiet i Brügge, Belgien,20002001.
\end{itemize}
\lettersection{Parlamentarisk Karriere}
\subsection*{Ministerposter}
\begin{itemize}
\item Uddannelses og forskningsminister3. februar 2014  28. juni 2015.
\end{itemize}
\subsection*{Parlamentariske Tillidsposter}
\begin{itemize}
\item Politisk leder for Radikale Venstrefra 2020.
\item Finansordfører20192020.
\item Politisk ordfører20192020.
\item Næstformand for Radikale Venstres folketingsgruppe20152020.
\item Stedfortrædende politisk leder for Radikale Venstre20142020.
\item Formand for Finansudvalget20122014.
\item Formand for Radikale Venstres folketingsgruppe20122014.
\item Næstformand for Radikale Venstres folketingsgruppe20112012.
\item Næstformand for Ligestillingsudvalget20112012.
\end{itemize}
\subsection*{Folketinget}
\subsubsection*{Medlemsperioder}
\begin{itemize}
\item Folketingsmedlem for Radikale Venstre i Københavns Omegns Storkreds fra 15. september 2011.
\end{itemize}
\subsubsection*{Kandidaturer}
\begin{itemize}
\item Kandidat for Radikale Venstre i Gentoftekredsenfra 2010.
\end{itemize}
\lettersection{Erhvervserfaring}
\begin{itemize}
\item Politisk chef, Ingeniørforeningen IDA,20102011.
\item Souschef, Ligestillingsministeriet,20042009.
\item Politisk konsulent, EuropaParlamentet,20022004.
\end{itemize}
\lettersection{Publikationer}
Bidragsyder til antologien raquoTilbage til roslashdderne  om de radikale vaeligrdierlaquo, 2008 ognbspantologiennbspnbspraquoDe roslashde skolaquo, 2002.

\end{cvletter}
\end{document}