%!TEX TS-program = xelatex
%!TEX encoding = UTF-8 Unicode
\documentclass[11pt, a4paper]{awesome-cv}
\geometry{left=1.4cm, top=.8cm, right=1.4cm, bottom=1.8cm, footskip=.5cm}
\fontdir[fonts/]
\colorlet{awesome}{S-colour}
\setbool{acvSectionColorHighlight}{true}
\renewcommand{\acvHeaderSocialSep}{\quad\textbar\quad}
\recipient{}{}
\name{Nick}{Hækkerup}
\mobile{+453337 5500}
\email{nick.haekkerupft.dk}
\position{Fhv. minister{\enskip\cdotp\enskip}Socialdemokratiet}
\address{}
\photo[circle,noedge,left]{"./party_S/Nick_Hækkerup_profile.jpg"}
\letterdate{\today}
\lettertitle{Nick Hækkerup - Blå Bog}
\letteropening{}
\letterclosing{}
\letterenclosure[Attached]{Stemme Statistik}
\begin{document}
\makecvheader[R]
\makecvfooter{\today}{\lettertitle{Nick Hækkerup - Blå Bog}}{}
\makelettertitle
\begin{cvletter}
\lettersection{Baggrund}
Nick Haeligkkerup, foslashdt 3. april 1968, soslashn af fhv. borgmester og fhv. MF Klaus Haeligkkerup og fhv. skoleinspektoslashr Irene Haeligkkerup.nbspGift med landskabsarkitekt Petra Freisleben Haeligkkerup. Har boslashrnene Fie, foslashdt i 1994, Mille, foslashdt i 1998, Emil, foslashdt i 2003 og Malthe, foslashdt i 2006.

\lettersection{Uddannelse}
\begin{itemize}
\item Ph.d., Københavns Universitet,19941998.
\item Cand.jur., Københavns Universitet,19881994.
\end{itemize}
\lettersection{Parlamentarisk Karriere}
\subsection*{Ministerposter}
\begin{itemize}
\item Justitsminister27. juni 2019  2. maj 2022.
\item Minister for sundhed og forebyggelse3. februar 2014  28. juni 2015.
\item Handels og europaminister9. august 2013  3. februar 2014.
\item Forsvarsminister3. oktober 2011  9. august 2013.
\end{itemize}
\subsection*{Parlamentariske Tillidsposter}
\begin{itemize}
\item Næstformand for Det Udenrigspolitiske Nævn20152019.
\item Medlem af Udenrigsudvalget formand 2015201620152019.
\item Medlem af Finansudvalget og Skatteudvalget20072011.
\item Næstformand for Socialdemokratiet20052012.
\end{itemize}
\subsection*{Folketinget}
\subsubsection*{Medlemsperioder}
\begin{itemize}
\item Folketingsmedlem for Socialdemokratiet i Nordsjællands Storkreds fra 13. november 2007.
\end{itemize}
\subsubsection*{Kandidaturer}
\begin{itemize}
\item Kandidat for Socialdemokratiet i Hillerødkredsenfra 2018.
\item Kandidat for Socialdemokratiet i Egedalkredsen20072018.
\end{itemize}
\lettersection{Erhvervserfaring}
\begin{itemize}
\item Underviser i forfatningsret, Københavns Universitet,20172019.
\item Adjunkt, Københavns Universitet,19982000.
\item Fuldmægtig ved Landsskatteretten,19941994.
\end{itemize}
\lettersection{Publikationer}
Har skrevet ph.d.afhandlingen raquoKontrol og sanktioner i EFrettenlaquo, 1998. Medforfatter til raquoSandheden Kort  Christiansborg fra A til Aringlaquo, Peoples Press, 2018, raquoControls and Sanctions in the EU Lawlaquo, Djoslashf Forlag, 2001 og raquoUdvikling i EU siden 1992 paring de omraringder, der er omfattet af de danske forbeholdlaquo Dansk Udenrigspolitisk Institut DUPI, 2001.

\end{cvletter}
\end{document}