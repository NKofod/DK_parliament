%!TEX TS-program = xelatex
%!TEX encoding = UTF-8 Unicode
\documentclass[11pt, a4paper]{awesome-cv}
\geometry{left=1.4cm, top=.8cm, right=1.4cm, bottom=1.8cm, footskip=.5cm}
\fontdir[fonts/]
\colorlet{awesome}{UFG-colour}
\setbool{acvSectionColorHighlight}{true}
\renewcommand{\acvHeaderSocialSep}{\quad\textbar\quad}
\recipient{}{}
\name{Søren}{Espersen}
\mobile{+45 3337 5124}
\email{soren.espersenft.dk}
\position{Fhv. medlem af Folketingets Præsidium{\enskip\cdotp\enskip}Uden for folketingsgrupperne}
\address{}
\photo[circle,noedge,left]{"./party_UFG/Søren_Espersen_profile.jpg"}
\letterdate{\today}
\lettertitle{Søren Espersen - Blå Bog}
\letteropening{}
\letterclosing{}
\letterenclosure[Attached]{Stemme Statistik}
\begin{document}
\makecvheader[R]
\makecvfooter{\today}{\lettertitle{Søren Espersen - Blå Bog}}{}
\makelettertitle
\begin{cvletter}
\lettersection{Baggrund}
Soslashren Espersen, foslashdt 20. juli 1953 i Svenstrup, Himmerland, soslashn af Peder Christian Espersen og Inger Espersen, foslashdt Lund.nbspGift med Yvette Kim Fay Espersen. Parret har boslashrnene Ditte, Anders, Robin, Susie og Peter og boslashrneboslashrnene Daniel, Mai, Josefine, Mathilde og Estrid.



\lettersection{Uddannelse}
\begin{itemize}
\item Journalist, Danmarks Journalisthøjskole,19761980.
\item Teologistudier, Københavns Universitet,19741976.
\item Værnepligt, Søværnet,19731974.
\item Studentereksamen, Hasseris Gymnasium,19701973.
\item Realeksamen, Svenstrup Skole,19671970.
\end{itemize}
\lettersection{Parlamentarisk Karriere}
\subsection*{Parlamentariske Tillidsposter}
\begin{itemize}
\item Beredskabsordfører, hjemmeværnsordfører, ordfører for det danske mindretal, ordfører for det tyske mindretal og veteranordfører20222022.
\item Forsvarsordfører20202022.
\item Formand for Det Udenrigspolitiske Nævn20152019.
\item Politisk næstformand for Dansk Folkeparti20122020.
\item Medlem af Forsvarskommissionen20072009.
\item Næstformand for Færøudvalget20072011.
\item Medlem af Dansk Folkepartis gruppebestyrelse, gruppesekretær20062022.
\item Færøerneordfører og grønlandsordfører20052022.
\item Grundlovsordfører, udenrigsordfører og ordfører for Rigsfællesskabet20052022.
\item Medlem af Nordisk Råd20052006.
\item Medlem af OSCEs Parlamentariske Forsamling20052006.
\item Medlem af den grønlandskdanske selvstyrekommission20052008.
\item nbsp
\end{itemize}
\subsection*{Folketinget}
\subsubsection*{Medlemsperioder}
\begin{itemize}
\item Folketingsmedlem for Uden for folketingsgrupperne i Sjællands Storkreds fra 25. juni 2022.
\item Folketingsmedlem for Dansk Folkeparti i Sjællands Storkreds18. juni 2015  24. juni 2022.
\item Folketingsmedlem for Dansk Folkeparti i Københavns Omegns Storkreds13. november 2007  18. juni 2015.
\item Folketingsmedlem for Dansk Folkeparti i Københavns Amtskreds8. februar 2005  13. november 2007.
\end{itemize}
\subsubsection*{Kandidaturer}
\begin{itemize}
\item Kandidat for Dansk Folkeparti i Kalundborgkredsenfra 2013.
\item Kandidat for Dansk Folkeparti i Ballerupkredsen20072013.
\item Kandidat for Dansk Folkeparti i Hvidovrekredsen20042006.
\item Kandidat for Dansk Folkeparti i Sundbykredsen19972004.
\item Kandidat for Fremskridtspartiet i Præstøkredsen19921995.
\end{itemize}
\subsection*{Folketingets Præsidium}
\begin{itemize}
\item Medlem af Folketingets Præsidium28. november 2007  2. oktober 2012.
\end{itemize}
\lettersection{Erhvervserfaring}
\begin{itemize}
\item Pressechef, Dansk Folkeparti,19972005.
\item Journalist, BilledBladet,19921996.
\item Freelancejournalist, London,19891992.
\item Journalist, BT,19801988.
\item Journalist, Dagbladet, Ringsted,19801980.
\end{itemize}
\lettersection{Publikationer}
Har skrevet raquoIsraels selvstaeligndighedskrig ndash og de danske frivilligelaquo, 2007, raquoValdemar Roslashrdam  Nationalskjald og Landsforraeligderlaquo, 2003 og raquoDanmarks fremtid ndash dit land, dit valghelliplaquo sammen med U. Dahlerup, K. Thulesen Dahl og A. Skjoslashdt, 2001.  Ansvarshavende redaktoslashr af Dansk Folkeblad 19972005.

\end{cvletter}
\end{document}