%!TEX TS-program = xelatex
%!TEX encoding = UTF-8 Unicode
\documentclass[11pt, a4paper]{awesome-cv}
\geometry{left=1.4cm, top=.8cm, right=1.4cm, bottom=1.8cm, footskip=.5cm}
\fontdir[fonts/]
\colorlet{awesome}{EL-colour}
\setbool{acvSectionColorHighlight}{true}
\renewcommand{\acvHeaderSocialSep}{\quad\textbar\quad}
\recipient{}{}
\name{Christian}{Juhl}
\mobile{+45 3337 5005}
\email{christian.juhl@ft.dk}
\position{Medlem af Folketinget{\enskip\cdotp\enskip}Enhedslisten}
\address{}
\photo[circle,noedge,left]{"./party_Enhedslisten/Christian_Juhl_profile.jpg"}
\letterdate{\today}
\lettertitle{Christian Juhl - Blå Bog}
\letteropening{}
\letterclosing{}
\letterenclosure[Attached]{Stemme Statistik}
\begin{document}
\makecvheader[R]
\makecvfooter{\today}{\lettertitle{Christian Juhl - Blå Bog}}{}
\makelettertitle
\begin{cvletter}
\lettersection{Baggrund}
Christian Juhl, født 24. februar 1953 i Hejls, søn af grisehandler Ole Juhl og husmoder Anna Juhl. Gift med socialrådgiver Karen Marie Møller.

\lettersection{Uddannelse}
\begin{itemize}
\item Læreruddannelsen, Silkeborg, 1975-1979.
\item Gymnasium, Haderslev, 1970-1973.
\item Realskole, Christiansfeld, 1967-1970.
\item Folkeskolen, Hejls, 1960-1967.
\end{itemize}
\lettersection{Parlamentarisk Karriere}
\subsection*{Ordførerskaber}
\begin{itemize}
\item Arbejdsmiljøordfører
\item Arktisordfører
\item Færøerneordfører
\item Grønlandsordfører
\item Kirkeordfører
\item Ordfører vedr. nordisk samarbejde
\item Udviklingsordfører
\end{itemize}
\subsection*{Parlamentariske Tillidsposter}
\begin{itemize}
\item Arktisordfører fra 2019.
\item Kirkeordfører fra 2019.
\item Tingssekretær fra 2019.
\item Arbejdsmiljøordfører, ordfører vedrørende nordisk samarbejde, grønlandsordfører og færøerneordfører fra 2015.
\item Medlem af Sydslesvigudvalget (formand fra 2019)  fra 2015.
\item Udviklingsordfører fra 2011.
\item Medlem af kontaktudvalget for Det Tyske Mindretal fra 2011.
\end{itemize}
\subsection*{Folketinget}
\subsubsection*{Medlemsperioder}
\begin{itemize}
\item Folketingsmedlem for Enhedslisten i Sjællands Storkreds fra 15. september 2011.
\end{itemize}
\subsubsection*{Kandidaturer}
\begin{itemize}
\item Kandidat for Enhedslisten i Roskildekredsen fra 2010.
\item Kandidat for Enhedslisten i Aalborg Østkredsen 2006-2010.
\end{itemize}
\subsection*{Folketingets Præsidium}
\begin{itemize}
\item Medlem af Folketingets Præsidium 1. februar 2016 - 5. juni 2019.
\end{itemize}
\lettersection{Erhvervserfaring}
\begin{itemize}
\item Fagforeningsformand, SiD Silkeborg/3F Silkeborg, 1998-2011.
\item Faglig medarbejder, SiD Silkeborg, 1994-1998.
\item Arbejdsmiljøkonsulent, Maler BST, 1990-1994.
\item Arbejdsmiljøkonsulent, SiD Viborg Amt, 1986-1990.
\item Skovarbejder, jord- og betonarbejder, chauffør, Silkeborg, 1979-1986.
\item Skovarbejder, Stursbøl, 1974-1975.
\item Aftjent værnepligt, Tønder, 1973-1974.
\end{itemize}
\end{cvletter}
\end{document}