%!TEX TS-program = xelatex
%!TEX encoding = UTF-8 Unicode
\documentclass[11pt, a4paper]{awesome-cv}
\geometry{left=1.4cm, top=.8cm, right=1.4cm, bottom=1.8cm, footskip=.5cm}
\fontdir[fonts/]
\colorlet{awesome}{EL-colour}
\setbool{acvSectionColorHighlight}{true}
\renewcommand{\acvHeaderSocialSep}{\quad\textbar\quad}
\recipient{}{}
\name{Eva}{Flyvholm}
\mobile{+45 3337 5064}
\email{eva.flyvholmft.dk}
\position{Medlem af Folketinget{\enskip\cdotp\enskip}Enhedslisten}
\address{}
\photo[circle,noedge,left]{"./party_EL/Eva_Flyvholm_profile.jpg"}
\letterdate{\today}
\lettertitle{Eva Flyvholm - Blå Bog}
\letteropening{}
\letterclosing{}
\letterenclosure[Attached]{Stemme Statistik}
\begin{document}
\makecvheader[R]
\makecvfooter{\today}{\lettertitle{Eva Flyvholm - Blå Bog}}{}
\makelettertitle
\begin{cvletter}
\lettersection{Baggrund}
Eva Flyvholm, født 18. april 1981 i Nykøbing Falster, datter af Lars Flyvholm og Bodil Dalbro.

\lettersection{Uddannelse}
\begin{itemize}
\item Cand.scient.adm. i offentlig administration og internationale udviklingsstudier, Roskilde Universitet,20022009.
\item Student, Kalundborg Gymnasium,19972000.
\item Folkeskole, Buerup og Høng,19871997.
\end{itemize}
\lettersection{Parlamentarisk Karriere}
\subsection*{Folketinget}
\subsubsection*{Medlemsperioder}
\begin{itemize}
\item Folketingsmedlem for Enhedslisten i Sjællands Storkreds fra 18. juni 2015.
\end{itemize}
\subsubsection*{Kandidaturer}
\begin{itemize}
\item Kandidat for Enhedslisten i Holbækkredsenfra 2012.
\end{itemize}
\lettersection{Erhvervserfaring}
\begin{itemize}
\item Politisk rådgiver, EU, forsvar og internationale forhold, Enhedslisten,20092015.
\item Underviser, CEVEA og politisk sommerhøjskole,20092014.
\item Studentermedhjælp, Undervisningsministeriet,20062008.
\item Vikar i hjemmeplejen og plejehjem, Betaniahjemmet og Tornved Kommune,20022006.
\item Vejarbejder, Vestsjællands Amts Vejvæsen,20012001.
\end{itemize}
\end{cvletter}
\end{document}