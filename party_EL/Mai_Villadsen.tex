%!TEX TS-program = xelatex
%!TEX encoding = UTF-8 Unicode
\documentclass[11pt, a4paper]{awesome-cv}
\geometry{left=1.4cm, top=.8cm, right=1.4cm, bottom=1.8cm, footskip=.5cm}
\fontdir[fonts/]
\colorlet{awesome}{EL-colour}
\setbool{acvSectionColorHighlight}{true}
\renewcommand{\acvHeaderSocialSep}{\quad\textbar\quad}
\recipient{}{}
\name{Mai}{Villadsen}
\mobile{+45 3337 5089}
\email{mai.villadsen@ft.dk}
\position{Medlem af Folketinget{\enskip\cdotp\enskip}Enhedslisten}
\address{}
\photo[circle,noedge,left]{"./party_Enhedslisten/Mai_Villadsen_profile.jpg"}
\letterdate{\today}
\lettertitle{Mai Villadsen - Blå Bog}
\letteropening{}
\letterclosing{}
\letterenclosure[Attached]{Stemme Statistik}
\begin{document}
\makecvheader[R]
\makecvfooter{\today}{\lettertitle{Mai Villadsen - Blå Bog}}{}
\makelettertitle
\begin{cvletter}
\lettersection{Baggrund}
Mai Villadsen, født 26. december 1991 i Herning, datter af folkeskolelærer Jørn Sloth Andersen og folkeskolelærer Lisbeth Ejby Villadsen.

\lettersection{Uddannelse}
\begin{itemize}
\item Student, Gefion Gymnasium, 2011-2011.
\item Folkeskole, Kibæk Skole, 1997-2007.
\end{itemize}
\lettersection{Parlamentarisk Karriere}
\subsection*{Ordførerskaber}
\begin{itemize}
\item Klimaordfører
\item Miljøordfører
\item Naturordfører
\end{itemize}
\subsection*{Parlamentariske Tillidsposter}
\begin{itemize}
\item Kulturordfører 2019-2020.
\item Ligestillingsordfører 2019-2020.
\item Uddannelsesordfører 2019-2020.
\end{itemize}
\subsection*{Folketinget}
\subsubsection*{Medlemsperioder}
\begin{itemize}
\item Folketingsmedlem for Enhedslisten i Nordsjællands Storkreds fra 5. juni 2019.
\end{itemize}
\subsubsection*{Kandidaturer}
\begin{itemize}
\item Kandidat for Enhedslisten i alle opstillingskredse i Nordsjællands Storkreds fra 2018.
\end{itemize}
\lettersection{Erhvervserfaring}
\begin{itemize}
\item Ekstern projektleder og underviser, DeltagerDanmark, 2015-2017.
\item Politisk rådgiver, Enhedslisten, 2015-2019.
\item Ekstern underviser, CEVEA, 2015-2015.
\item Ungdomskonsulent, HK, 2013-2015.
\end{itemize}
\end{cvletter}
\end{document}