%!TEX TS-program = xelatex
%!TEX encoding = UTF-8 Unicode
\documentclass[11pt, a4paper]{awesome-cv}
\geometry{left=1.4cm, top=.8cm, right=1.4cm, bottom=1.8cm, footskip=.5cm}
\fontdir[fonts/]
\colorlet{awesome}{KF-colour}
\setbool{acvSectionColorHighlight}{true}
\renewcommand{\acvHeaderSocialSep}{\quad\textbar\quad}
\recipient{}{}
\name{Per}{Larsen}
\mobile{+45 3337 4212}
\email{per.larsen@ft.dk}
\position{Medlem af Folketinget{\enskip\cdotp\enskip}Det Konservative Folkeparti}
\address{}
\photo[circle,noedge,left]{"./Per_Larsen_profile.jpg"}
\letterdate{\today}
\lettertitle{Per Larsen - Blå Bog}
\letteropening{}
\letterclosing{}
\letterenclosure[Attached]{Stemme Statistik}
\begin{document}
\makecvheader[R]
\makecvfooter{\today}{\lettertitle{Per Larsen - Blå Bog}}{}
\makelettertitle
\begin{cvletter}
\lettersection{Baggrund}
Per Larsen, født 27. juni 1965 i Jerslev J, søn af gårdejer Rejnar Larsen og sygehjælper Bodil Larsen.

\lettersection{Uddannelse}
\begin{itemize}
\item Agrarøkonom, Den Classenske Agerbrugsskole på Næsgaard, 1987-1989.
\item Landbrugsuddannelse, Lundbæk Landbrugsskole, 1981-1982.
\end{itemize}
\lettersection{Parlamentarisk Karriere}
\subsection*{Ordførerskaber}
\begin{itemize}
\item Forebyggelsesordfører
\item Ældreordfører
\item Fiskeriordfører
\item Fødevareordfører
\item Landbrugsordfører
\item Psykiatriordfører
\item Sundhedsordfører
\end{itemize}
\subsection*{Folketinget}
\subsubsection*{Medlemsperioder}
\begin{itemize}
\item Folketingsmedlem for Det Konservative Folkeparti i Nordjyllands Storkreds fra 5. juni 2019.
\end{itemize}
\subsubsection*{Kandidaturer}
\begin{itemize}
\item Kandidat for Det Konservative Folkeparti i Aalborg Nordkredsen fra 2018.
\end{itemize}
\lettersection{Erhvervserfaring}
\begin{itemize}
\item Produktchef, Hedegaard A/S, Nørresundby, 2008-2019.
\item Produktkonsulent, Vitfoss A/S, Gråsten, 1993-2008.
\item Repræsentant, Korn og Foderstof Kompagniet, Brønderslev, 1989-1993.
\item Beskæftiget i landbrug, 1981-1987.
\end{itemize}
\end{cvletter}
\end{document}