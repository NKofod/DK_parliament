%!TEX TS-program = xelatex
%!TEX encoding = UTF-8 Unicode
\documentclass[11pt, a4paper]{awesome-cv}
\geometry{left=1.4cm, top=.8cm, right=1.4cm, bottom=1.8cm, footskip=.5cm}
\fontdir[fonts/]
\colorlet{awesome}{KF-colour}
\setbool{acvSectionColorHighlight}{true}
\renewcommand{\acvHeaderSocialSep}{\quad\textbar\quad}
\recipient{}{}
\name{Katarina}{Ammitzbøll}
\mobile{+45 3337 4214}
\email{katarina.ammitzboll@ft.dk}
\position{Medlem af Folketinget{\enskip\cdotp\enskip}Det Konservative Folkeparti}
\address{}
\photo[circle,noedge,left]{"./party_Det Konservative Folkeparti/Katarina_Ammitzbøll_profile.jpg"}
\letterdate{\today}
\lettertitle{Katarina Ammitzbøll - Blå Bog}
\letteropening{}
\letterclosing{}
\letterenclosure[Attached]{Stemme Statistik}
\begin{document}
\makecvheader[R]
\makecvfooter{\today}{\lettertitle{Katarina Ammitzbøll - Blå Bog}}{}
\makelettertitle
\begin{cvletter}
\lettersection{Baggrund}
Katarina Ammitzbøll, født 15. juli 1969 i Charlottenlund, datter af civilingeniør Kurt Ammitzbøll og cand.jur. Elisabeth Lawrence. Gift.

\lettersection{Uddannelse}
\begin{itemize}
\item Certifikat i lederskab, Harvard Business School, Monaco og AVT Business School, København, 2009-2010.
\item LLM (master in  international development law), Warwick University, Storbritannien, 2004-2005.
\item Cand.scient.soc (offentlig driftsøkonomi og international udvikling), Roskilde Universitet, 1989-1997.
\end{itemize}
\lettersection{Parlamentarisk Karriere}
\subsection*{Ordførerskaber}
\begin{itemize}
\item Uddannelsesordfører
\item Ungdomsuddannelsesordfører
\item EU-ordfører
\item Forskningsordfører
\item Kommunalordfører
\item Udviklingsordfører
\end{itemize}
\subsection*{Folketinget}
\subsubsection*{Medlemsperioder}
\begin{itemize}
\item Folketingsmedlem for Det Konservative Folkeparti i Københavns Storkreds fra 5. juni 2019.
\end{itemize}
\subsubsection*{Kandidaturer}
\begin{itemize}
\item Kandidat for Det Konservative Folkeparti i Falkonerkredsen fra 2018.
\item Kandidat for Det Konservative Folkeparti i Århus Østkredsen 2014-2015.
\end{itemize}
\lettersection{Erhvervserfaring}
\begin{itemize}
\item Senior Manager, Strategy og Shared Value Sustainability, A.P. Møller Mærsk, 2017-2019.
\item Manager, Risk Advisory, Deloitte, 2016-2017.
\item Senior Manager, Peace Nexus Foundation, Schweiz, 2012-2015.
\item Ministerråd, Danmarks faste Mission ved FN, Genève, 2011-2012.
\item Seniorrådgiver, Udenrigsministeriet, 2007-2011.
\item Vicedirektør, Control Risks, London, 2006-2007.
\item Seniorrådgiver, NATO Allied Command Transformation (ACT), 2006-2006.
\item Forskningskonsulent, Norwegian Institute of International Affairs (NUPI), 2006-2006.
\item Demokratiseringsekspert, European Union, Electoral Commission of Afghanistan, 2005-2005.
\item Vicedirektør, FN's Udviklingsprogram, Afghanistan, 2002-2004.
\item Seniorrådgiver, African Centre for the Constructive Resolution of Disputes, Sydafrika, Burundi og Mozambique, 2002-2002.
\item Særlig rådgiver, FN's Udviklingsprogram (nedrustning af mindre våben), New York, 2002-2002.
\item Seniorrådgiver og koordinator, UN Department of Peacekeeping Operations, Østtimor, 2000-2001.
\item Manager for vidensdeling (governance), FN's Udviklingsprogram, New York, 1999-2000.
\item Policy Specialist (JPO), FN's Udviklingsprogram, Egypten, 1997-1999.
\end{itemize}
\lettersection{Publikationer}
Medforfatter til »Programming for catalytic effects in peacebuilding - a guide«, PeaceNexus, 2012,  »First steps in post-conflict state building: a UNDP-USAID study«, United States Agency for International Development, 2007, »Maximum or minimum? policy options for democratisation initiatives in UN peace operations«, Norwegian Institute of International Affairs, 2007. Bidragsyder til »International crisis management: squaring the circle«, National Defence Academy, Austrian Ministry of Defence and Sports and Geneva Centre for Security Policy, 2011, »Unintended consequences of peace building operations«, United Nations University Press, 2006, »Afghanistan - a development democracy?«, Swedish Committee for Afghanistan, 2005.

\end{cvletter}
\end{document}