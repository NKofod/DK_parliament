%!TEX TS-program = xelatex
%!TEX encoding = UTF-8 Unicode
\documentclass[11pt, a4paper]{awesome-cv}
\geometry{left=1.4cm, top=.8cm, right=1.4cm, bottom=1.8cm, footskip=.5cm}
\fontdir[fonts/]
\colorlet{awesome}{V-colour}
\setbool{acvSectionColorHighlight}{true}
\renewcommand{\acvHeaderSocialSep}{\quad\textbar\quad}
\recipient{}{}
\name{Ellen Trane}{Nørby}
\mobile{+45 3337 4561}
\email{ellen.trane.norby@ft.dk}
\position{Fhv. minister{\enskip\cdotp\enskip}Venstre}
\address{}
\photo[circle,noedge,left]{"./Ellen Trane_Nørby_profile.jpg"}
\letterdate{\today}
\lettertitle{Ellen Trane Nørby - Blå Bog}
\letteropening{}
\letterclosing{}
\letterenclosure[Attached]{Stemme Statistik}
\begin{document}
\makecvheader[R]
\makecvfooter{\today}{\lettertitle{Ellen Trane Nørby - Blå Bog}}{}
\makelettertitle
\begin{cvletter}
\lettersection{Baggrund}
Ellen Trane Nørby, født 1. februar 1980 i Herning, opvokset i Nørre Nissum. Datter af biolog og fhv. borgmester i Lemvig Jørgen Andreas Nørby og arkitekt MAA Merete Nørby. Samlevende, har to børn.

\lettersection{Uddannelse}
\begin{itemize}
\item Cand.mag. i kunsthistorie, Københavns Universitet, 1998-2005.
\item Sidefag i samfundsfag, Københavns Universitet, 1999-2001.
\item Gymnasium, Lemvig, 1995-1998.
\item Folkeskole, Nørre Nissum, 1985-1995.
\end{itemize}
\lettersection{Parlamentarisk Karriere}
\subsection*{Ordførerskaber}
\begin{itemize}
\item Børneordfører
\item Undervisningsordfører
\item Ungdomsuddannelsesordfører
\end{itemize}
\subsection*{Parlamentariske Tillidsposter}
\begin{itemize}
\item Børne- og undervisningsordfører fra 2019.
\item Politisk ordfører 2011-2014.
\item Socialordfører og ligestillingsordfører 2007-2011.
\item Medieordfører 2006-2015.
\item Kulturordfører 2005-2007.
\end{itemize}
\subsection*{Folketinget}
\subsubsection*{Medlemsperioder}
\begin{itemize}
\item Folketingsmedlem for Venstre i Sydjyllands Storkreds fra 13. november 2007.
\item Folketingsmedlem for Venstre i Sønderjyllands Amtskreds 8. februar 2005 - 13. november 2007.
\end{itemize}
\subsubsection*{Kandidaturer}
\begin{itemize}
\item Kandidat for Venstre i Sønderborgkredsen fra 2004.
\end{itemize}
\lettersection{Erhvervserfaring}
\begin{itemize}
\item Politisk medarbejder, Christiansborg, 1999-2004.
\item Diverse studie- og vikarstillinger,.
\end{itemize}
\lettersection{Publikationer}
Medforfatter til bogen »Smag på Europa &ndash; med honning \& chili«, Saxo, 2004. Initiativtager til og redaktør af bogen »Flere med - nye regler«, Forlagskompagniet, 2004. Har bidraget til flere andre bøger. Har skrevet klummer til LO's Ugebrevet A4, det politiske magasin R&AElig;SON og diverse andre blade og magasiner.

\end{cvletter}
\end{document}