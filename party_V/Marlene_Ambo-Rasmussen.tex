%!TEX TS-program = xelatex
%!TEX encoding = UTF-8 Unicode
\documentclass[11pt, a4paper]{awesome-cv}
\geometry{left=1.4cm, top=.8cm, right=1.4cm, bottom=1.8cm, footskip=.5cm}
\fontdir[fonts/]
\colorlet{awesome}{V-colour}
\setbool{acvSectionColorHighlight}{true}
\renewcommand{\acvHeaderSocialSep}{\quad\textbar\quad}
\recipient{}{}
\name{Marlene}{Ambo-Rasmussen}
\mobile{+45 3337 4507}
\email{marlene.ambo-rasmussen@ft.dk}
\position{Medlem af Folketinget{\enskip\cdotp\enskip}Venstre}
\address{}
\photo[circle,noedge,left]{"./Marlene_Ambo-Rasmussen_profile.jpg"}
\letterdate{\today}
\lettertitle{Marlene Ambo-Rasmussen - Blå Bog}
\letteropening{}
\letterclosing{}
\letterenclosure[Attached]{Stemme Statistik}
\begin{document}
\makecvheader[R]
\makecvfooter{\today}{\lettertitle{Marlene Ambo-Rasmussen - Blå Bog}}{}
\makelettertitle
\begin{cvletter}
\lettersection{Baggrund}
Marlene Ambo-Rasmussen, født 14. marts 1986 i Magleby, Langeland, datter af gårdejer Niels August Ambo Rasmussen og fhv. revisorassistent og social- og sundhedsassistent Vita Mie Bruus Rasmussen. Samlevende med murermester Lars Aaholm. Parret har to børn, Aksel og Agnes. 

\lettersection{Uddannelse}
\begin{itemize}
\item Ba i erhvervsjura og erhvervsøkonomi, Syddansk Universitet, 2010-2014.
\item Markedsføringsøkonom, Tietgen Business College, 2008-2010.
\end{itemize}
\lettersection{Parlamentarisk Karriere}
\subsection*{Ordførerskaber}
\begin{itemize}
\item Familieordfører
\item Trivselsordfører
\end{itemize}
\subsection*{Folketinget}
\subsubsection*{Medlemsperioder}
\begin{itemize}
\item Folketingsmedlem for Venstre i Fyns Storkreds fra 5. juni 2019.
\end{itemize}
\subsubsection*{Kandidaturer}
\begin{itemize}
\item Kandidat for Venstre i Odense Vestkredsen fra 2012.
\end{itemize}
\lettersection{Erhvervserfaring}
\begin{itemize}
\item Handicapmedhjælper, Autismetilbuddet Tornhuset, 2010-2017.
\item Lærervikar, Sct. Hans Skole, Odense, 2006-2008.
\end{itemize}
\end{cvletter}
\end{document}