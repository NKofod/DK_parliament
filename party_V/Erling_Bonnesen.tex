%!TEX TS-program = xelatex
%!TEX encoding = UTF-8 Unicode
\documentclass[11pt, a4paper]{awesome-cv}
\geometry{left=1.4cm, top=.8cm, right=1.4cm, bottom=1.8cm, footskip=.5cm}
\fontdir[fonts/]
\colorlet{awesome}{V-colour}
\setbool{acvSectionColorHighlight}{true}
\renewcommand{\acvHeaderSocialSep}{\quad\textbar\quad}
\recipient{}{}
\name{Erling}{Bonnesen}
\mobile{+45 3337 4521}
\email{erling.bonnesen@ft.dk}
\position{Medlem af Folketinget{\enskip\cdotp\enskip}Venstre}
\address{}
\photo[circle,noedge,left]{"./party_V/Erling_Bonnesen_profile.jpg"}
\letterdate{\today}
\lettertitle{Erling Bonnesen - Blå Bog}
\letteropening{}
\letterclosing{}
\letterenclosure[Attached]{Stemme Statistik}
\begin{document}
\makecvheader[R]
\makecvfooter{\today}{\lettertitle{Erling Bonnesen - Blå Bog}}{}
\makelettertitle
\begin{cvletter}
\lettersection{Baggrund}
Erling Bonnesen, født 28. marts 1955.

\lettersection{Uddannelse}
\begin{itemize}
\item Økonomikonsulent og revisoruddannelse, 1980-1989.
\item Militærtjeneste, Den Kongelige Livgarde, 1978-1979.
\item Tekniker af økonomilinjen med ophold i Malmø og Oslo, 1974-1978.
\end{itemize}
\lettersection{Parlamentarisk Karriere}
\subsection*{Ordførerskaber}
\begin{itemize}
\item Dyrevelfærdsordfører
\item Fødevareordfører
\item Landbrugsordfører
\end{itemize}
\subsection*{Parlamentariske Tillidsposter}
\begin{itemize}
\item Gruppesekretær for Venstres folketingsgruppe fra 2015.
\item Tingssekretær fra 2015.
\end{itemize}
\subsection*{Folketinget}
\subsubsection*{Medlemsperioder}
\begin{itemize}
\item Folketingsmedlem for Venstre i Fyns Storkreds fra 13. november 2007.
\item Folketingsmedlem for Venstre i Fyns Amtskreds 7. oktober 2004 - 13. november 2007.
\end{itemize}
\subsubsection*{Kandidaturer}
\begin{itemize}
\item Kandidat for Venstre i Svendborgkredsen fra 2005.
\item Kandidat for Venstre i Middelfartkredsen 2001-2005.
\end{itemize}
\lettersection{Erhvervserfaring}
\begin{itemize}
\item Revisor, fra 1990.
\end{itemize}
\end{cvletter}
\end{document}