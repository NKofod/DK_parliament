%!TEX TS-program = xelatex
%!TEX encoding = UTF-8 Unicode
\documentclass[11pt, a4paper]{awesome-cv}
\geometry{left=1.4cm, top=.8cm, right=1.4cm, bottom=1.8cm, footskip=.5cm}
\fontdir[fonts/]
\colorlet{awesome}{V-colour}
\setbool{acvSectionColorHighlight}{true}
\renewcommand{\acvHeaderSocialSep}{\quad\textbar\quad}
\recipient{}{}
\name{Karen}{Ellemann}
\mobile{+45 3337 4502}
\email{karen.ellemann@ft.dk}
\position{Medlem af Folketinget{\enskip\cdotp\enskip}Venstre}
\address{}
\photo[circle,noedge,left]{"./Karen_Ellemann_profile.jpg"}
\letterdate{\today}
\lettertitle{Karen Ellemann - Blå Bog}
\letteropening{}
\letterclosing{}
\letterenclosure[Attached]{Stemme Statistik}
\begin{document}
\makecvheader[R]
\makecvfooter{\today}{\lettertitle{Karen Ellemann - Blå Bog}}{}
\makelettertitle
\begin{cvletter}
\lettersection{Baggrund}
Karen Ellemann Kloch, født 26. august 1969 i Charlottenlund, datter af fhv. udenrigsminister Uffe Ellemann-Jensen og sekretær Hanne Ellemann-Jensen. Gift med Kresten Kloch. Mor til to børn, født i 1998 og 1999.

\lettersection{Uddannelse}
\begin{itemize}
\item Uddannet lærer, N. Zahles Seminarium, 2002-2004.
\item Student, Holte Gymnasium, 1986-1989.
\end{itemize}
\lettersection{Parlamentarisk Karriere}
\subsection*{Ordførerskaber}
\begin{itemize}
\item Udviklingsordfører
\end{itemize}
\subsection*{Parlamentariske Tillidsposter}
\begin{itemize}
\item Formand for Folketingets Tværpolitiske netværk for seksuel og reproduktiv sundhed og rettigheder fra 2020.
\item Næstformand for Udvalget for Forretningsordenen fra 2019.
\item Udviklingsordfører fra 2019.
\item Formand for Venstres folketingsgruppe 2018-2019.
\item Socialordfører 2014-2015.
\item Folkeskoleordfører, ungdomsuddannelsesordfører 2011-2014.
\item Medlem af Børne- og Undervisningsudvalget, Socialudvalget, Ligestillingsudvalget, Udvalget for Udlændinge- og Integrationspolitik og Nordisk Råd 2011-2015.
\item Ordfører for familiepolitik og udlændinge og integration 2007-2009.
\item Medlem af Socialudvalget, Udenrigsudvalget og Udvalget for Udlændinge- og Integrationspolitik 2007-2009.
\end{itemize}
\subsection*{Folketinget}
\subsubsection*{Medlemsperioder}
\begin{itemize}
\item Folketingsmedlem for Venstre i Københavns Omegns Storkreds fra 13. november 2007.
\end{itemize}
\subsubsection*{Kandidaturer}
\begin{itemize}
\item Kandidat for Venstre i Brøndbykredsen fra 2007.
\end{itemize}
\subsection*{Folketingets Præsidium}
\begin{itemize}
\item Medlem af Folketingets Præsidium fra 13. august 2019.
\end{itemize}
\lettersection{Erhvervserfaring}
\begin{itemize}
\item Lærer, Rungsted Skole, 2003-2007.
\item Selvstændig journalist, 1999-2002.
\item Daglig leder, Dagmar Teatret, 1996-1999.
\item Administrationschef, Brinkmann Kommunikation, 1995-1996.
\item Daglig leder, Scandinavian Models, 1993-1995.
\item Konsulent, The Voice, 1990-1993.
\end{itemize}
\end{cvletter}
\end{document}