%!TEX TS-program = xelatex
%!TEX encoding = UTF-8 Unicode
\documentclass[11pt, a4paper]{awesome-cv}
\geometry{left=1.4cm, top=.8cm, right=1.4cm, bottom=1.8cm, footskip=.5cm}
\fontdir[fonts/]
\colorlet{awesome}{V-colour}
\setbool{acvSectionColorHighlight}{true}
\renewcommand{\acvHeaderSocialSep}{\quad\textbar\quad}
\recipient{}{}
\name{Heidi}{Bank}
\mobile{+45 3337 4530}
\email{heidi.bank@ft.dk}
\position{Medlem af Folketinget{\enskip\cdotp\enskip}Venstre}
\address{}
\photo[circle,noedge,left]{"./Heidi_Bank_profile.jpg"}
\letterdate{\today}
\lettertitle{Heidi Bank - Blå Bog}
\letteropening{}
\letterclosing{}
\letterenclosure[Attached]{Stemme Statistik}
\begin{document}
\makecvheader[R]
\makecvfooter{\today}{\lettertitle{Heidi Bank - Blå Bog}}{}
\makelettertitle
\begin{cvletter}
\lettersection{Baggrund}
Heidi Farsøe Bank, født 1. august 1972 i Hørsholm, datter af ejendomsmægler, valuar og deltidslandmand Niels Farsøe og medhjælpende hustru Hanne Farsøe. Gift med Per Bank. Har to sønner. 

\lettersection{Uddannelse}
\begin{itemize}
\item Bankuddannet, Unibank,.
\item Stx, Frederiksborg Gymnasium,.
\item Teologistudier, Københavns Universitet,.
\end{itemize}
\lettersection{Parlamentarisk Karriere}
\subsection*{Ordførerskaber}
\begin{itemize}
\item Boligordfører
\end{itemize}
\subsection*{Parlamentariske Tillidsposter}
\begin{itemize}
\item Formand for Ligestillingsudvalget fra 2019.
\item Næstformand for Boligudvalget fra 2019.
\item Boligordfører fra 2019.
\end{itemize}
\subsection*{Folketinget}
\subsubsection*{Medlemsperioder}
\begin{itemize}
\item Folketingsmedlem for Venstre i Østjyllands Storkreds fra 5. juni 2019.
\end{itemize}
\subsubsection*{Kandidaturer}
\begin{itemize}
\item Kandidat for Venstre i Århus Østkredsen fra 2018.
\end{itemize}
\lettersection{Erhvervserfaring}
\begin{itemize}
\item Ejendomsmæglerassistent og ejendomsmægler (herunder selvstændig ejendomsmægler), Farsøe, 1999-2014.
\item Sognemedhjælper, Vindinge Kirke, 1997-1999.
\item Bankelev og bankrådgiver, Unibank,.
\end{itemize}
\end{cvletter}
\end{document}