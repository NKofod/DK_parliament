%!TEX TS-program = xelatex
%!TEX encoding = UTF-8 Unicode
\documentclass[11pt, a4paper]{awesome-cv}
\geometry{left=1.4cm, top=.8cm, right=1.4cm, bottom=1.8cm, footskip=.5cm}
\fontdir[fonts/]
\colorlet{awesome}{V-colour}
\setbool{acvSectionColorHighlight}{true}
\renewcommand{\acvHeaderSocialSep}{\quad\textbar\quad}
\recipient{}{}
\name{Bertel}{Haarder}
\mobile{+45 3337 5500}
\email{bertel.haarder@ft.dk}
\position{Fhv. minister{\enskip\cdotp\enskip}Venstre}
\address{}
\photo[circle,noedge,left]{"./party_Venstre/Bertel_Haarder_profile.jpg"}
\letterdate{\today}
\lettertitle{Bertel Haarder - Blå Bog}
\letteropening{}
\letterclosing{}
\letterenclosure[Attached]{Stemme Statistik}
\begin{document}
\makecvheader[R]
\makecvfooter{\today}{\lettertitle{Bertel Haarder - Blå Bog}}{}
\makelettertitle
\begin{cvletter}
\lettersection{Baggrund}
Bertel Geismar Haarder, født 7. september 1944 på Rønshoved Højskole, søn af højskoleforstander Hans Haarder og Agnete Haarder. Gift med Birgitte, født Præstholm. Parret har fire børn.

\lettersection{Uddannelse}
\begin{itemize}
\item Kandidat i statskundskab med speciale i Grundtvigs frihedssyn, Aarhus Universitet, 1970-1970.
\item Legatstudier, USA, 1964-1965.
\item Student, Sønderborg Statsskole, 1964-1964.
\end{itemize}
\lettersection{Parlamentarisk Karriere}
\subsection*{Ordførerskaber}
\begin{itemize}
\item Færøerneordfører
\item Ordfører vedr. nordisk samarbejde
\end{itemize}
\subsection*{Parlamentariske Tillidsposter}
\begin{itemize}
\item Formand for Udenrigsudvalget fra 2020.
\item Præsident for Nordisk Råd 2020-2021.
\item Formand for Kulturudvalget 2019-2020.
\item Formand for den danske delegation til Nordisk Råd fra 2019.
\item Færøordfører fra 2019.
\item Kulturordfører 2017-2020.
\item Nordisk ordfører fra 2016.
\item Formand for den danske delegation til Nordisk Råd 2011-2015.
\item Næstformand for Udvalget for Forretningsordenen 2011-2015.
\item Nordisk ordfører 2011-2014.
\item Præsident for Nordisk Råd 2011-2011.
\item Næstformand og udenrigsordfører i Europa-Parlamentets liberale gruppe, ELDR 1999-2001.
\item Medlem af Europa-Parlamentet (næstformand 1997-1999) 1994-2001.
\item Kulturordfører 1979-1982.
\item Finansordfører 1977-1982.
\item Undervisnings- og forskningsordfører 1977-1982.
\end{itemize}
\subsection*{Folketinget}
\subsubsection*{Medlemsperioder}
\begin{itemize}
\item Folketingsmedlem for Venstre i Sjællands Storkreds fra 15. september 2011.
\item Folketingsmedlem for Venstre i Københavns Omegns Storkreds 13. november 2007 - 15. september 2011.
\item Folketingsmedlem for Venstre i Vestsjællands Amtskreds 8. februar 2005 - 13. november 2007.
\item Folketingsmedlem for Venstre i Københavns Amtskreds 15. februar 1977 - 30. september 1999.
\item Folketingsmedlem for Venstre i Nordjyllands Amtskreds 9. januar 1975 - 15. februar 1977.
\end{itemize}
\subsubsection*{Kandidaturer}
\begin{itemize}
\item Kandidat for Venstre i Grevekredsen 2009-2021.
\item Kandidat for Venstre i Taastrupkredsen 2007-2009.
\item Kandidat for Venstre i Kalundborgkredsen 2002-2006.
\item Kandidat for Venstre i Lyngbykredsen 1977-1999.
\item Kandidat for Venstre i Gladsaxekredsen 1975-1977.
\item Kandidat for Venstre i Sæbykredsen 1974-1975.
\item Kandidat for Venstre i Thistedkredsen 1973-1974.
\end{itemize}
\subsection*{Folketingets Præsidium}
\begin{itemize}
\item Medlem af Folketingets Præsidium 30. september 2011 - 18. juni 2015.
\end{itemize}
\lettersection{Erhvervserfaring}
\begin{itemize}
\item Seminarieadjunkt, Aalborg Seminarium, 1973-1975.
\item Timelærer, Haderslev Statsseminarium, 1971-1973.
\item Højskolelærer, Askov Højskole, 1968-1973.
\end{itemize}
\lettersection{Publikationer}
Har skrevet »Statskollektivisme og Spildproduktion«, 1973, »Institutionernes Tyranni«, 1974, »Den organiserede arbejdsløshed«, 1975, »Danskerne år 2002«, 1977, »Midt i en klynketid«, 1980, »Grænser for politik«, 1990, »Slip friheden løs«, 1990, »Lille land, hvad nu?«, 1994, »Den bløde kynisme«, 1997, »Op mod strømmen - med højskolen i ryggen«, 2012 og »Bertels bedste - sange og fortællinger fra Borgen«, 2018. Medforfatter til Venstres principprogram, 1979, »Kampen om gymnasiet«, 1982, »Ny-liberalismen &ndash; og dens rødder«, 1982, Europa-Parlamentets menneskerettighedsrapport, 1998-1999, og Venstres EU-program, 2001.

\end{cvletter}
\end{document}