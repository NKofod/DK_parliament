%!TEX TS-program = xelatex
%!TEX encoding = UTF-8 Unicode
\documentclass[11pt, a4paper]{awesome-cv}
\geometry{left=1.4cm, top=.8cm, right=1.4cm, bottom=1.8cm, footskip=.5cm}
\fontdir[fonts/]
\colorlet{awesome}{RV-colour}
\setbool{acvSectionColorHighlight}{true}
\renewcommand{\acvHeaderSocialSep}{\quad\textbar\quad}
\recipient{}{}
\name{Kristian}{Hegaard}
\mobile{+45 3337 4705}
\email{kristian.hegaard@ft.dk}
\position{Medlem af Folketinget{\enskip\cdotp\enskip}Radikale Venstre}
\address{}
\photo[circle,noedge,left]{"./party_Radikale Venstre/Kristian_Hegaard_profile.jpg"}
\letterdate{\today}
\lettertitle{Kristian Hegaard - Blå Bog}
\letteropening{}
\letterclosing{}
\letterenclosure[Attached]{Stemme Statistik}
\begin{document}
\makecvheader[R]
\makecvfooter{\today}{\lettertitle{Kristian Hegaard - Blå Bog}}{}
\makelettertitle
\begin{cvletter}
\lettersection{Baggrund}
Kristian W&uuml;rtz Hegaard, født 18. januar 1991 på Frederiksberg, søn af økonom Mikael Hegaard og lektor og jordemoder Hanne Hegaard.

\lettersection{Uddannelse}
\begin{itemize}
\item Cand.jur., Københavns Universitet, 2010-2019.
\end{itemize}
\lettersection{Parlamentarisk Karriere}
\subsection*{Ordførerskaber}
\begin{itemize}
\item Grundlovsordfører
\item Retsordfører
\item Ordfører vedr. § 71-tilsynet 
\end{itemize}
\subsection*{Parlamentariske Tillidsposter}
\begin{itemize}
\item Næstformand for Retsudvalget fra 2019.
\end{itemize}
\subsection*{Folketinget}
\subsubsection*{Medlemsperioder}
\begin{itemize}
\item Folketingsmedlem for Radikale Venstre i Nordsjællands Storkreds fra 5. juni 2019.
\end{itemize}
\subsubsection*{Kandidaturer}
\begin{itemize}
\item Kandidat for Radikale Venstre i Frederikssundkredsen fra 2016.
\item Kandidat for Radikale Venstre i Fredensborgkredsen fra 2013.
\end{itemize}
\end{cvletter}
\end{document}