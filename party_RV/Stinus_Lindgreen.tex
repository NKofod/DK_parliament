%!TEX TS-program = xelatex
%!TEX encoding = UTF-8 Unicode
\documentclass[11pt, a4paper]{awesome-cv}
\geometry{left=1.4cm, top=.8cm, right=1.4cm, bottom=1.8cm, footskip=.5cm}
\fontdir[fonts/]
\colorlet{awesome}{RV-colour}
\setbool{acvSectionColorHighlight}{true}
\renewcommand{\acvHeaderSocialSep}{\quad\textbar\quad}
\recipient{}{}
\name{Stinus}{Lindgreen}
\mobile{+45 3337 4712}
\email{stinus.lindgreen@ft.dk}
\position{Medlem af Folketinget{\enskip\cdotp\enskip}Radikale Venstre}
\address{}
\photo[circle,noedge,left]{"./party_Radikale Venstre/Stinus_Lindgreen_profile.jpg"}
\letterdate{\today}
\lettertitle{Stinus Lindgreen - Blå Bog}
\letteropening{}
\letterclosing{}
\letterenclosure[Attached]{Stemme Statistik}
\begin{document}
\makecvheader[R]
\makecvfooter{\today}{\lettertitle{Stinus Lindgreen - Blå Bog}}{}
\makelettertitle
\begin{cvletter}
\lettersection{Baggrund}
Stinus Lindgreen, født 9. august 1980 i Lyngby-Taarbæk, søn af tandlæge Søren Lindgreen og skolelærer Marit Lindgreen. Gift med Mirjam Theresa Lindgreen. Parret har børnene Thore, født 2008, Asta, født 2011, og Elvi, født 2018.

\lettersection{Uddannelse}
\begin{itemize}
\item Ph.d. i bioinformatik, Københavns Universitet, 2006-2010.
\item Kandidat i bioinformatik, Københavns Universitet, 2003-2006.
\item Bachelor i datalogi, Københavns Universitet, 2000-2003.
\end{itemize}
\lettersection{Parlamentarisk Karriere}
\subsection*{Ordførerskaber}
\begin{itemize}
\item Ældreordfører
\item Etikordfører
\item Forebyggelsesordfører
\item Forskningsordfører
\item Ordfører vedr. nordisk samarbejde
\item Psykiatriordfører
\item Sundhedsordfører
\end{itemize}
\subsection*{Folketinget}
\subsubsection*{Medlemsperioder}
\begin{itemize}
\item Folketingsmedlem for Radikale Venstre i Københavns Omegns Storkreds fra 5. juni 2019.
\end{itemize}
\subsubsection*{Kandidaturer}
\begin{itemize}
\item Kandidat for Radikale Venstre i Lyngbykredsen fra 2016.
\end{itemize}
\lettersection{Erhvervserfaring}
\begin{itemize}
\item Specialist, H. Lundbeck A/S, 2017-2019.
\item Forsker, Steno Diabetes Center Copenhagen, 2016-2017.
\item Forsker, Carlsberg Laboratorierne, 2015-2016.
\item Marie Curie Research Fellow, Københavns Universitet og University of Canterbury, 2013-2015.
\item Postdoc, University of Canterbury, New Zealand, 2012-2013.
\item Forskningsassistent, Københavns Universitet, 2010-2010.
\item Postdoc, Københavns Universitet, 2010-2011.
\item Forskningsassistent, Københavns Universitet, 2006-2006.
\end{itemize}
\lettersection{Publikationer}
Forfatter og medforfatter til en lang række videnskabelige artikler, heriblandt: »Above and belowground community strategies respond to different global change drivers«, Scientific Reports, 2019, »Robust identification of noncoding RNA from transcriptomes requires phylogenetically-informed sampling«, PLoS Computational Biology, 2014, »An Aboriginal Australian Genome Reveals Separate Human Dispersals into Asia«, Science, 2011 og »Ancient human genome sequence of an extinct Palaeo-Eskimo«, Nature, 2010. 

\end{cvletter}
\end{document}