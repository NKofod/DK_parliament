%!TEX TS-program = xelatex
%!TEX encoding = UTF-8 Unicode
\documentclass[11pt, a4paper]{awesome-cv}
\geometry{left=1.4cm, top=.8cm, right=1.4cm, bottom=1.8cm, footskip=.5cm}
\fontdir[fonts/]
\colorlet{awesome}{RV-colour}
\setbool{acvSectionColorHighlight}{true}
\renewcommand{\acvHeaderSocialSep}{\quad\textbar\quad}
\recipient{}{}
\name{Rasmus Helveg}{Petersen}
\mobile{+45 3337 4720}
\email{rasmus.helveg.petersen@ft.dk}
\position{Fhv. minister{\enskip\cdotp\enskip}Radikale Venstre}
\address{}
\photo[circle,noedge,left]{"./party_RV/Rasmus Helveg_Petersen_profile.jpg"}
\letterdate{\today}
\lettertitle{Rasmus Helveg Petersen - Blå Bog}
\letteropening{}
\letterclosing{}
\letterenclosure[Attached]{Stemme Statistik}
\begin{document}
\makecvheader[R]
\makecvfooter{\today}{\lettertitle{Rasmus Helveg Petersen - Blå Bog}}{}
\makelettertitle
\begin{cvletter}
\lettersection{Baggrund}
Rasmus Helveg Petersen, født 17. juni 1968 i København, søn af fhv. MF og fhv. minister Niels Helveg Petersen og cand.mag. og fhv. lektor Hanne Aarup. Gift med Helle Bruun.

\lettersection{Uddannelse}
\begin{itemize}
\item Journalist, Danmarks Journalisthøjskole, Aarhus, 1989-1993.
\end{itemize}
\lettersection{Parlamentarisk Karriere}
\subsection*{Ministerposter}
\begin{itemize}
\item Klima-, energi- og bygningsminister 3. februar 2014 - 28. juni 2015.
\item Udviklingsminister 21. november 2013 - 3. februar 2014.
\end{itemize}
\subsection*{Ordførerskaber}
\begin{itemize}
\item Familieordfører
\item Fiskeriordfører
\item Landdistrikts- og øordfører
\item Socialordfører
\end{itemize}
\subsection*{Parlamentariske Tillidsposter}
\begin{itemize}
\item Formand for Klima-, Energi- og Forsyningsudvalget fra 2020.
\item Næstformand for Radikale Venstres folketingsgruppe 2012-2013.
\end{itemize}
\subsection*{Folketinget}
\subsubsection*{Medlemsperioder}
\begin{itemize}
\item Folketingsmedlem for Radikale Venstre i Fyns Storkreds fra 5. juni 2019.
\item Folketingsmedlem for Radikale Venstre i Sjællands Storkreds 15. september 2011 - 20. juni 2015.
\end{itemize}
\subsubsection*{Kandidaturer}
\begin{itemize}
\item Kandidat for Radikale Venstre i Odense Østkredsen fra 2018.
\item Kandidat for Radikale Venstre i Svendborgkredsen fra 2018.
\item Kandidat for Radikale Venstre i Køgekredsen 2012-2015.
\item Kandidat for Radikale Venstre i Grevekredsen 2008-2012.
\item Kandidat for Radikale Venstre i Holbækkredsen 2008-2015.
\end{itemize}
\lettersection{Erhvervserfaring}
\begin{itemize}
\item Direktør, Dansk Institut for Partier og Demokrati (orlov 2019), 2016-2019.
\item Kommunikationschef, Verdensnaturfonden (WWF), 2009-2011.
\item Kommunikationschef, Gigtforeningen, 2007-2009.
\item Strategirådgiver, Folkekirkens Nødhjælp, 1999-2006.
\item Politisk journalist ved Jyllands-Posten, Information og de radikale blade, Christiansborg, 1995-1999.
\item Kommunikationsmedarbejder, Folketingets EU-Oplysning, 1993-1995.
\end{itemize}
\end{cvletter}
\end{document}