%!TEX TS-program = xelatex
%!TEX encoding = UTF-8 Unicode
\documentclass[11pt, a4paper]{awesome-cv}
\geometry{left=1.4cm, top=.8cm, right=1.4cm, bottom=1.8cm, footskip=.5cm}
\fontdir[fonts/]
\colorlet{awesome}{RV-colour}
\setbool{acvSectionColorHighlight}{true}
\renewcommand{\acvHeaderSocialSep}{\quad\textbar\quad}
\recipient{}{}
\name{Jens}{Rohde}
\mobile{+45 3337 4701}
\email{jens.rohde@ft.dk}
\position{Medlem af Folketinget{\enskip\cdotp\enskip}Radikale Venstre}
\address{}
\photo[circle,noedge,left]{"./party_Radikale Venstre/Jens_Rohde_profile.jpg"}
\letterdate{\today}
\lettertitle{Jens Rohde - Blå Bog}
\letteropening{}
\letterclosing{}
\letterenclosure[Attached]{Stemme Statistik}
\begin{document}
\makecvheader[R]
\makecvfooter{\today}{\lettertitle{Jens Rohde - Blå Bog}}{}
\makelettertitle
\begin{cvletter}
\lettersection{Baggrund}
Jens Rohde, født 18. april 1970 i Holstebro, søn af tidligere koncertmester Ib Rohde og sygehjælper Renate Rohde.

\lettersection{Uddannelse}
\begin{itemize}
\item Statskundskab, Aarhus Universitet, 1994-1995.
\item Student, Viborg Katedralskole, 1989-1989.
\item 9. klasse, Østre Skole, Viborg, 1984-1984.
\end{itemize}
\lettersection{Parlamentarisk Karriere}
\subsection*{Ordførerskaber}
\begin{itemize}
\item Boligordfører
\item EU-ordfører
\item Medieordfører
\end{itemize}
\subsection*{Parlamentariske Tillidsposter}
\begin{itemize}
\item Medlem af Europa-Parlamentet 2009-2019.
\end{itemize}
\subsection*{Folketinget}
\subsubsection*{Medlemsperioder}
\begin{itemize}
\item Folketingsmedlem for Radikale Venstre i Københavns Storkreds fra 5. juni 2019.
\item Folketingsmedlem for Venstre i Søndre Storkreds 8. februar 2005 - 9. januar 2007.
\item Folketingsmedlem for Venstre i Viborg Amtskreds 11. marts 1998 - 7. februar 2005.
\end{itemize}
\subsubsection*{Kandidaturer}
\begin{itemize}
\item Kandidat for Radikale Venstre i Slotskredsen fra 2018.
\item Kandidat for Venstre i Rådhuskredsen 2004-2006.
\item Kandidat for Venstre i Kjellerupkredsen 1997-2004.
\end{itemize}
\subsection*{Folketingets Præsidium}
\begin{itemize}
\item Medlem af Folketingets Præsidium fra 21. juni 2019.
\end{itemize}
\lettersection{Erhvervserfaring}
\begin{itemize}
\item Forlagsejer, Forlaget Kant, fra 2019.
\item CEO, PN1 Evolution, 2008-2010.
\item CEO, TV 2 Radio, 2006-2008.
\item Tv-anmelder, Ekstra Bladet, 1999-2003.
\item Daglig leder, Viborg Handelsstands Forening, 1995-1998.
\item Koncertanmelder og klummeskribent, Viborg Nyt, 1995-1998.
\item Programmedarbejder, Radio Viborg, 1984-1997.
\end{itemize}
\lettersection{Publikationer}
Har skrevet bogen »Tro, håb og ungdomsoprør«, 2019.

\end{cvletter}
\end{document}