%!TEX TS-program = xelatex
%!TEX encoding = UTF-8 Unicode
\documentclass[11pt, a4paper]{awesome-cv}
\geometry{left=1.4cm, top=.8cm, right=1.4cm, bottom=1.8cm, footskip=.5cm}
\fontdir[fonts/]
\colorlet{awesome}{RV-colour}
\setbool{acvSectionColorHighlight}{true}
\renewcommand{\acvHeaderSocialSep}{\quad\textbar\quad}
\recipient{}{}
\name{Sofie Carsten}{Nielsen}
\mobile{+45 3337 5500}
\email{sofie.carsten.nielsen@ft.dk}
\position{Medlem af Folketinget{\enskip\cdotp\enskip}Radikale Venstre}
\address{}
\photo[circle,noedge,left]{"./party_Radikale Venstre/Sofie Carsten_Nielsen_profile.jpg"}
\letterdate{\today}
\lettertitle{Sofie Carsten Nielsen - Blå Bog}
\letteropening{}
\letterclosing{}
\letterenclosure[Attached]{Stemme Statistik}
\begin{document}
\makecvheader[R]
\makecvfooter{\today}{\lettertitle{Sofie Carsten Nielsen - Blå Bog}}{}
\makelettertitle
\begin{cvletter}
\lettersection{Baggrund}
Sofie Carsten Nielsen, født 24. maj 1975 i Birkerød, datter af pensionist Jens Carsten Nielsen og pensionist Kirsten Carsten Nielsen. Gift med Rasmus Dalsgaard. Parret har børnene Gustav og Villads.

\lettersection{Uddannelse}
\begin{itemize}
\item Cand.scient.pol., Københavns Universitet, 1995-2002.
\item Master i europæisk politik og administration, Europakollegiet i Brügge, Belgien, 2000-2001.
\end{itemize}
\lettersection{Parlamentarisk Karriere}
\subsection*{Parlamentariske Tillidsposter}
\begin{itemize}
\item Formand for Radikale Venstres folketingsgruppe fra 2020.
\item Politisk leder for Radikale Venstre fra 2020.
\item Finansordfører 2019-2020.
\item Politisk ordfører 2019-2020.
\item Næstformand for Radikale Venstres folketingsgruppe 2015-2020.
\item Stedfortrædende politisk leder for Radikale Venstre 2014-2020.
\item Formand for Finansudvalget 2012-2014.
\item Formand for Radikale Venstres folketingsgruppe 2012-2014.
\item Næstformand for Radikale Venstres folketingsgruppe 2011-2012.
\item Næstformand for Ligestillingsudvalget 2011-2012.
\end{itemize}
\subsection*{Folketinget}
\subsubsection*{Medlemsperioder}
\begin{itemize}
\item Folketingsmedlem for Radikale Venstre i Københavns Omegns Storkreds fra 15. september 2011.
\end{itemize}
\subsubsection*{Kandidaturer}
\begin{itemize}
\item Kandidat for Radikale Venstre i Gentoftekredsen fra 2010.
\end{itemize}
\lettersection{Erhvervserfaring}
\begin{itemize}
\item Politisk chef, Ingeniørforeningen IDA, 2010-2011.
\item Souschef, Ligestillingsministeriet, 2004-2009.
\item Politisk konsulent, Europa-Parlamentet, 2002-2004.
\end{itemize}
\lettersection{Publikationer}
Bidragsyder til antologien »Tilbage til rødderne - om de radikale værdier«, 2008 og antologien  »De røde sko«, 2002.

\end{cvletter}
\end{document}