%!TEX TS-program = xelatex
%!TEX encoding = UTF-8 Unicode
\documentclass[11pt, a4paper]{awesome-cv}
\geometry{left=1.4cm, top=.8cm, right=1.4cm, bottom=1.8cm, footskip=.5cm}
\fontdir[fonts/]
\colorlet{awesome}{RV-colour}
\setbool{acvSectionColorHighlight}{true}
\renewcommand{\acvHeaderSocialSep}{\quad\textbar\quad}
\recipient{}{}
\name{Martin}{Lidegaard}
\mobile{+45 3337 4716}
\email{martin.lidegaard@ft.dk}
\position{Fhv. minister{\enskip\cdotp\enskip}Radikale Venstre}
\address{}
\photo[circle,noedge,left]{"./party_RV/Martin_Lidegaard_profile.jpg"}
\letterdate{\today}
\lettertitle{Martin Lidegaard - Blå Bog}
\letteropening{}
\letterclosing{}
\letterenclosure[Attached]{Stemme Statistik}
\begin{document}
\makecvheader[R]
\makecvfooter{\today}{\lettertitle{Martin Lidegaard - Blå Bog}}{}
\makelettertitle
\begin{cvletter}
\lettersection{Baggrund}
Martin Lidegaard, født 12. december 1966 i København, søn af højskolelærer og forfatter Mads Lidegaard og journalist og forfatter Else Lidegaard.

\lettersection{Uddannelse}
\begin{itemize}
\item Cand.comm., Roskilde Universitetscenter, 1987-1993.
\item Student, Helsingør Amtsgymnasium, 1982-1985.
\item Humlebæk Kommuneskole, 1973-1982.
\end{itemize}
\lettersection{Parlamentarisk Karriere}
\subsection*{Ministerposter}
\begin{itemize}
\item Udenrigsminister 3. februar 2014 - 28. juni 2015.
\item Klima-, energi- og bygningsminister 3. oktober 2011 - 3. februar 2014.
\end{itemize}
\subsection*{Ordførerskaber}
\begin{itemize}
\item Arktisordfører
\item Færøerneordfører
\item Fiskeriordfører
\item Forsvarsordfører
\item Grønlandsordfører
\item Udenrigsordfører
\end{itemize}
\subsection*{Parlamentariske Tillidsposter}
\begin{itemize}
\item 1. næstformand for Radikale Venstres folketingsgruppe 2021-2021.
\item Formand for Det Udenrigspolitiske Nævn fra 2019.
\item Udenrigsordfører, forsvarsordfører, rigsfællesskabsordfører, grønlandsordfører, færøerneordfører, arktiskordfører og fiskeriordfører.
\end{itemize}
\subsection*{Folketinget}
\subsubsection*{Medlemsperioder}
\begin{itemize}
\item Folketingsmedlem for Radikale Venstre i Nordsjællands Storkreds fra 18. juni 2015.
\item Folketingsmedlem for Radikale Venstre i Roskilde Amtskreds 8. februar 2005 - 13. november 2007.
\item Folketingsmedlem for Radikale Venstre i Vestre Storkreds 20. november 2001 - 8. februar 2005.
\end{itemize}
\subsubsection*{Kandidaturer}
\begin{itemize}
\item Kandidat for Radikale Venstre i Rudersdalkredsen fra 2014.
\item Kandidat for Radikale Venstre i Roskildekredsen 2003-2007.
\item Kandidat for Radikale Venstre i Vesterbrokredsen 2000-2002.
\end{itemize}
\lettersection{Erhvervserfaring}
\begin{itemize}
\item Kommunikationsrådgiver, RelationPeople, 2008-2008.
\item Medstifter af og arbejdende formand for CONCITO, 2008-2011.
\item Informationschef og vicegeneralsekretær, Mellemfolkeligt Samvirke, 1996-2001.
\item Informationsmedarbejder, chef, Kommunernes gensidige Forsikringsselskab, 1993-1996.
\item Freelancekonsulent, Arbejderbevægelsens Internationale Forum, 1992-1993.
\item Redaktør, RUC-nyt, 1988-1992.
\item Medarbejder, Dansk Røde Kors asylcenter, 1986-1987.
\item Assistent, Bikuben, 1985-1985.
\end{itemize}
\end{cvletter}
\end{document}