%!TEX TS-program = xelatex
%!TEX encoding = UTF-8 Unicode
\documentclass[11pt, a4paper]{awesome-cv}
\geometry{left=1.4cm, top=.8cm, right=1.4cm, bottom=1.8cm, footskip=.5cm}
\fontdir[fonts/]
\colorlet{awesome}{RV-colour}
\setbool{acvSectionColorHighlight}{true}
\renewcommand{\acvHeaderSocialSep}{\quad\textbar\quad}
\recipient{}{}
\name{Martin}{Lidegaard}
\mobile{+45 3337 4716}
\email{martin.lidegaardft.dk}
\position{Fhv. minister{\enskip\cdotp\enskip}Radikale Venstre}
\address{}
\photo[circle,noedge,left]{"./party_RV/Martin_Lidegaard_profile.jpg"}
\letterdate{\today}
\lettertitle{Martin Lidegaard - Blå Bog}
\letteropening{}
\letterclosing{}
\letterenclosure[Attached]{Stemme Statistik}
\begin{document}
\makecvheader[R]
\makecvfooter{\today}{\lettertitle{Martin Lidegaard - Blå Bog}}{}
\makelettertitle
\begin{cvletter}
\lettersection{Baggrund}
Martin Lidegaard, født 12. december 1966 i København, søn af højskolelærer og forfatter Mads Lidegaard og journalist og forfatter Else Lidegaard.

\lettersection{Uddannelse}
\begin{itemize}
\item Cand.comm., Roskilde Universitetscenter,19871993.
\item Student, Helsingør Amtsgymnasium,19821985.
\item Humlebæk Kommuneskole,19731982.
\end{itemize}
\lettersection{Parlamentarisk Karriere}
\subsection*{Ministerposter}
\begin{itemize}
\item Udenrigsminister3. februar 2014  28. juni 2015.
\item Klima, energi og bygningsminister3. oktober 2011  3. februar 2014.
\end{itemize}
\subsection*{Ordførerskaber}
\begin{itemize}
\item Arktisordfører
\item Færøerneordfører
\item Fiskeriordfører
\item Forsvarsordfører
\item Grønlandsordfører
\item Udenrigsordfører
\end{itemize}
\subsection*{Parlamentariske Tillidsposter}
\begin{itemize}
\item Politisk leder for Radikale Venstrefra 2022.
\item 1. næstformand for Radikale Venstres folketingsgruppe20212021.
\item Formand for Det Udenrigspolitiske Nævnfra 2019.
\item Fiskeriordførerfra 2019.
\item 2. næstformand for Radikale Venstres folketingsgruppe20152019.
\item Finansordfører20152019.
\item Udenrigsordfører, forsvarsordfører, grønlandsordfører, færøerneordfører og arktisordførerfra 2015.
\item Klimaordfører og socialordfører20052007.
\item Energiordfører, fødevareordfører, fiskeriordfører og trafikordfører20012007.
\end{itemize}
\subsection*{Folketinget}
\subsubsection*{Medlemsperioder}
\begin{itemize}
\item Folketingsmedlem for Radikale Venstre i Nordsjællands Storkreds fra 18. juni 2015.
\item Folketingsmedlem for Radikale Venstre i Nordsjællands Storkredsfra 18. juni 2015.
\item Folketingsmedlem for Radikale Venstre i Roskilde Amtskreds8. februar 2005  13. november 2007.
\item Folketingsmedlem for Radikale Venstre i Vestre Storkreds20. november 2001  8. februar 2005.
\end{itemize}
\subsubsection*{Kandidaturer}
\begin{itemize}
\item Kandidat for Radikale Venstre i Rudersdalkredsenfra 2014.
\item Kandidat for Radikale Venstre i Roskildekredsen20032007.
\item Kandidat for Radikale Venstre i Vesterbrokredsen20002002.
\end{itemize}
\lettersection{Erhvervserfaring}
\begin{itemize}
\item Kommunikationsrådgiver, RelationPeople,20082008.
\item Medstifter af og arbejdende formand for CONCITO,20082011.
\item Informationschef og vicegeneralsekretær, Mellemfolkeligt Samvirke,19962001.
\item Informationsmedarbejder, chef, Kommunernes gensidige Forsikringsselskab,19931996.
\item Freelancekonsulent, Arbejderbevægelsens Internationale Forum,19921993.
\item Redaktør, RUCnyt,19881992.
\item Medarbejder, Dansk Røde Kors asylcenter,19861987.
\item Assistent, Bikuben,19851985.
\end{itemize}
\end{cvletter}
\end{document}