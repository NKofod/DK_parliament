%!TEX TS-program = xelatex
%!TEX encoding = UTF-8 Unicode
\documentclass[11pt, a4paper]{awesome-cv}
\geometry{left=1.4cm, top=.8cm, right=1.4cm, bottom=1.8cm, footskip=.5cm}
\fontdir[fonts/]
\colorlet{awesome}{RV-colour}
\setbool{acvSectionColorHighlight}{true}
\renewcommand{\acvHeaderSocialSep}{\quad\textbar\quad}
\recipient{}{}
\name{Anne Sophie}{Callesen}
\mobile{+45 6162 4120}
\email{anne.sophie.callesen@ft.dk}
\position{Medlem af Folketinget{\enskip\cdotp\enskip}Radikale Venstre}
\address{}
\photo[circle,noedge,left]{"./party_Radikale Venstre/Anne Sophie_Callesen_profile.jpg"}
\letterdate{\today}
\lettertitle{Anne Sophie Callesen - Blå Bog}
\letteropening{}
\letterclosing{}
\letterenclosure[Attached]{Stemme Statistik}
\begin{document}
\makecvheader[R]
\makecvfooter{\today}{\lettertitle{Anne Sophie Callesen - Blå Bog}}{}
\makelettertitle
\begin{cvletter}
\lettersection{Baggrund}
Anne Sophie Callesen, født 25. december 1988 i Aarhus, datter af Lorenz Callesen og Merete Callesen. Gift med Anders Mihle. Har sønnen Frej Mihle Callesen.

\lettersection{Uddannelse}
\begin{itemize}
\item Kandidatuddannelsen i statskundskab, Københavns Universitet, 2013-2015.
\item Master of Arts i Middle East og Mediterranean Studies, Kings's College London, 2012-2013.
\item Bacheloruddannelsen i statskundskab, Københavns Universitet, 2009-2012.
\item Samfundsvidenskabelig student, Aarhus Katedralskole, 2005-2008.
\end{itemize}
\lettersection{Parlamentarisk Karriere}
\subsection*{Ordførerskaber}
\begin{itemize}
\item Integrationsordfører
\item Udviklingsordfører
\item Ungdomsuddannelsesordfører
\end{itemize}
\subsection*{Folketinget}
\subsubsection*{Medlemsperioder}
\begin{itemize}
\item Folketingsmedlem for Radikale Venstre i Østjyllands Storkreds fra 5. juni 2019.
\end{itemize}
\subsubsection*{Kandidaturer}
\begin{itemize}
\item Kandidat for Radikale Venstre i Aarhus Sydkredsen fra 2016.
\end{itemize}
\lettersection{Erhvervserfaring}
\begin{itemize}
\item Specialkonsulent (orlov), Roskilde Universitet, fra 2019.
\item Fuldmægtig, Styrelsen for Undervisning og Kvalitet, 2017-2019.
\item Akkrediteringkonsulent, Danmarks Akkrediteringsinstitution, 2015-2017.
\end{itemize}
\end{cvletter}
\end{document}