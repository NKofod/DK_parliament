%!TEX TS-program = xelatex
%!TEX encoding = UTF-8 Unicode
\documentclass[11pt, a4paper]{awesome-cv}
\geometry{left=1.4cm, top=.8cm, right=1.4cm, bottom=1.8cm, footskip=.5cm}
\fontdir[fonts/]
\colorlet{awesome}{RV-colour}
\setbool{acvSectionColorHighlight}{true}
\renewcommand{\acvHeaderSocialSep}{\quad\textbar\quad}
\recipient{}{}
\name{Marianne}{Jelved}
\mobile{+45 3337 4703}
\email{marianne.jelved@ft.dk}
\position{Fhv. minister{\enskip\cdotp\enskip}Radikale Venstre}
\address{}
\photo[circle,noedge,left]{"./party_RV/Marianne_Jelved_profile.jpg"}
\letterdate{\today}
\lettertitle{Marianne Jelved - Blå Bog}
\letteropening{}
\letterclosing{}
\letterenclosure[Attached]{Stemme Statistik}
\begin{document}
\makecvheader[R]
\makecvfooter{\today}{\lettertitle{Marianne Jelved - Blå Bog}}{}
\makelettertitle
\begin{cvletter}
\lettersection{Baggrund}
Marianne Bruus Jelved, født 5. september 1943 i Charlottenlund, datter af arkitekt Sigurd Hirsbro og Else Hirsbro. Gift med Jan Jelved. 

\lettersection{Uddannelse}
\begin{itemize}
\item Cand.pæd. i dansk, Danmarks Lærerhøjskole, 1979-1979.
\item Lærereksamen, Hellerup Seminarium, 1967-1967.
\item Student, Statens Studenterkursus, Frederiksberg, 1964-1964.
\item Maglegårdsskolen, Charlottenlund, 1950-1960.
\end{itemize}
\lettersection{Parlamentarisk Karriere}
\subsection*{Ministerposter}
\begin{itemize}
\item Kulturminister og kirkeminister 3. februar 2014 - 28. juni 2015.
\item Kulturminister 6. december 2012 - 28. juni 2015.
\item Økonomiminister og minister for nordisk samarbejde 27. september 1994 - 27. november 2001.
\item Økonomiminister 25. januar 1993 - 27. september 1994.
\end{itemize}
\subsection*{Ordførerskaber}
\begin{itemize}
\item Ordfører for frie skoleformer
\item Skoleordfører
\item Folkeoplysningsordfører
\item Kirkeordfører
\end{itemize}
\subsection*{Parlamentariske Tillidsposter}
\begin{itemize}
\item Formand for Radikale Venstres folketingsgruppe 2011-2012.
\item Formand for Udvalget for Videnskab og Teknologi 2007-2011.
\item Formand for Radikale Venstres folketingsgruppe 2001-2007.
\item Medlem af Nordisk Råd 2001-2003.
\item Formand for Radikale Venstres folketingsgruppe 1988-1993.
\item Medlem af Nordisk Råd 1988-1993.
\end{itemize}
\subsection*{Folketinget}
\subsubsection*{Medlemsperioder}
\begin{itemize}
\item Folketingsmedlem for Radikale Venstre i Nordjyllands Storkreds fra 13. november 2007.
\item Folketingsmedlem for Radikale Venstre i Nordjyllands Amtskreds 21. september 1994 - 13. november 2007.
\item Folketingsmedlem for Radikale Venstre i Nordjyllands Amtskreds 12. december 1990 - 1. december 1993.
\item Folketingsmedlem for Radikale Venstre i Roskilde Amtskreds 8. september 1987 - 12. december 1990.
\end{itemize}
\subsubsection*{Kandidaturer}
\begin{itemize}
\item Kandidat for Radikale Venstre i Hjørringkredsen fra 1990.
\item Kandidat for Radikale Venstre i Roskildekredsen 1982-1990.
\end{itemize}
\subsection*{Folketingets Præsidium}
\begin{itemize}
\item Medlem af Folketingets Præsidium 30. september 2011 - 5. december 2012.
\end{itemize}
\lettersection{Erhvervserfaring}
\begin{itemize}
\item Timelærer ved Danmarks Lærerhøjskole, 1979-1987.
\item Lærer ved Jyllinge Skole, Gundsø, 1975-1989.
\item Lærer ved Ny Østensgård Skole, Valby, 1967-1975.
\end{itemize}
\lettersection{Publikationer}
Har skrevet »Alt har sin pris«, 1999, og undervisningsmateriale, bl.a. »Danskbogen 3«, 1988, og »Hvad er meningen? – Folketinget i arbejde i verden«, 2005. Redaktør af »Meddelelser fra Dansklærerforeningen«, 1975-1982, og af »Det er dansk - læseplan og hverdag«, 1984.

\end{cvletter}
\end{document}