%!TEX TS-program = xelatex
%!TEX encoding = UTF-8 Unicode
\documentclass[11pt, a4paper]{awesome-cv}
\geometry{left=1.4cm, top=.8cm, right=1.4cm, bottom=1.8cm, footskip=.5cm}
\fontdir[fonts/]
\colorlet{awesome}{SIU-colour}
\setbool{acvSectionColorHighlight}{true}
\renewcommand{\acvHeaderSocialSep}{\quad\textbar\quad}
\recipient{}{}
\name{Aki-Matilda}{Høegh-Dam}
\mobile{+45 3337 5407}
\email{akimatilda@ft.dk}
\position{Medlem af Folketinget{\enskip\cdotp\enskip}Siumut}
\address{}
\photo[circle,noedge,left]{"./party_Siumut/Aki-Matilda_Høegh-Dam_profile.jpg"}
\letterdate{\today}
\lettertitle{Aki-Matilda Høegh-Dam - Blå Bog}
\letteropening{}
\letterclosing{}
\letterenclosure[Attached]{Stemme Statistik}
\begin{document}
\makecvheader[R]
\makecvfooter{\today}{\lettertitle{Aki-Matilda Høegh-Dam - Blå Bog}}{}
\makelettertitle
\begin{cvletter}
\lettersection{Baggrund}
Aki-Matilda Tilia Ditte Høegh-Dam, født 17. oktober 1996 i Hillerød, datter af fisker og skibsfører Kim Høegh-Dam og folkeskolelærer Bitten Høegh-Dam. Samlevende med Kuno Fencker. 

\lettersection{Uddannelse}
\begin{itemize}
\item Ba i statskundskab, Københavns Universitet, 2015-2019.
\item Almen studentereksamen (stx), Sisimiut og Nuuk, Grønland, 2011-2014.
\item Stud.scient.pol., Københavns Universitet, fra 2019.
\end{itemize}
\lettersection{Parlamentarisk Karriere}
\subsection*{Parlamentariske Tillidsposter}
\begin{itemize}
\item Formand for Grønlandsudvalget fra 2019.
\end{itemize}
\subsection*{Folketinget}
\subsubsection*{Medlemsperioder}
\begin{itemize}
\item Folketingsmedlem for Siumut i Grønland fra 5. juni 2019.
\end{itemize}
\subsubsection*{Kandidaturer}
\begin{itemize}
\item Kandidat for Siumut i Grønland fra 2019.
\end{itemize}
\lettersection{Erhvervserfaring}
\begin{itemize}
\item Gæsteforelæser for DIS (Study Abroad in Scandinavia), København, 2017-2018.
\item Studentermedhjælper for UNICEF, FN Byen, København, 2016-2017.
\item Studentermedhjælper for Siumut i Folketinget, Folketinget, Christiansborg, 2015-2016.
\item Studentermedhjælper for Siumut i Inatsisartut, Inatsisartut, det grønlandske parlament, Nuuk, 2014-2015.
\end{itemize}
\end{cvletter}
\end{document}