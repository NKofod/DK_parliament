%!TEX TS-program = xelatex
%!TEX encoding = UTF-8 Unicode
\documentclass[11pt, a4paper]{awesome-cv}
\geometry{left=1.4cm, top=.8cm, right=1.4cm, bottom=1.8cm, footskip=.5cm}
\fontdir[fonts/]
\colorlet{awesome}{UFG-colour}
\setbool{acvSectionColorHighlight}{true}
\renewcommand{\acvHeaderSocialSep}{\quad\textbar\quad}
\recipient{}{}
\name{Naser}{Khader}
\mobile{+45 3337 5751}
\email{naser.khader@ft.dk}
\position{Medlem af Folketinget{\enskip\cdotp\enskip}Uden for folketingsgrupperne}
\address{}
\photo[circle,noedge,left]{"./party_UFG/Naser_Khader_profile.jpg"}
\letterdate{\today}
\lettertitle{Naser Khader - Blå Bog}
\letteropening{}
\letterclosing{}
\letterenclosure[Attached]{Stemme Statistik}
\begin{document}
\makecvheader[R]
\makecvfooter{\today}{\lettertitle{Naser Khader - Blå Bog}}{}
\makelettertitle
\begin{cvletter}
\lettersection{Baggrund}
Naser Khader, født 1. juli 1963 i Damaskus, søn af ufaglært Ahmed Abu Khader og ufaglært Sada Abu Khader.

\lettersection{Uddannelse}
\begin{itemize}
\item Master i teologi, Københavns Universitet, 2015-2015.
\item Supplerende uddannelse i retorik og formidling, Aarhus Universitet, 1994-1995.
\item Mellemøststudier, Odense Universitet, 1991-1993.
\item Cand.polit, Københavns Universitet, 1986-1993.
\item Rysensteen Gymnasium, 1980-1983.
\item Oehlenschlægersgades Skole, 1975-1980.
\end{itemize}
\lettersection{Parlamentarisk Karriere}
\subsection*{Parlamentariske Tillidsposter}
\begin{itemize}
\item Formand for Forsvarsudvalget fra 2018.
\item Næstformand for Udenrigsudvalget 2015-2018.
\end{itemize}
\subsection*{Folketinget}
\subsubsection*{Medlemsperioder}
\begin{itemize}
\item Folketingsmedlem for Uden for folketingsgrupperne i Sjællands Storkreds fra 19. august 2021.
\item Folketingsmedlem for Det Konservative Folkeparti i Sjællands Storkreds 5. juni 2019 - 18. august 2021.
\item Folketingsmedlem for Det Konservative Folkeparti i Østjyllands Storkreds 18. juni 2015 - 5. juni 2019.
\item Folketingsmedlem for Det Konservative Folkeparti i Københavns Storkreds 18. marts 2009 - 15. september 2011.
\item Folketingsmedlem for Uden for folketingsgrupperne i Københavns Storkreds 5. januar 2009 - 17. marts 2009.
\item Folketingsmedlem for Liberal Alliance i Københavns Storkreds 28. august 2008 - 4. januar 2009.
\item Folketingsmedlem for Ny Alliance i Københavns Storkreds 13. november 2007 - 27. august 2008.
\item Folketingsmedlem for Ny Alliance i Østre Storkreds 10. juli 2007 - 13. november 2007.
\item Folketingsmedlem for Uden for folketingsgrupperne i Østre Storkreds 9. maj 2007 - 9. juli 2007.
\item Folketingsmedlem for Radikale Venstre i Østre Storkreds 20. november 2001 - 8. maj 2007.
\end{itemize}
\subsubsection*{Kandidaturer}
\begin{itemize}
\item Kandidat for Det Konservative Folkeparti i Ringstedkredsen 2018-2021.
\item Kandidat for Det Konservative Folkeparti i Roskildekredsen 2018-2021.
\item Kandidat for Det Konservative Folkeparti i Aarhus Sydkredsen 2015-2018.
\item Kandidat for Det Konservative Folkeparti i Skanderborgkredsen 2015-2018.
\item Kandidat for Det Konservative Folkeparti i Rødovrekredsen 2010-2011.
\item Kandidat for Ny Alliance i alle opstillingskredse i Københavns Storkreds 2007-2009.
\item Kandidat for Radikale Venstre i Brønshøjkredsen 2000-2006.
\item Kandidat for Radikale Venstre i Vestre Storkreds 1994-1999.
\end{itemize}
\lettersection{Erhvervserfaring}
\begin{itemize}
\item Adjunct fellow, Hudson Institute, Washington DC, fra 2017.
\item Non-resident senior fellow, Hudson Institute, Washington DC, 2015-2017.
\item Senior fellow, Hudson Institute, Washington DC, 2011-2015.
\item Konsulent, boligselskabet DAB, 1996-1997.
\item Konsulent, DR, 1989-1997.
\item Arabisk tolk, oversætter og tekster, 1983-1998.
\end{itemize}
\lettersection{Publikationer}
Har skrevet »Den duftende have«, 2019, »Hjertet bløder - arabisk forår og opløsning«, 2015, »Bekendelser fra en kulturkristen muslim», 2013, »Naser Khader og folkestyret«, 2005, »Tro mod tro« med Kathrine Lilleør, 2005, »Modsætninger mødes« med Bent Melchior, 2003, »Nasers Brevkasse«, 2001, »Khader.dk« med Jakob Kvist, 2000 and »&AElig;re og Skam«, 1996.  

\end{cvletter}
\end{document}