%!TEX TS-program = xelatex
%!TEX encoding = UTF-8 Unicode
\documentclass[11pt, a4paper]{awesome-cv}
\geometry{left=1.4cm, top=.8cm, right=1.4cm, bottom=1.8cm, footskip=.5cm}
\fontdir[fonts/]
\colorlet{awesome}{UFG-colour}
\setbool{acvSectionColorHighlight}{true}
\renewcommand{\acvHeaderSocialSep}{\quad\textbar\quad}
\recipient{}{}
\name{Naser}{Khader}
\mobile{+45 3337 5751}
\email{naser.khaderft.dk}
\position{Medlem af Folketinget{\enskip\cdotp\enskip}Uden for folketingsgrupperne}
\address{}
\photo[circle,noedge,left]{"./party_UFG/Naser_Khader_profile.jpg"}
\letterdate{\today}
\lettertitle{Naser Khader - Blå Bog}
\letteropening{}
\letterclosing{}
\letterenclosure[Attached]{Stemme Statistik}
\begin{document}
\makecvheader[R]
\makecvfooter{\today}{\lettertitle{Naser Khader - Blå Bog}}{}
\makelettertitle
\begin{cvletter}
\lettersection{Baggrund}
Naser Khader, født 1. juli 1963 i Damaskus, søn af ufaglært Ahmed Abu Khader og ufaglært Sada Abu Khader.

\lettersection{Uddannelse}
\begin{itemize}
\item Master i teologi, Københavns Universitet,20152015.
\item Supplerende uddannelse i retorik og formidling, Aarhus Universitet,19941995.
\item Mellemøststudier, Odense Universitet,19911993.
\item Cand.polit, Københavns Universitet,19861993.
\item Rysensteen Gymnasium,19801983.
\item Oehlenschlægersgades Skole,19751980.
\end{itemize}
\lettersection{Parlamentarisk Karriere}
\subsection*{Parlamentariske Tillidsposter}
\begin{itemize}
\item Formand for Forsvarsudvalget20182021.
\item Næstformand for Udenrigsudvalget20152018.
\end{itemize}
\subsection*{Folketinget}
\subsubsection*{Medlemsperioder}
\begin{itemize}
\item Folketingsmedlem for Uden for folketingsgrupperne i Sjællands Storkreds fra 19. august 2021.
\item Folketingsmedlem for Det Konservative Folkeparti i Sjællands Storkreds5. juni 2019  18. august 2021.
\item Folketingsmedlem for Det Konservative Folkeparti i Østjyllands Storkreds18. juni 2015  5. juni 2019.
\item Folketingsmedlem for Det Konservative Folkeparti i Københavns Storkreds18. marts 2009  15. september 2011.
\item Folketingsmedlem for Uden for folketingsgrupperne i Københavns Storkreds5. januar 2009  17. marts 2009.
\item Folketingsmedlem for Liberal Alliance i Københavns Storkreds28. august 2008  4. januar 2009.
\item Folketingsmedlem for Ny Alliance i Københavns Storkreds13. november 2007  27. august 2008.
\item Folketingsmedlem for Ny Alliance i Østre Storkreds10. juli 2007  13. november 2007.
\item Folketingsmedlem for Uden for folketingsgrupperne i Østre Storkreds9. maj 2007  9. juli 2007.
\item Folketingsmedlem for Radikale Venstre i Østre Storkreds20. november 2001  8. maj 2007.
\end{itemize}
\subsubsection*{Kandidaturer}
\begin{itemize}
\item Kandidat for Det Konservative Folkeparti i Ringstedkredsen20182021.
\item Kandidat for Det Konservative Folkeparti i Roskildekredsen20182021.
\item Kandidat for Det Konservative Folkeparti i Aarhus Sydkredsen20152018.
\item Kandidat for Det Konservative Folkeparti i Skanderborgkredsen20152018.
\item Kandidat for Det Konservative Folkeparti i Rødovrekredsen20102011.
\item Kandidat for Ny Alliance i alle opstillingskredse i Københavns Storkreds20072009.
\item Kandidat for Radikale Venstre i Brønshøjkredsen20002006.
\item Kandidat for Radikale Venstre i Vestre Storkreds19941999.
\end{itemize}
\lettersection{Erhvervserfaring}
\begin{itemize}
\item Adjunct fellow, Hudson Institute, Washington DC,fra 2017.
\item Nonresident senior fellow, Hudson Institute, Washington DC,20152017.
\item Senior fellow, Hudson Institute, Washington DC,20112015.
\item Konsulent, boligselskabet DAB,19961997.
\item Konsulent, DR,19891997.
\item Arabisk tolk, oversætter og tekster,19831998.
\end{itemize}
\lettersection{Publikationer}
Har skrevet raquoDen duftende havelaquo, 2019, raquoHjertet bloslashder  arabisk foraringr og oploslashsninglaquo, 2015, raquoBekendelser fra en kulturkristen muslimraquo, 2013, raquoNaser Khader og folkestyretlaquo, 2005, raquoTro mod trolaquo med Kathrine Lilleoslashr, 2005, raquoModsaeligtninger moslashdeslaquo med Bent Melchior, 2003, raquoNasers Brevkasselaquo, 2001, raquoKhader.dklaquo med Jakob Kvist, 2000 and raquoAEligre og Skamlaquo, 1996.nbspnbsp

\end{cvletter}
\end{document}