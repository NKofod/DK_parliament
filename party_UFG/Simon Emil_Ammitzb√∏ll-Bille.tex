%!TEX TS-program = xelatex
%!TEX encoding = UTF-8 Unicode
\documentclass[11pt, a4paper]{awesome-cv}
\geometry{left=1.4cm, top=.8cm, right=1.4cm, bottom=1.8cm, footskip=.5cm}
\fontdir[fonts/]
\colorlet{awesome}{UFG-colour}
\setbool{acvSectionColorHighlight}{true}
\renewcommand{\acvHeaderSocialSep}{\quad\textbar\quad}
\recipient{}{}
\name{Simon Emil}{Ammitzbøll-Bille}
\mobile{+45 3337 4903}
\email{simon.ammitzboll@ft.dk}
\position{Fhv. minister{\enskip\cdotp\enskip}Uden for folketingsgrupperne}
\address{}
\photo[circle,noedge,left]{"./party_Uden for folketingsgrupperne/Simon Emil_Ammitzbøll-Bille_profile.jpg"}
\letterdate{\today}
\lettertitle{Simon Emil Ammitzbøll-Bille - Blå Bog}
\letteropening{}
\letterclosing{}
\letterenclosure[Attached]{Stemme Statistik}
\begin{document}
\makecvheader[R]
\makecvfooter{\today}{\lettertitle{Simon Emil Ammitzbøll-Bille - Blå Bog}}{}
\makelettertitle
\begin{cvletter}
\lettersection{Baggrund}
Simon Emil Ammitzbøll-Bille, født 20. oktober 1977 i Hillerød, søn af folkepensionist Nils Ammitzbøll Henriksen og pensioneret socialrådgiver Annie Henriksen. Gift med Kristine Rishøj Ammitzbøll-Bille og far til Lili.

\lettersection{Uddannelse}
\begin{itemize}
\item Bachelor i socialvidenskab, Roskilde Universitetscenter, 2000-2003.
\item Højskoleophold, Krogerup Højskole, 1999-1999.
\item Student, Marie Kruses Skole, 1993-1996.
\item Grundskole, Marie Kruses Skole, 1983-1993.
\end{itemize}
\lettersection{Parlamentarisk Karriere}
\subsection*{Parlamentariske Tillidsposter}
\begin{itemize}
\item Integrationsordfører 2015-2016.
\item Sundhedsordfører 2013-2015.
\item Integrationsordfører 2012-2013.
\item Formand for Ligestillingsudvalget 2011-2012.
\item Formand for Liberal Alliances folketingsgruppe 2010-2016.
\item Politisk ordfører, grundlovsordfører og retsordfører 2009-2016.
\item Kulturordfører og idrætsordfører 2009-2013.
\item Ligestillingsordfører 2009-2012.
\end{itemize}
\subsection*{Folketinget}
\subsubsection*{Medlemsperioder}
\begin{itemize}
\item Folketingsmedlem for Uden for folketingsgrupperne i Københavns Storkreds fra 23. oktober 2019.
\item Folketingsmedlem for Liberal Alliance i Københavns Storkreds 15. september 2011 - 22. oktober 2019.
\item Folketingsmedlem for Liberal Alliance i Sjællands Storkreds 16. juni 2009 - 15. september 2011.
\item Folketingsmedlem for Uden for folketingsgrupperne i Sjællands Storkreds 8. oktober 2008 - 15. juni 2009.
\item Folketingsmedlem for Radikale Venstre i Sjællands Storkreds 13. november 2007 - 7. oktober 2008.
\item Folketingsmedlem for Radikale Venstre i Storstrøms Amtskreds 8. februar 2005 - 13. november 2007.
\end{itemize}
\subsubsection*{Kandidaturer}
\begin{itemize}
\item Kandidat for Liberal Alliance i alle opstillingskredse i Københavns Storkreds 2009-2019.
\item Kandidat for Radikale Venstre i Vordingborgkredsen 2004-2006.
\item Kandidat for Radikale Venstre i Næstvedkredsen 2003-2008.
\end{itemize}
\lettersection{Erhvervserfaring}
\begin{itemize}
\item Redaktør af Radikal Politik, 2004-2005.
\end{itemize}
\lettersection{Publikationer}
Redaktør sammen med Erik Boel af antologien »Europa i alle palettens farver &ndash; 11 essays om europæiske værdier«, 2007 og redaktør af jubilæumsbogen »Tidens Tanker«, 2001.

\end{cvletter}
\end{document}