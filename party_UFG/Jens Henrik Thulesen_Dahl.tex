%!TEX TS-program = xelatex
%!TEX encoding = UTF-8 Unicode
\documentclass[11pt, a4paper]{awesome-cv}
\geometry{left=1.4cm, top=.8cm, right=1.4cm, bottom=1.8cm, footskip=.5cm}
\fontdir[fonts/]
\colorlet{awesome}{UFG-colour}
\setbool{acvSectionColorHighlight}{true}
\renewcommand{\acvHeaderSocialSep}{\quad\textbar\quad}
\recipient{}{}
\name{Jens Henrik Thulesen}{Dahl}
\mobile{+45 6162 5159}
\email{jens.henrik.thulesen.dahlft.dk}
\position{Medlem af Folketinget{\enskip\cdotp\enskip}Uden for folketingsgrupperne}
\address{}
\photo[circle,noedge,left]{"./party_UFG/Jens Henrik Thulesen_Dahl_profile.jpg"}
\letterdate{\today}
\lettertitle{Jens Henrik Thulesen Dahl - Blå Bog}
\letteropening{}
\letterclosing{}
\letterenclosure[Attached]{Stemme Statistik}
\begin{document}
\makecvheader[R]
\makecvfooter{\today}{\lettertitle{Jens Henrik Thulesen Dahl - Blå Bog}}{}
\makelettertitle
\begin{cvletter}
\lettersection{Baggrund}
Jens Henrik Winther Thulesen Dahl, foslashdt 20. juli 1961 i Braeligdstrup, soslashn af afdoslashde overlaeligrer Anders Thulesen Dahl og pensioneret overlaeligrer Inge Margrethe Simoni Dahl.nbspGift med Helle Winther Dahl. Har tre soslashnner.

\lettersection{Uddannelse}
\begin{itemize}
\item Diplomuddannelse i ledelse, VIA University College,20092009.
\item Voksenpædagogisk grundkursus,19961996.
\item Civilingeniør, Aalborg Universitet,19851985.
\item Nysproglig student, Rosborg Gymnasium,19801980.
\end{itemize}
\lettersection{Parlamentarisk Karriere}
\subsection*{Parlamentariske Tillidsposter}
\begin{itemize}
\item Handicapordfører, ligestillingsordfører, psykiatriordfører og sundhedsordfører20222022.
\item STUordfører20222022.
\item Tingsekretærfra 2022.
\item Kommunalordfører20192022.
\item Næstformand for Undervisningsudvalget20162019.
\item Næstformand for Børne og Undervisningsudvalget20152016.
\item Forebyggelsesordfører20112015.
\item Uddannelses og forskningsordfører20112022.
\end{itemize}
\subsection*{Folketinget}
\subsubsection*{Medlemsperioder}
\begin{itemize}
\item Folketingsmedlem for Uden for folketingsgrupperne i Fyns Storkreds fra 25. juni 2022.
\item Folketingsmedlem for Dansk Folkeparti i Fyns Storkreds15. september 2011  24. juni 2022.
\end{itemize}
\subsubsection*{Kandidaturer}
\begin{itemize}
\item Kandidat for Dansk Folkeparti i Assenskredsen20102022.
\end{itemize}
\lettersection{Erhvervserfaring}
\begin{itemize}
\item Konsulentopgaver, eget konsulentfirma,fra 2008.
\item Arbejdsmiljøkonsulent, hrafdelingen, borgmesterens afdeling, Aarhus Kommune,20082011.
\item Undervisning i indeklima og støj, Ergo og Fysioterapiskolen, Aarhus,20012011.
\item Souschef, Aarhus Kommunes BST,20002006.
\item Konstitueret leder, Aarhus Kommunes BST,19991999.
\item Arbejdsmiljøkonsulent, Udviklingshuset, Aarhus Kommune tidligere Aarhus Kommunes BST,19942007.
\item Projektingeniør og VKOkonsulent, Carl Bro AS,19891994.
\item Ekstern undervisningsassistent, Institut for Bygningsteknik, Aalborg Universitet,19861991.
\item Fagingeniør i VVSafdelingen, rådgivende ingeniørfirma Nellemann, Aalborg,19851989.
\end{itemize}
\end{cvletter}
\end{document}