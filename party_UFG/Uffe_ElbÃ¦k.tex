%!TEX TS-program = xelatex
%!TEX encoding = UTF-8 Unicode
\documentclass[11pt, a4paper]{awesome-cv}
\geometry{left=1.4cm, top=.8cm, right=1.4cm, bottom=1.8cm, footskip=.5cm}
\fontdir[fonts/]
\colorlet{awesome}{UFG-colour}
\setbool{acvSectionColorHighlight}{true}
\renewcommand{\acvHeaderSocialSep}{\quad\textbar\quad}
\recipient{}{}
\name{Uffe}{Elbæk}
\mobile{+45 3337 5721}
\email{uffe.elbaek@ft.dk}
\position{Fhv. minister{\enskip\cdotp\enskip}Uden for folketingsgrupperne}
\address{}
\photo[circle,noedge,left]{"./party_UFG/Uffe_Elbæk_profile.jpg"}
\letterdate{\today}
\lettertitle{Uffe Elbæk - Blå Bog}
\letteropening{}
\letterclosing{}
\letterenclosure[Attached]{Stemme Statistik}
\begin{document}
\makecvheader[R]
\makecvfooter{\today}{\lettertitle{Uffe Elbæk - Blå Bog}}{}
\makelettertitle
\begin{cvletter}
\lettersection{Baggrund}
Uffe Elbæk, født 15. juni 1954 i Ry.

\lettersection{Uddannelse}
\begin{itemize}
\item Journalistisk tillægsuddannelse, Danmarks Journalisthøjskole, 1987-1987.
\item Socialpædagog, Peter Sabroe Seminariet, 1982-1982.
\item Studentereksamen, Aalborg Katedralskole, 1975-1975.
\end{itemize}
\lettersection{Parlamentarisk Karriere}
\subsection*{Ministerposter}
\begin{itemize}
\item Kulturminister 3. oktober 2011 - 6. december 2012.
\end{itemize}
\subsection*{Parlamentariske Tillidsposter}
\begin{itemize}
\item Initiativtager til Alternativet 2013 og politisk leder af Alternativet 2013-2020.
\end{itemize}
\subsection*{Folketinget}
\subsubsection*{Medlemsperioder}
\begin{itemize}
\item Folketingsmedlem for Uden for folketingsgrupperne i Københavns Storkreds fra 10. marts 2020.
\item Folketingsmedlem for Alternativet i Københavns Storkreds 13. marts 2015 - 9. marts 2020.
\item Folketingsmedlem for Uden for folketingsgrupperne i Københavns Storkreds 17. september 2013 - 12. marts 2015.
\item Folketingsmedlem for Radikale Venstre i Københavns Storkreds 15. september 2011 - 16. september 2013.
\end{itemize}
\subsubsection*{Kandidaturer}
\begin{itemize}
\item Kandidat for Alternativet i alle opstillingskredse i Københavns Storkreds fra 2015.
\item Kandidat for Radikale Venstre i Sundbyvesterkredsen 2009-2013.
\end{itemize}
\lettersection{Erhvervserfaring}
\begin{itemize}
\item Grundlægger af strategifirmaet Change the Game, 2010-2010.
\item Direktør, World Outgames, 2007-2009.
\item Grundlægger og rektor, KaosPiloterne, 1991-2006.
\item Grundlægger og leder, Frontløberne, 1982-1991.
\item Journalist og klummeskribent, Dagbladet Information, Månedsmagasinet Press, Weekendavisen, Berlingske Tidende, Ugebrevet Mandag Morgen, Dagbladet Dagen og Morgenavisen Jyllands-Posten,.
\end{itemize}
\lettersection{Publikationer}
Har skrevet »Ledelse på kanten«, 2010, »KaosPilot fra A til Z«, 2001 og 2006, »KaosPilot A&ndash;Z« (engelsk udgave), 2003, og »KaosPilot &ndash; en personlig beretning om en skole, en uddannelse og et miljø«, 1998. Har bidraget til »Robustness« af Mika Aaltonen, 2010, »Den Store Danske Encyklopædi«, 2002, »Opbrud på midten«, 2002, »Kompetence Guldet«, 2000, »Fra id&eacute; til projekt - Håndbog om Kulturprojekter«, 1998, og »Kulturliv &ndash; en håndbog«, 1991.

\end{cvletter}
\end{document}