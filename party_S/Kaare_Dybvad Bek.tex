%!TEX TS-program = xelatex
%!TEX encoding = UTF-8 Unicode
\documentclass[11pt, a4paper]{awesome-cv}
\geometry{left=1.4cm, top=.8cm, right=1.4cm, bottom=1.8cm, footskip=.5cm}
\fontdir[fonts/]
\colorlet{awesome}{S-colour}
\setbool{acvSectionColorHighlight}{true}
\renewcommand{\acvHeaderSocialSep}{\quad\textbar\quad}
\recipient{}{}
\name{Kaare}{Dybvad Bek}
\mobile{+45 7228 2400}
\email{min@im.dk}
\position{Indenrigs- og boligminister{\enskip\cdotp\enskip}Socialdemokratiet}
\address{}
\photo[circle,noedge,left]{"./party_S/Kaare_Dybvad Bek_profile.jpg"}
\letterdate{\today}
\lettertitle{Kaare Dybvad Bek - Blå Bog}
\letteropening{}
\letterclosing{}
\letterenclosure[Attached]{Stemme Statistik}
\begin{document}
\makecvheader[R]
\makecvfooter{\today}{\lettertitle{Kaare Dybvad Bek - Blå Bog}}{}
\makelettertitle
\begin{cvletter}
\lettersection{Baggrund}
Kaare Dybvad Bek, født 5. august 1984 i Holbæk, søn af automekaniker Jens Juul Dybvad Olesen og sygeplejerske Dorte Simonsen. Gift med Maiken Bek. Parret har sønnerne Aksel og Pelle.

\lettersection{Uddannelse}
\begin{itemize}
\item Cand.scient. i geografi og geoinformatik, Københavns Universitet, 2009-2012.
\item Bachelor i geografi/teksam, RUC, 2005-2008.
\item Højskole, Vinterskolen, Helsingør, 2005-2005.
\item Sproglig student, Amtsgymnasiet i Roskilde, 2001-2004.
\item 10. klasse, Eisbjerghus Efterskole, Nørre Aaby, 2000-2001.
\item Folkeskole, Vipperød Skole, 1990-2000.
\end{itemize}
\lettersection{Parlamentarisk Karriere}
\subsection*{Ministerposter}
\begin{itemize}
\item Indenrigs- og boligminister fra 21. januar 2021.
\item Boligminister 27. juni 2019 - 21. januar 2021.
\end{itemize}
\subsection*{Folketinget}
\subsubsection*{Medlemsperioder}
\begin{itemize}
\item Folketingsmedlem for Socialdemokratiet i Sjællands Storkreds fra 18. juni 2015.
\end{itemize}
\subsubsection*{Kandidaturer}
\begin{itemize}
\item Kandidat for Socialdemokratiet i Holbækkredsen fra 2010.
\end{itemize}
\lettersection{Erhvervserfaring}
\begin{itemize}
\item Projektchef for videnprojekter, Væksthus Sjælland, 2013-2015.
\item Projektleder, Væksthus Sjælland, 2012-2013.
\item Studentermedarbejder, By og Havn, 2008-2011.
\item Studentermedarbejder for Magnus Heunicke, Socialdemokratiet, 2006-2008.
\item Rengøringsmedarbejder, Superfos, Vipperød, 2005-2006.
\item Lagermedarbejder, LEMAN, Kastrup Lufthavn, 2004-2005.
\item Butiksassistent, Holbæk Fiskehus, 2001-2004.
\end{itemize}
\lettersection{Publikationer}
Har skrevet »De lærdes tyranni &ndash; hvordan den kreative klasse skaber ulighed og undergraver verdens bedste samfund«, People&rsquo;s Press, 2017, og »Udkantsmyten &ndash; hvordan centraliseringen af Danmark ødelægger vores økonomi og sociale sammenhængskraft«, People&rsquo;s Press, 2015. 

\end{cvletter}
\end{document}