%!TEX TS-program = xelatex
%!TEX encoding = UTF-8 Unicode
\documentclass[11pt, a4paper]{awesome-cv}
\geometry{left=1.4cm, top=.8cm, right=1.4cm, bottom=1.8cm, footskip=.5cm}
\fontdir[fonts/]
\colorlet{awesome}{S-colour}
\setbool{acvSectionColorHighlight}{true}
\renewcommand{\acvHeaderSocialSep}{\quad\textbar\quad}
\recipient{}{}
\name{Jesper}{Petersen}
\mobile{+45 3337 4005}
\email{jesper.petersen@ft.dk}
\position{Medlem af Folketinget{\enskip\cdotp\enskip}Socialdemokratiet}
\address{}
\photo[circle,noedge,left]{"./party_Socialdemokratiet/Jesper_Petersen_profile.jpg"}
\letterdate{\today}
\lettertitle{Jesper Petersen - Blå Bog}
\letteropening{}
\letterclosing{}
\letterenclosure[Attached]{Stemme Statistik}
\begin{document}
\makecvheader[R]
\makecvfooter{\today}{\lettertitle{Jesper Petersen - Blå Bog}}{}
\makelettertitle
\begin{cvletter}
\lettersection{Baggrund}
Jesper Petersen, født 25. august 1981 i Haderslev, søn af pensioneret folkeskolelærer Ejvind Petersen og pensioneret skolepsykolog Elna Petersen. Gift med pædagog og cand. pæd.pæd. Pernille Gøttske-Christoffersen. Far til Leo Gøttske Petersen (2010) og Johan Gøttske Petersen (2016).

\lettersection{Uddannelse}
\begin{itemize}
\item BA i statskundskab, Københavns Universitet, 2007-2007.
\item Student, Haderslev Katedralskole, 2000-2000.
\end{itemize}
\lettersection{Parlamentarisk Karriere}
\subsection*{Ordførerskaber}
\begin{itemize}
\item Politisk ordfører
\end{itemize}
\subsection*{Parlamentariske Tillidsposter}
\begin{itemize}
\item Politisk ordfører fra 2019.
\item Medlem af Socialdemokratiets gruppeledelse fra 2019.
\item Skatteordfører 2015-2019.
\item Finansordfører 2013-2015.
\item Medlem af Socialdemokratiets gruppeledelse 2013-2015.
\item Politisk ordfører for Socialistisk Folkeparti 2011-2013.
\end{itemize}
\subsection*{Folketinget}
\subsubsection*{Medlemsperioder}
\begin{itemize}
\item Folketingsmedlem for Socialdemokratiet i Sydjyllands Storkreds fra 21. marts 2013.
\item Folketingsmedlem for Socialistisk Folkeparti i Sydjyllands Storkreds 13. november 2007 - 20. marts 2013.
\end{itemize}
\subsubsection*{Kandidaturer}
\begin{itemize}
\item Kandidat for Socialdemokratiet i Haderslevkredsen fra 2014.
\item Kandidat for Socialistisk Folkeparti i Sønderborgkredsen 2007-2013.
\item Kandidat for Socialistisk Folkeparti i Åbenråkredsen 2004-2006.
\item Kandidat for Socialistisk Folkeparti i Røddingkredsen 2001-2004.
\end{itemize}
\lettersection{Erhvervserfaring}
\begin{itemize}
\item Undervist på folkeskoler i Vojens Kommune, arbejdet i sekretariatet på Center for Ungdomsforskning (DPU) og været organisationskonsulent. Dertil frivilligt arbejde som idrætsleder,.
\end{itemize}
\end{cvletter}
\end{document}