%!TEX TS-program = xelatex
%!TEX encoding = UTF-8 Unicode
\documentclass[11pt, a4paper]{awesome-cv}
\geometry{left=1.4cm, top=.8cm, right=1.4cm, bottom=1.8cm, footskip=.5cm}
\fontdir[fonts/]
\colorlet{awesome}{S-colour}
\setbool{acvSectionColorHighlight}{true}
\renewcommand{\acvHeaderSocialSep}{\quad\textbar\quad}
\recipient{}{}
\name{Rasmus}{Stoklund}
\mobile{+45 3337 4039}
\email{rasmus.stoklund@ft.dk}
\position{Medlem af Folketinget{\enskip\cdotp\enskip}Socialdemokratiet}
\address{}
\photo[circle,noedge,left]{"./Rasmus_Stoklund_profile.jpg"}
\letterdate{\today}
\lettertitle{Rasmus Stoklund - Blå Bog}
\letteropening{}
\letterclosing{}
\letterenclosure[Attached]{Stemme Statistik}
\begin{document}
\makecvheader[R]
\makecvfooter{\today}{\lettertitle{Rasmus Stoklund - Blå Bog}}{}
\makelettertitle
\begin{cvletter}
\lettersection{Baggrund}
Rasmus Stoklund Holm-Nielsen, født 17. marts 1984, søn af pedel Søren Holm-Nielsen og frisør Marianne Ravnsgaard Stoklund. Gift med Anna Andreasen. Parret har tre børn.

\lettersection{Uddannelse}
\begin{itemize}
\item Cand.scient.pol., Københavns Universitet, 2009-2012.
\item BSc.pol., Københavns Universitet, 2006-2009.
\item Student, Fredericia Amtsgymnasium, 2000-2003.
\end{itemize}
\lettersection{Parlamentarisk Karriere}
\subsection*{Ordførerskaber}
\begin{itemize}
\item Integrationsordfører
\item Udlændingeordfører
\end{itemize}
\subsection*{Parlamentariske Tillidsposter}
\begin{itemize}
\item Udlændinge- og Integrationsordfører fra 2019.
\end{itemize}
\subsection*{Folketinget}
\subsubsection*{Medlemsperioder}
\begin{itemize}
\item Folketingsmedlem for Socialdemokratiet i Nordsjællands Storkreds fra 5. juni 2019.
\end{itemize}
\subsubsection*{Kandidaturer}
\begin{itemize}
\item Kandidat for Socialdemokratiet i Frederikssundkredsen fra 2016.
\end{itemize}
\lettersection{Erhvervserfaring}
\begin{itemize}
\item Erhvervspolitisk chef, Dansk Metal, 2016-2019.
\item Erhvervspolitisk konsulent, Dansk Metal, 2014-2016.
\item Fuldmægtig, Erhvervsstyrelsen, 2013-2014.
\item Konsulent, Dansk Metal, 2011-2013.
\end{itemize}
\lettersection{Publikationer}
Forfatter til »Til blå Bjarne - en debatbog om Socialdemokratiet, globaliseringen og fremtiden«, 2016. Bidragsyder til »Farvel til nullerne - Frit Forum København«, 2011.

\end{cvletter}
\end{document}