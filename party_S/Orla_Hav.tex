%!TEX TS-program = xelatex
%!TEX encoding = UTF-8 Unicode
\documentclass[11pt, a4paper]{awesome-cv}
\geometry{left=1.4cm, top=.8cm, right=1.4cm, bottom=1.8cm, footskip=.5cm}
\fontdir[fonts/]
\colorlet{awesome}{S-colour}
\setbool{acvSectionColorHighlight}{true}
\renewcommand{\acvHeaderSocialSep}{\quad\textbar\quad}
\recipient{}{}
\name{Orla}{Hav}
\mobile{+45 3337 4024}
\email{orla.hav@ft.dk}
\position{Medlem af Folketinget{\enskip\cdotp\enskip}Socialdemokratiet}
\address{}
\photo[circle,noedge,left]{"./party_Socialdemokratiet/Orla_Hav_profile.jpg"}
\letterdate{\today}
\lettertitle{Orla Hav - Blå Bog}
\letteropening{}
\letterclosing{}
\letterenclosure[Attached]{Stemme Statistik}
\begin{document}
\makecvheader[R]
\makecvfooter{\today}{\lettertitle{Orla Hav - Blå Bog}}{}
\makelettertitle
\begin{cvletter}
\lettersection{Baggrund}
Orla Hav, født 3. april 1952 i Nørresundby, søn af arbejdsmand Kristian Josefsen Hav og dameskrædder og hjemmegående Lilly Hav. Gift med Benthe Hav.

\lettersection{Uddannelse}
\begin{itemize}
\item Uddannet lærer, Aalborg Seminarium, 1972-1976.
\end{itemize}
\lettersection{Parlamentarisk Karriere}
\subsection*{Ordførerskaber}
\begin{itemize}
\item Erhvervsordfører
\end{itemize}
\subsection*{Parlamentariske Tillidsposter}
\begin{itemize}
\item Næstformand for den danske delegation til Nordisk Råd fra 2019.
\item Næstformand for den danske delegation til OSCEs Parlamentariske Forsamling 2015-2018.
\item Næstformand for Dansk Interparlamentarisk Gruppes bestyrelse 2012-2015.
\item Formand for Børne- og Undervisningsudvalget 2011-2014.
\end{itemize}
\subsection*{Folketinget}
\subsubsection*{Medlemsperioder}
\begin{itemize}
\item Folketingsmedlem for Socialdemokratiet i Nordjyllands Storkreds fra 13. november 2007.
\end{itemize}
\subsubsection*{Kandidaturer}
\begin{itemize}
\item Kandidat for Socialdemokratiet i Aalborg Nordkredsen fra 2007.
\end{itemize}
\lettersection{Erhvervserfaring}
\begin{itemize}
\item Lærer, Aalborg Tekniske Skole, 1980-1998.
\item Lærer i folkeskolen, Dronninglund Kommune og Aalborg Kommune, 1976-1980.
\end{itemize}
\end{cvletter}
\end{document}