%!TEX TS-program = xelatex
%!TEX encoding = UTF-8 Unicode
\documentclass[11pt, a4paper]{awesome-cv}
\geometry{left=1.4cm, top=.8cm, right=1.4cm, bottom=1.8cm, footskip=.5cm}
\fontdir[fonts/]
\colorlet{awesome}{S-colour}
\setbool{acvSectionColorHighlight}{true}
\renewcommand{\acvHeaderSocialSep}{\quad\textbar\quad}
\recipient{}{}
\name{Rasmus}{Prehn}
\mobile{+45 3814 2142}
\email{fvm@fvm.dk}
\position{Minister for fødevarer, landbrug og fiskeri{\enskip\cdotp\enskip}Socialdemokratiet}
\address{}
\photo[circle,noedge,left]{"./Rasmus_Prehn_profile.jpg"}
\letterdate{\today}
\lettertitle{Rasmus Prehn - Blå Bog}
\letteropening{}
\letterclosing{}
\letterenclosure[Attached]{Stemme Statistik}
\begin{document}
\makecvheader[R]
\makecvfooter{\today}{\lettertitle{Rasmus Prehn - Blå Bog}}{}
\makelettertitle
\begin{cvletter}
\lettersection{Baggrund}
Rasmus Prehn, født 18. juni 1973 i Høje Taastrup, søn af elektriker Flemming Prehn og pædagogmedhjælper Birte Hanne Prehn. Gift med cand.mag. Heidi Linnemann Prehn, født Linnemann Nielsen. Har børnene Nikoline, født i 2002, Liva, født i 2004, og Oskar, født i 2009.

\lettersection{Uddannelse}
\begin{itemize}
\item Kandidat i samfundsfag og sociologi, Aalborg Universitet, 1999-2002.
\item Sociologi på masterniveau, University of Leeds, 1998-1999.
\item Bachelor i offentlig administration, Aalborg Universitet, 1995-1998.
\item Matematisk student, Rødovre Gymnasium, 1990-1993.
\end{itemize}
\lettersection{Parlamentarisk Karriere}
\subsection*{Parlamentariske Tillidsposter}
\begin{itemize}
\item Formand for Uddannelses- og Forskningsudvalget 2013-2015.
\item Formand for Udvalget for Forskning, Innovation og Videregående Uddannelser 2011-2013.
\end{itemize}
\subsection*{Folketinget}
\subsubsection*{Medlemsperioder}
\begin{itemize}
\item Folketingsmedlem for Socialdemokratiet i Nordjyllands Storkreds fra 13. november 2007.
\item Folketingsmedlem for Socialdemokratiet i Nordjyllands Amtskreds 8. februar 2005 - 13. november 2007.
\end{itemize}
\subsubsection*{Kandidaturer}
\begin{itemize}
\item Kandidat for Socialdemokratiet i Aalborg Vestkredsen fra 2003.
\end{itemize}
\lettersection{Erhvervserfaring}
\begin{itemize}
\item Gymnasielærer (deltid), Aalborghus Gymnasium, 2008-2015.
\item Højskolelærer, Højskolen for Politik, 2004-2005.
\item Sekretariatsleder, Mindscope, Esbjerg Højskole, 2002-2004.
\item Projektleder, LO's projektmedlemshvervning, 2001-2002.
\item Studentermedhjælper, Center for Arbejdsmarkedsforskning, 1997-2000.
\item Pædagogmedhjælper, Edderkoppen, Fløng, 1993-1995.
\end{itemize}
\lettersection{Publikationer}
Medforfatter til »Det brune Aalborg &ndash;  bogen om Aalborgs brune værtshuse«, 2017 og »Frihed til fællesskab &ndash; vejen til mere demokrati«, 2013.

\end{cvletter}
\end{document}