%!TEX TS-program = xelatex
%!TEX encoding = UTF-8 Unicode
\documentclass[11pt, a4paper]{awesome-cv}
\geometry{left=1.4cm, top=.8cm, right=1.4cm, bottom=1.8cm, footskip=.5cm}
\fontdir[fonts/]
\colorlet{awesome}{S-colour}
\setbool{acvSectionColorHighlight}{true}
\renewcommand{\acvHeaderSocialSep}{\quad\textbar\quad}
\recipient{}{}
\name{Pernille}{Rosenkrantz-Theil}
\mobile{+45 3392 5000}
\email{minister@uvm.dk}
\position{Børne- og undervisningsminister{\enskip\cdotp\enskip}Socialdemokratiet}
\address{}
\photo[circle,noedge,left]{"./party_S/Pernille_Rosenkrantz-Theil_profile.jpg"}
\letterdate{\today}
\lettertitle{Pernille Rosenkrantz-Theil - Blå Bog}
\letteropening{}
\letterclosing{}
\letterenclosure[Attached]{Stemme Statistik}
\begin{document}
\makecvheader[R]
\makecvfooter{\today}{\lettertitle{Pernille Rosenkrantz-Theil - Blå Bog}}{}
\makelettertitle
\begin{cvletter}
\lettersection{Baggrund}
Pernille Rosenkrantz-Theil, født 17. januar 1977 i Skælskør, datter af skoleleder Jørgen Rosenkrantz-Theil og psykoterapeut Linda Rosenkrantz-Theil.

\lettersection{Uddannelse}
\begin{itemize}
\item Bachelor i statskundskab, Københavns Universitet, 1998-2003.
\end{itemize}
\lettersection{Parlamentarisk Karriere}
\subsection*{Ministerposter}
\begin{itemize}
\item Børne- og undervisningsminister fra 27. juni 2019.
\end{itemize}
\subsection*{Parlamentariske Tillidsposter}
\begin{itemize}
\item Børneordfører 2017-2019.
\item Formand for Undervisningsudvalget 2015-2019.
\item Handicapordfører 2014-2015.
\item Socialordfører 2014-2019.
\item Formand for Beskæftigelsesudvalget 2013-2015.
\item Klima- og energiordfører 2011-2014.
\end{itemize}
\subsection*{Folketinget}
\subsubsection*{Medlemsperioder}
\begin{itemize}
\item Folketingsmedlem for Socialdemokratiet i Københavns Storkreds fra 5. juni 2019.
\item Folketingsmedlem for Socialdemokratiet i Sjællands Storkreds 18. juni 2015 - 5. juni 2019.
\item Folketingsmedlem for Socialdemokratiet i Fyns Storkreds 15. september 2011 - 18. juni 2015.
\item Folketingsmedlem for Enhedslisten i Østre Storkreds 8. februar 2005 - 13. november 2007.
\item Folketingsmedlem for Enhedslisten i Fyns Amtskreds 20. november 2001 - 8. februar 2005.
\end{itemize}
\subsubsection*{Kandidaturer}
\begin{itemize}
\item Kandidat for Socialdemokratiet i Vordingborgkredsen fra 2013.
\item Kandidat for Socialdemokratiet i Odense Østkredsen 2009-2013.
\item Kandidat for Enhedslisten i Brønshøjkredsen 2003-2006.
\item Kandidat for Enhedslisten i Odense Østkredsen 2001-2003.
\item Kandidat for Enhedslisten i Otterupkredsen 2001-2003.
\item Kandidat for Enhedslisten i Nørrebrokredsen 1996-2001.
\end{itemize}
\lettersection{Erhvervserfaring}
\begin{itemize}
\item Ledelseskonsulent, 3F's formandssekretariat, 2009-2011.
\item Konsulent og rådgiver, FOA, 2008-2008.
\item Underviser, Krogerup Højskole, 2006-2006.
\item Kampagneleder for Operation Dagsværk, København, 1998-1998.
\item Koordinator for den fri ungdomsuddannelse, Det frie Gymnasium, 1996-1997.
\item Hjemmehjælper, Søllerød Kommune, 1995-1995.
\item Folkekirkens Nødhjælp, 1999 og 2001,.
\end{itemize}
\lettersection{Publikationer}
Har skrevet »Hvilket velfærdssamfund?«, 2019. Medforfatter til »Det betaler sig at investere i mennesker - en bog om sociale investeringer, tidlig indsats, finansministeriets regnemodeller \& SØM«, 2018 og »Ned og op med stress«, 2010. Har bidraget til »Fra kamp til kultur - 20 smagsdommere skyder med skarpt«, 2004 og »En dollar om dagen«, 2001.

\end{cvletter}
\end{document}