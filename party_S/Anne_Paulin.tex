%!TEX TS-program = xelatex
%!TEX encoding = UTF-8 Unicode
\documentclass[11pt, a4paper]{awesome-cv}
\geometry{left=1.4cm, top=.8cm, right=1.4cm, bottom=1.8cm, footskip=.5cm}
\fontdir[fonts/]
\colorlet{awesome}{S-colour}
\setbool{acvSectionColorHighlight}{true}
\renewcommand{\acvHeaderSocialSep}{\quad\textbar\quad}
\recipient{}{}
\name{Anne}{Paulin}
\mobile{+45 6162 5140}
\email{anne.paulin@ft.dk}
\position{Medlem af Folketinget{\enskip\cdotp\enskip}Socialdemokratiet}
\address{}
\photo[circle,noedge,left]{"./party_Socialdemokratiet/Anne_Paulin_profile.jpg"}
\letterdate{\today}
\lettertitle{Anne Paulin - Blå Bog}
\letteropening{}
\letterclosing{}
\letterenclosure[Attached]{Stemme Statistik}
\begin{document}
\makecvheader[R]
\makecvfooter{\today}{\lettertitle{Anne Paulin - Blå Bog}}{}
\makelettertitle
\begin{cvletter}
\lettersection{Baggrund}
Anne Borch Paulin, født 5. januar 1988 i Skive. Gift.

\lettersection{Uddannelse}
\begin{itemize}
\item MSc. International Business and Politics, Copenhagen Business School og Columbia University, 2015-2015.
\item BSc. International Business and Politics, Copenhagen Business School, 2011-2011.
\item Sproglig student, Skive Gymnasium og HF, 2008-2008.
\end{itemize}
\lettersection{Parlamentarisk Karriere}
\subsection*{Ordførerskaber}
\begin{itemize}
\item Energiordfører
\item Forsyningsordfører
\item Klimaordfører
\end{itemize}
\subsection*{Folketinget}
\subsubsection*{Medlemsperioder}
\begin{itemize}
\item Folketingsmedlem for Socialdemokratiet i Vestjyllands Storkreds fra 5. juni 2019.
\end{itemize}
\subsubsection*{Kandidaturer}
\begin{itemize}
\item Kandidat for Socialdemokratiet i Skivekredsen fra 2013.
\end{itemize}
\lettersection{Erhvervserfaring}
\begin{itemize}
\item Rådgiver, Energinet, 2017-2019.
\item Konsulent, TDC Group, 2015-2016.
\item Politisk medarbejder, Danmarks Socialdemokratiske Ungdom, 2013-2014.
\item Praktikant, Geelmuyden Kiese, 2013-2013.
\item Studentermedarbejder, Socialforvaltningen, Københavns Kommune, 2012-2012.
\item Studentermedarbejder, 3F - Fælles Fagligt Forbund, 2010-2011.
\end{itemize}
\end{cvletter}
\end{document}