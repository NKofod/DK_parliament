%!TEX TS-program = xelatex
%!TEX encoding = UTF-8 Unicode
\documentclass[11pt, a4paper]{awesome-cv}
\geometry{left=1.4cm, top=.8cm, right=1.4cm, bottom=1.8cm, footskip=.5cm}
\fontdir[fonts/]
\colorlet{awesome}{S-colour}
\setbool{acvSectionColorHighlight}{true}
\renewcommand{\acvHeaderSocialSep}{\quad\textbar\quad}
\recipient{}{}
\name{Julie}{Skovsby}
\mobile{+45 3337 4047}
\email{julie.skovsby@ft.dk}
\position{Medlem af Folketinget{\enskip\cdotp\enskip}Socialdemokratiet}
\address{}
\photo[circle,noedge,left]{"./party_Socialdemokratiet/Julie_Skovsby_profile.jpg"}
\letterdate{\today}
\lettertitle{Julie Skovsby - Blå Bog}
\letteropening{}
\letterclosing{}
\letterenclosure[Attached]{Stemme Statistik}
\begin{document}
\makecvheader[R]
\makecvfooter{\today}{\lettertitle{Julie Skovsby - Blå Bog}}{}
\makelettertitle
\begin{cvletter}
\lettersection{Baggrund}
Julie Skovsby, født 15. november 1978 i Odense, datter af tømrer Knud Skovsby og økonoma, MPA, Kirsten Skovsby. Gift med Jacob Skovsby Olesen. Parret har to børn.

\lettersection{Uddannelse}
\begin{itemize}
\item Kandidat i statskundskab, Syddansk Universitet, Odense, 2012-2012.
\item Hhx-student, Hillerød Handelsskole, 1998-1998.
\item Folkeskole, Skanseskolen, Hillerød, 1995-1995.
\end{itemize}
\lettersection{Parlamentarisk Karriere}
\subsection*{Ordførerskaber}
\begin{itemize}
\item Kirkeordfører
\item Psykiatriordfører
\end{itemize}
\subsection*{Parlamentariske Tillidsposter}
\begin{itemize}
\item Formand for Tilsynet i henhold til grundlovens § 71 fra 2019.
\item Formand for Kirkeudvalget fra 2019.
\item Næstformand for Udvalget vedrørende Det Etiske Råd fra 2019.
\item Næstformand for Sundheds- og Ældreudvalget 2016-2019.
\item Formand for Sundheds- og Forebyggelsesudvalget 2011-2012.
\end{itemize}
\subsection*{Folketinget}
\subsubsection*{Medlemsperioder}
\begin{itemize}
\item Folketingsmedlem for Socialdemokratiet i Fyns Storkreds fra 13. november 2007.
\end{itemize}
\subsubsection*{Kandidaturer}
\begin{itemize}
\item Kandidat for Socialdemokratiet i Assenskredsen fra 2007.
\item Kandidat for Socialdemokratiet i Otterupkredsen 2006-2007.
\end{itemize}
\end{cvletter}
\end{document}