%!TEX TS-program = xelatex
%!TEX encoding = UTF-8 Unicode
\documentclass[11pt, a4paper]{awesome-cv}
\geometry{left=1.4cm, top=.8cm, right=1.4cm, bottom=1.8cm, footskip=.5cm}
\fontdir[fonts/]
\colorlet{awesome}{S-colour}
\setbool{acvSectionColorHighlight}{true}
\renewcommand{\acvHeaderSocialSep}{\quad\textbar\quad}
\recipient{}{}
\name{Mattias}{Tesfaye}
\mobile{+45 6198 4000}
\email{uim@uim.dk}
\position{Udlændinge- og integrationsminister{\enskip\cdotp\enskip}Socialdemokratiet}
\address{}
\photo[circle,noedge,left]{"./party_S/Mattias_Tesfaye_profile.jpg"}
\letterdate{\today}
\lettertitle{Mattias Tesfaye - Blå Bog}
\letteropening{}
\letterclosing{}
\letterenclosure[Attached]{Stemme Statistik}
\begin{document}
\makecvheader[R]
\makecvfooter{\today}{\lettertitle{Mattias Tesfaye - Blå Bog}}{}
\makelettertitle
\begin{cvletter}
\lettersection{Baggrund}
Mattias Tesfaye, født 31. marts 1981 i Århus, søn af Tesfaye Mamo og social- og sundhedsassistent Jytte Svensson. Gift og har to børn.

\lettersection{Uddannelse}
\begin{itemize}
\item Uddannet murersvend, Skanska og Århus Tekniske Skole, 1998-2001.
\end{itemize}
\lettersection{Parlamentarisk Karriere}
\subsection*{Ministerposter}
\begin{itemize}
\item Udlændinge- og integrationsminister fra 27. juni 2019.
\end{itemize}
\subsection*{Folketinget}
\subsubsection*{Medlemsperioder}
\begin{itemize}
\item Folketingsmedlem for Socialdemokratiet i Københavns Omegns Storkreds fra 18. juni 2015.
\end{itemize}
\subsubsection*{Kandidaturer}
\begin{itemize}
\item Kandidat for Socialdemokratiet i Brøndbykredsen fra 2014.
\end{itemize}
\lettersection{Publikationer}
Har skrevet »Velkommen Mustafa – 50 års socialdemokratisk udlændingepolitik«, 2017, »Kloge hænder ‒ et forsvar for håndværk og faglighed«, 2013, »Vi er ikke dyr, men vi er tyskere – Working poor på Danmarks dørtrin«, 2010, og »Livremmen«, 2004.

\end{cvletter}
\end{document}