%!TEX TS-program = xelatex
%!TEX encoding = UTF-8 Unicode
\documentclass[11pt, a4paper]{awesome-cv}
\geometry{left=1.4cm, top=.8cm, right=1.4cm, bottom=1.8cm, footskip=.5cm}
\fontdir[fonts/]
\colorlet{awesome}{S-colour}
\setbool{acvSectionColorHighlight}{true}
\renewcommand{\acvHeaderSocialSep}{\quad\textbar\quad}
\recipient{}{}
\name{Kasper Sand}{Kjær}
\mobile{+45 3337 4049}
\email{kasper.kjaer@ft.dk}
\position{Medlem af Folketinget{\enskip\cdotp\enskip}Socialdemokratiet}
\address{}
\photo[circle,noedge,left]{"./party_S/Kasper Sand_Kjær_profile.jpg"}
\letterdate{\today}
\lettertitle{Kasper Sand Kjær - Blå Bog}
\letteropening{}
\letterclosing{}
\letterenclosure[Attached]{Stemme Statistik}
\begin{document}
\makecvheader[R]
\makecvfooter{\today}{\lettertitle{Kasper Sand Kjær - Blå Bog}}{}
\makelettertitle
\begin{cvletter}
\lettersection{Baggrund}
Kasper Sand Kjær, født 4. maj 1989 i Randers, søn af pensioneret taxachauffør Mogens Kjær og plejer Hanne Sand.

\lettersection{Uddannelse}
\begin{itemize}
\item Ba i uddannelsesvidenskab, Danmarks Pædagogiske Universitet, 2015-2017.
\item Uddannet underviser  i Arbejderbevægelsen,  Voksen Pædagogisk Grunduddannelse (VPG), 2010-2010.
\item Samfundsfaglig-matematisk student, Randers Statsskole, 2005-2008.
\item Folkeskolens afgangseksamen, Nyvangsskolen, Randers, 1995-2005.
\end{itemize}
\lettersection{Parlamentarisk Karriere}
\subsection*{Ordførerskaber}
\begin{itemize}
\item Kulturordfører
\item Medieordfører
\end{itemize}
\subsection*{Parlamentariske Tillidsposter}
\begin{itemize}
\item Formand for Uddannelses- og Forskningsudvalget fra 2020.
\item Kulturordfører og medieordfører fra 2020.
\item Formand for Beskæftigelsesudvalget 2019-2020.
\item Uddannelses- og forskningsordfører 2019-2020.
\end{itemize}
\subsection*{Folketinget}
\subsubsection*{Medlemsperioder}
\begin{itemize}
\item Folketingsmedlem for Socialdemokratiet i Københavns Omegns Storkreds fra 5. juni 2019.
\end{itemize}
\subsubsection*{Kandidaturer}
\begin{itemize}
\item Kandidat for Socialdemokratiet i Ballerupkredsen fra 2018.
\end{itemize}
\lettersection{Erhvervserfaring}
\begin{itemize}
\item Uddannelseskonsulent, DeltagerDanmark, 2016-2016.
\item Formand, Dansk Ungdoms Fællesråd (DUF), 2015-2019.
\item Medlemskonsulent, 3F, 2012-2012.
\item Ungdomskonsulent, LO, 2012-2015.
\item Kampagneleder ved folketingsvalget, Socialdemokratiet, 2011-2011.
\item Politisk medarbejder, Danmarks Socialdemokratiske Ungdom, 2008-2010.
\item Kasseassistent, Super Best, 2005-2008.
\item Avisbud, 2002-2005.
\end{itemize}
\end{cvletter}
\end{document}