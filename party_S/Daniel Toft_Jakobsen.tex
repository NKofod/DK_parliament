%!TEX TS-program = xelatex
%!TEX encoding = UTF-8 Unicode
\documentclass[11pt, a4paper]{awesome-cv}
\geometry{left=1.4cm, top=.8cm, right=1.4cm, bottom=1.8cm, footskip=.5cm}
\fontdir[fonts/]
\colorlet{awesome}{S-colour}
\setbool{acvSectionColorHighlight}{true}
\renewcommand{\acvHeaderSocialSep}{\quad\textbar\quad}
\recipient{}{}
\name{Daniel Toft}{Jakobsen}
\mobile{+45 3337 4001}
\email{daniel.toft.jakobsen@ft.dk}
\position{Medlem af Folketinget{\enskip\cdotp\enskip}Socialdemokratiet}
\address{}
\photo[circle,noedge,left]{"./Daniel Toft_Jakobsen_profile.jpg"}
\letterdate{\today}
\lettertitle{Daniel Toft Jakobsen - Blå Bog}
\letteropening{}
\letterclosing{}
\letterenclosure[Attached]{Stemme Statistik}
\begin{document}
\makecvheader[R]
\makecvfooter{\today}{\lettertitle{Daniel Toft Jakobsen - Blå Bog}}{}
\makelettertitle
\begin{cvletter}
\lettersection{Baggrund}
Daniel Toft Jakobsen, født 12. maj 1978 i Kongens Lyngby, søn af radiomekaniker Henning Jakobsen og grafisk designer Lene Toft Jakobsen. Gift med Mia Christina Broe Jakobsen. Parret har tre børn.

\lettersection{Uddannelse}
\begin{itemize}
\item Cand.scient.pol., Aarhus Universitet, 1999-2005.
\end{itemize}
\lettersection{Parlamentarisk Karriere}
\subsection*{Ordførerskaber}
\begin{itemize}
\item Handicapordfører
\item Udviklingsordfører
\end{itemize}
\subsection*{Parlamentariske Tillidsposter}
\begin{itemize}
\item Handicapordfører og udviklingsordfører fra 2019.
\item Næstformand for Udlændinge-, Integrations- og Boligudvalget 2015-2016.
\end{itemize}
\subsection*{Folketinget}
\subsubsection*{Medlemsperioder}
\begin{itemize}
\item Folketingsmedlem for Socialdemokratiet i Østjyllands Storkreds fra 18. juni 2015.
\end{itemize}
\subsubsection*{Kandidaturer}
\begin{itemize}
\item Kandidat for Socialdemokratiet i Favrskovkredsen fra 2014.
\item Kandidat for Socialdemokratiet i Hedenstedkredsen 2010-2014.
\end{itemize}
\lettersection{Erhvervserfaring}
\begin{itemize}
\item Specialkonsulent, Health Aarhus Universitet, 2011-2015.
\item Sekretariatsleder, Kristeligt Forbund for Studerende, 2005-2011.
\end{itemize}
\end{cvletter}
\end{document}