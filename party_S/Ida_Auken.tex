%!TEX TS-program = xelatex
%!TEX encoding = UTF-8 Unicode
\documentclass[11pt, a4paper]{awesome-cv}
\geometry{left=1.4cm, top=.8cm, right=1.4cm, bottom=1.8cm, footskip=.5cm}
\fontdir[fonts/]
\colorlet{awesome}{S-colour}
\setbool{acvSectionColorHighlight}{true}
\renewcommand{\acvHeaderSocialSep}{\quad\textbar\quad}
\recipient{}{}
\name{Ida}{Auken}
\mobile{+45 3337 4040}
\email{ida.auken@ft.dk}
\position{Fhv. minister{\enskip\cdotp\enskip}Socialdemokratiet}
\address{}
\photo[circle,noedge,left]{"./party_Socialdemokratiet/Ida_Auken_profile.jpg"}
\letterdate{\today}
\lettertitle{Ida Auken - Blå Bog}
\letteropening{}
\letterclosing{}
\letterenclosure[Attached]{Stemme Statistik}
\begin{document}
\makecvheader[R]
\makecvfooter{\today}{\lettertitle{Ida Auken - Blå Bog}}{}
\makelettertitle
\begin{cvletter}
\lettersection{Baggrund}
Ida Margrete Meier Auken, født 22. april 1978 på Frederiksberg, datter af professor Erik A. Nielsen og medlem af Europa-Parlamentet Margrete Auken. Gift med professor Bent Meier Sørensen. Mor til Niels og Simon Meier Auken.

\lettersection{Uddannelse}
\begin{itemize}
\item Cand.theol., Københavns Universitet, 1998-2006.
\end{itemize}
\lettersection{Parlamentarisk Karriere}
\subsection*{Ordførerskaber}
\begin{itemize}
\item Forskningsordfører
\item Hovedstadsordfører
\item Uddannelsesordfører
\end{itemize}
\subsection*{Parlamentariske Tillidsposter}
\begin{itemize}
\item Formand for Klima-, Energi- og Forsyningsudvalget 2019-2020.
\item Formand for Miljø- og Planlægningsudvalget 2009-2011.
\item Tidligere klima-, energi-, og forsyningsordfører, miljøordfører, erhvervsordfører, iværksætteriordfører, forbrugerordfører og landbrugsordfører.
\end{itemize}
\subsection*{Folketinget}
\subsubsection*{Medlemsperioder}
\begin{itemize}
\item Folketingsmedlem for Socialdemokratiet i Københavns Storkreds fra 3. februar 2021.
\item Folketingsmedlem for Radikale Venstre i Københavns Storkreds 4. februar 2014 - 2. februar 2021.
\item Folketingsmedlem for Socialistisk Folkeparti i Københavns Storkreds 13. november 2007 - 3. februar 2014.
\end{itemize}
\subsubsection*{Kandidaturer}
\begin{itemize}
\item Kandidat for Socialdemokratiet i Slotskredsen fra 2021.
\item Kandidat for Radikale Venstre i Vesterbrokredsen 2014-2021.
\item Kandidat for Socialistisk Folkeparti i Valbykredsen 2007-2014.
\item Kandidat for Socialistisk Folkeparti i Vesterbrokredsen 2007-2014.
\item Kandidat for Socialistisk Folkeparti i Tårnbykredsen fra 2007.
\item Kandidat for Socialistisk Folkeparti i Esbjergkredsen 2004-2006.
\end{itemize}
\lettersection{Erhvervserfaring}
\begin{itemize}
\item Ekstern lektor, Det Teologiske Fakultet, Københavns Universitet, 2006-2007.
\item Forlagsredaktør, Forlaget Alfa, 2004-2007.
\end{itemize}
\lettersection{Publikationer}
Har skrevet bogen »Dansk«, 2018. Redaktør af »Jesus går til filmen ‒ Jesusfiguren i moderne film«, 2007. Medredaktør af »Konstellationer ‒ kirkerne og det europæiske projekt«, 2007, og »Livet efter døden ‒ i de store verdensreligioner«, 2006. Har skrevet adskillige teologiske artikler om bl.a. forholdet mellem stat og kirke, religion, politik og ret samt Giorgio Agambens politiske teologi.

\end{cvletter}
\end{document}