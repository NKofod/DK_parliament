%!TEX TS-program = xelatex
%!TEX encoding = UTF-8 Unicode
\documentclass[11pt, a4paper]{awesome-cv}
\geometry{left=1.4cm, top=.8cm, right=1.4cm, bottom=1.8cm, footskip=.5cm}
\fontdir[fonts/]
\colorlet{awesome}{S-colour}
\setbool{acvSectionColorHighlight}{true}
\renewcommand{\acvHeaderSocialSep}{\quad\textbar\quad}
\recipient{}{}
\name{Dan}{Jørgensen}
\mobile{+45 3392 2800}
\email{kefmkefm.dk}
\position{Klima-, energi- og forsyningsminister{\enskip\cdotp\enskip}Socialdemokratiet}
\address{}
\photo[circle,noedge,left]{"./party_S/Dan_Jørgensen_profile.jpg"}
\letterdate{\today}
\lettertitle{Dan Jørgensen - Blå Bog}
\letteropening{}
\letterclosing{}
\letterenclosure[Attached]{Stemme Statistik}
\begin{document}
\makecvheader[R]
\makecvfooter{\today}{\lettertitle{Dan Jørgensen - Blå Bog}}{}
\makelettertitle
\begin{cvletter}
\lettersection{Baggrund}
Dan Jannik Joslashrgensen, foslashdt 12. juni 1975 i Odense.

\lettersection{Uddannelse}
\begin{itemize}
\item Cand. scient. pol., Aarhus Universitet,20042004.
\end{itemize}
\lettersection{Parlamentarisk Karriere}
\subsection*{Ministerposter}
\begin{itemize}
\item Klima, energi og forsyningsministerfra 27. juni 2019.
\item Minister for fødevarer, landbrug og fiskeri12. december 2013  28. juni 2015.
\end{itemize}
\subsection*{Parlamentariske Tillidsposter}
\begin{itemize}
\item Næstformand for Socialdemokratiets folketingsgruppe20172019.
\item Næstformand for den danske delegation til NATOs Parlamentariske Forsamling20152019.
\item Medlem af EuropaParlamentet formand for Animal Welfare Intergroup, næstformand for EuropaParlamentets udvalg for miljø, folkesundhed og fødevaresikkerhed, formand for de danske socialdemokrater i EuropaParlamentet20042013.
\end{itemize}
\subsection*{Folketinget}
\subsubsection*{Medlemsperioder}
\begin{itemize}
\item Folketingsmedlem for Socialdemokratiet i Fyns Storkreds fra 18. juni 2015.
\item Folketingsmedlem for Socialdemokratiet i Fyns Storkredsfra 18. juni 2015.
\end{itemize}
\subsubsection*{Kandidaturer}
\begin{itemize}
\item Kandidat for Socialdemokratiet i Middelfartkredsenfra 2011.
\end{itemize}
\lettersection{Erhvervserfaring}
\begin{itemize}
\item Adjungeret professor, Aalborg Universitet,20162019.
\item Ekstern lektor, Institut for Statskundskab, Københavns Universitet,20132013.
\item Ekstern lektor, Seattle University,20122013.
\item Ekstern lektor, Sciences Po, Paris,20122013.
\item Ekstern lektor, DIS, Danish Institute for Study Abroad, København,20112013.
\item Ekstern lektor, Institut for Statskundskab, Aarhus Universitet,20102010.
\end{itemize}
\lettersection{Publikationer}
Forfatter tilnbspraquoStaunings arv  vejen til et lykkeligt Danmarklaquo, 2018. raquoGroslashnt haringb  Klimapolitik 2.0laquo, 2010. raquoMellem Mars og Venus  EUs rolle i fremtidens verdensordenlaquo, 2009. raquoPolitikere med begge ben paring jorden haelignger ikke paring traeligernelaquo, 2009 og raquoGroslashn Globalisering  miljoslashpolitik i forandringlaquo, 2007. Medforfatter til raquoBeyond deniers and belivers  towards a map of the politics of climate changelaquo i tidsskriftet Global Environmental Change, 2015. Redaktoslashr afnbspraquoDe rejste sig trodsigt i vrimlenlaquo, 2021 og raquoEurovisioner  Essays om fremtidens Europalaquo, 2006.nbsp

\end{cvletter}
\end{document}