%!TEX TS-program = xelatex
%!TEX encoding = UTF-8 Unicode
\documentclass[11pt, a4paper]{awesome-cv}
\geometry{left=1.4cm, top=.8cm, right=1.4cm, bottom=1.8cm, footskip=.5cm}
\fontdir[fonts/]
\colorlet{awesome}{S-colour}
\setbool{acvSectionColorHighlight}{true}
\renewcommand{\acvHeaderSocialSep}{\quad\textbar\quad}
\recipient{}{}
\name{Nicolai}{Wammen}
\mobile{+45 3392 3333}
\email{minister@fm.dk}
\position{Finansminister{\enskip\cdotp\enskip}Socialdemokratiet}
\address{}
\photo[circle,noedge,left]{"./party_Socialdemokratiet/Nicolai_Wammen_profile.jpg"}
\letterdate{\today}
\lettertitle{Nicolai Wammen - Blå Bog}
\letteropening{}
\letterclosing{}
\letterenclosure[Attached]{Stemme Statistik}
\begin{document}
\makecvheader[R]
\makecvfooter{\today}{\lettertitle{Nicolai Wammen - Blå Bog}}{}
\makelettertitle
\begin{cvletter}
\lettersection{Baggrund}
Nicolai Halby Wammen, født 7. februar 1971, søn af journalist Christian Frederik Wammen og talepædagog Lotte Halby Wammen. Gift med Karen Lund Wammen. Far til Carl og Christian.

\lettersection{Uddannelse}
\begin{itemize}
\item Cand.scient.pol., Aarhus Universitet, 2001-2001.
\item Student, Marselisborg Gymnasium, 1991-1991.
\item High schoolophold, i Massachusetts, USA, 1987-1988.
\item 10. klasse, Brobjergskolen i Aarhus, 1987-1987.
\end{itemize}
\lettersection{Parlamentarisk Karriere}
\subsection*{Parlamentariske Tillidsposter}
\begin{itemize}
\item Politisk ordfører 2015-2019.
\item Næstformand for Socialdemokratiet 2005-2011.
\end{itemize}
\subsection*{Folketinget}
\subsubsection*{Medlemsperioder}
\begin{itemize}
\item Folketingsmedlem for Socialdemokratiet i Østjyllands Storkreds fra 15. september 2011.
\item Folketingsmedlem for Socialdemokratiet i Århus Amtskreds 20. november 2001 - 8. februar 2005.
\end{itemize}
\subsubsection*{Kandidaturer}
\begin{itemize}
\item Kandidat for Socialdemokratiet i Aarhus Sydkredsen fra 2010.
\item Kandidat for Socialdemokratiet i Skanderborgkredsen 1997-2004.
\end{itemize}
\end{cvletter}
\end{document}