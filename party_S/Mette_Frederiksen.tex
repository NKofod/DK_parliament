%!TEX TS-program = xelatex
%!TEX encoding = UTF-8 Unicode
\documentclass[11pt, a4paper]{awesome-cv}
\geometry{left=1.4cm, top=.8cm, right=1.4cm, bottom=1.8cm, footskip=.5cm}
\fontdir[fonts/]
\colorlet{awesome}{S-colour}
\setbool{acvSectionColorHighlight}{true}
\renewcommand{\acvHeaderSocialSep}{\quad\textbar\quad}
\recipient{}{}
\name{Mette}{Frederiksen}
\mobile{+45 3392 3300}
\email{stm@stm.dk}
\position{Statsminister{\enskip\cdotp\enskip}Socialdemokratiet}
\address{}
\photo[circle,noedge,left]{"./party_S/Mette_Frederiksen_profile.jpg"}
\letterdate{\today}
\lettertitle{Mette Frederiksen - Blå Bog}
\letteropening{}
\letterclosing{}
\letterenclosure[Attached]{Stemme Statistik}
\begin{document}
\makecvheader[R]
\makecvfooter{\today}{\lettertitle{Mette Frederiksen - Blå Bog}}{}
\makelettertitle
\begin{cvletter}
\lettersection{Baggrund}
Mette Frederiksen, født 19. november 1977 i Aalborg, datter af typograf Flemming Frederiksen og pædagog Anette Frederiksen. Har datteren Ida Feline og sønnen Magne.

\lettersection{Uddannelse}
\begin{itemize}
\item Masteruddannelse i afrikastudier, Københavns Universitet, 2009-2009.
\item Bachelor i administration og samfundsfag, Aalborg Universitet, 2007-2007.
\item Student, Aalborghus Gymnasium, 1993-1996.
\item Byplanvejen Skole, 1983-1993.
\end{itemize}
\lettersection{Parlamentarisk Karriere}
\subsection*{Ministerposter}
\begin{itemize}
\item Statsminister fra 27. juni 2019.
\item Justitsminister 10. oktober 2014 - 28. juni 2015.
\item Beskæftigelsesminister 3. oktober 2011 - 10. oktober 2014.
\end{itemize}
\subsection*{Parlamentariske Tillidsposter}
\begin{itemize}
\item Formand for Socialdemokratiet fra 2015.
\item Næstformand for Socialdemokratiets folketingsgruppe 2005-2011.
\item Tidligere socialordfører, kulturordfører og ligestillingsordfører.
\end{itemize}
\subsection*{Folketinget}
\subsubsection*{Medlemsperioder}
\begin{itemize}
\item Folketingsmedlem for Socialdemokratiet i Nordjyllands Storkreds fra 5. juni 2019.
\item Folketingsmedlem for Socialdemokratiet i Københavns Omegns Storkreds 13. november 2007 - 5. juni 2019.
\item Folketingsmedlem for Socialdemokratiet i Københavns Amtskreds 20. november 2001 - 13. november 2007.
\end{itemize}
\subsubsection*{Kandidaturer}
\begin{itemize}
\item Kandidat for Socialdemokratiet i Aalborg Østkredsen fra 2018.
\item Kandidat for Socialdemokratiet i Ballerupkredsen 2000-2018.
\end{itemize}
\lettersection{Erhvervserfaring}
\begin{itemize}
\item Ungdomskonsulent i LO, 2000-2001.
\end{itemize}
\lettersection{Publikationer}
Medforfatter til bøgerne »Fra kamp til kultur &ndash; 20 smagsdommere skyder med skarpt«, 2004 og »Epostler«, 2003. 

\end{cvletter}
\end{document}