%!TEX TS-program = xelatex
%!TEX encoding = UTF-8 Unicode
\documentclass[11pt, a4paper]{awesome-cv}
\geometry{left=1.4cm, top=.8cm, right=1.4cm, bottom=1.8cm, footskip=.5cm}
\fontdir[fonts/]
\colorlet{awesome}{S-colour}
\setbool{acvSectionColorHighlight}{true}
\renewcommand{\acvHeaderSocialSep}{\quad\textbar\quad}
\recipient{}{}
\name{Astrid}{Krag}
\mobile{+45 3392 9300}
\email{min@sim.dk}
\position{Social- og ældreminister{\enskip\cdotp\enskip}Socialdemokratiet}
\address{}
\photo[circle,noedge,left]{"./party_Socialdemokratiet/Astrid_Krag_profile.jpg"}
\letterdate{\today}
\lettertitle{Astrid Krag - Blå Bog}
\letteropening{}
\letterclosing{}
\letterenclosure[Attached]{Stemme Statistik}
\begin{document}
\makecvheader[R]
\makecvfooter{\today}{\lettertitle{Astrid Krag - Blå Bog}}{}
\makelettertitle
\begin{cvletter}
\lettersection{Baggrund}
Astrid Krag Kristensen, født 17. november 1982 på Sct. Maria Hospital, Vejle, datter af pensioneret tekstforfatter Ole Jakob Krag Kristensen og pensioneret gymnasielærer Åse Fogh Pedersen. Gift med Andreas Seebach. Parret har tre børn.

\lettersection{Uddannelse}
\begin{itemize}
\item Bachelorstuderende i statskundskab, Københavns Universitet, 2003-2007.
\item Sproglig studentereksamen, Tørring Amtsgymnasium, 1998-2001.
\end{itemize}
\lettersection{Parlamentarisk Karriere}
\subsection*{Parlamentariske Tillidsposter}
\begin{itemize}
\item Næstformand for Udlændinge- og Integrationsudvalget 2016-2019.
\item Tidligere skatteordfører, udlændingeordfører, integrationsordfører, indfødsretsordfører, velfærdsordfører og ældreordfører.
\end{itemize}
\subsection*{Folketinget}
\subsubsection*{Medlemsperioder}
\begin{itemize}
\item Folketingsmedlem for Socialdemokratiet i Sjællands Storkreds fra 4. februar 2014.
\item Folketingsmedlem for Socialistisk Folkeparti i Sjællands Storkreds 13. november 2007 - 3. februar 2014.
\end{itemize}
\subsubsection*{Kandidaturer}
\begin{itemize}
\item Kandidat for Socialdemokratiet i Køgekredsen fra 2019.
\item Kandidat for Socialdemokratiet i Grevekredsen 2014-2019.
\item Kandidat for Socialistisk Folkeparti i Roskildekredsen 2007-2014.
\item Kandidat for Socialistisk Folkeparti i Vejlekredsen 2001-2006.
\end{itemize}
\lettersection{Erhvervserfaring}
\begin{itemize}
\item Ansat i hjemmeplejen, Sundparken, Amager, 2007-2007.
\item Studentermedhjælp, Social- og Sundhedsskolen i København, 2003-2005.
\item Pædagogmedhjælper, Frederiksberg Hospitalsvuggestue, 2001-2002.
\item Service- og kasseassistent, Netto, Tørring, 1999-2001.
\end{itemize}
\end{cvletter}
\end{document}