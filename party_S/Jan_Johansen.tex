%!TEX TS-program = xelatex
%!TEX encoding = UTF-8 Unicode
\documentclass[11pt, a4paper]{awesome-cv}
\geometry{left=1.4cm, top=.8cm, right=1.4cm, bottom=1.8cm, footskip=.5cm}
\fontdir[fonts/]
\colorlet{awesome}{S-colour}
\setbool{acvSectionColorHighlight}{true}
\renewcommand{\acvHeaderSocialSep}{\quad\textbar\quad}
\recipient{}{}
\name{Jan}{Johansen}
\mobile{+45 3337 4004}
\email{jan.johansen@ft.dk}
\position{Medlem af Folketinget{\enskip\cdotp\enskip}Socialdemokratiet}
\address{}
\photo[circle,noedge,left]{"./Jan_Johansen_profile.jpg"}
\letterdate{\today}
\lettertitle{Jan Johansen - Blå Bog}
\letteropening{}
\letterclosing{}
\letterenclosure[Attached]{Stemme Statistik}
\begin{document}
\makecvheader[R]
\makecvfooter{\today}{\lettertitle{Jan Johansen - Blå Bog}}{}
\makelettertitle
\begin{cvletter}
\lettersection{Baggrund}
Jan Johansen, født 12. december 1955 i Svendborg, søn af specialarbejder Preben Johansen og hjemmegående Gerda Johansen. Gift med Jette Hye Johansen.

\lettersection{Parlamentarisk Karriere}
\subsection*{Ordførerskaber}
\begin{itemize}
\item Beredskabsordfører
\item Forsvarsordfører
\end{itemize}
\subsection*{Parlamentariske Tillidsposter}
\begin{itemize}
\item Færdselssikkerhedsordfører 2019-2020.
\item Næstformand for Forsvarsudvalget fra 2015.
\end{itemize}
\subsection*{Folketinget}
\subsubsection*{Medlemsperioder}
\begin{itemize}
\item Folketingsmedlem for Socialdemokratiet i Fyns Storkreds fra 15. september 2011.
\end{itemize}
\subsubsection*{Kandidaturer}
\begin{itemize}
\item Kandidat for Socialdemokratiet i Nyborgkredsen fra 2010.
\end{itemize}
\lettersection{Erhvervserfaring}
\begin{itemize}
\item Næstformand, 3F Østfyn, 2005-2011.
\item Formand, SiD Munkebo, 2001-2005.
\item Specialarbejder, Lindøvæftet, 1974-2001.
\item Specialarbejder, Bullerup Møbelfabrik, 1973-1974.
\item Sømand, 1970-1972.
\end{itemize}
\end{cvletter}
\end{document}