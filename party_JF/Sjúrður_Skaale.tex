%!TEX TS-program = xelatex
%!TEX encoding = UTF-8 Unicode
\documentclass[11pt, a4paper]{awesome-cv}
\geometry{left=1.4cm, top=.8cm, right=1.4cm, bottom=1.8cm, footskip=.5cm}
\fontdir[fonts/]
\colorlet{awesome}{JF-colour}
\setbool{acvSectionColorHighlight}{true}
\renewcommand{\acvHeaderSocialSep}{\quad\textbar\quad}
\recipient{}{}
\name{Sjúrður}{Skaale}
\mobile{+45 3337 4015}
\email{sjurdur.skaale@ft.dk}
\position{Medlem af Folketinget{\enskip\cdotp\enskip}Javnaðarflokkurin}
\address{}
\photo[circle,noedge,left]{"./party_Javnaðarflokkurin/Sjúrður_Skaale_profile.jpg"}
\letterdate{\today}
\lettertitle{Sjúrður Skaale - Blå Bog}
\letteropening{}
\letterclosing{}
\letterenclosure[Attached]{Stemme Statistik}
\begin{document}
\makecvheader[R]
\makecvfooter{\today}{\lettertitle{Sjúrður Skaale - Blå Bog}}{}
\makelettertitle
\begin{cvletter}
\lettersection{Baggrund}
Sj&uacute;r&eth;ur Skaale, født 8. marts 1967 i T&oacute;rshavn, søn af afdøde navigationslærer J&oacute;annes Skaale og pensioneret handicaplærer &Oacute;luva Skaale. Gift med cand.psyk.aut. S&uacute;sanna O. Skaale.

\lettersection{Uddannelse}
\begin{itemize}
\item Cand.scient.pol., Københavns Universitet og Universidad Complutense de Madrid, 1992-1997.
\item Suppleringsuddannelse i spansk, Københavns Universitet, 1990-1991.
\end{itemize}
\lettersection{Parlamentarisk Karriere}
\subsection*{Parlamentariske Tillidsposter}
\begin{itemize}
\item Formand for Folketingets Arktiske Delegation fra 2019.
\item Næstformand for Færøudvalget fra 2019.
\item Stedfortræder i den danske delegation i Nordisk Råd fra 2019.
\item Medlem af Udenrigsudvalget fra 2019.
\item Formand for Folketingets Arktiske Arbejdsgruppe fra 2015.
\item Formand for Færøudvalget 2011-2017.
\item Medlem af den danske delegation i Nordisk Råd 2011-2015.
\item Stedfortræder i Det Udenrigspolitiske Nævn fra 2011.
\item Medlem af Føroya Løgting (Færøernes Lagting) 2008-2011.
\end{itemize}
\subsection*{Folketinget}
\subsubsection*{Medlemsperioder}
\begin{itemize}
\item Folketingsmedlem for Javnaðarflokkurin i Færøerne fra 15. september 2011.
\end{itemize}
\subsubsection*{Kandidaturer}
\begin{itemize}
\item Kandidat for Javnaðarflokkurin i Færøerne fra 2011.
\item Kandidat for Tjóðveldi i Færøerne 2007-2011.
\item Kandidat for Tjóðveldisflokkurin i Færøerne 2005-2007.
\end{itemize}
\lettersection{Erhvervserfaring}
\begin{itemize}
\item Selvstændig, gymnasielærer, reklamemand, Færøerne, 2005-2008.
\item Sekretær for Den Nordatlantiske Gruppe i Folketinget, 2001-2005.
\item Embedsmand, rådgiver, Føroya Landsstýri (Færøernes Landsstyre), 1999-2001.
\item Journalist, Sjónvarp Føroya (færøsk tv), 1997-1998.
\item Journalist, Dimmalætting/Sosialurin (færøske aviser), 1989-1997.
\end{itemize}
\lettersection{Publikationer}
Forfatter til »Da Færøerne ville løsrive sig«, 2020; »Skerpikjøt« (Skærpekød), 2018; »2000 ‒ t&aacute; Føroyar skuldu loysa« (2000 ‒ da Færøerne skulle løsrives), 2016 og »T&aacute; Jesus gjørdi nekaran hv&iacute;tan« (Da Jesus gjorde negeren hvid), artikelsamling, 2010. Medforfatter til »The Emergence of a Democratic Right to Self Determination in Europe«, 2016; »Krossvegir« (Korsveje), artikelsamling om religion og politik, 2007 og »Hall&oacute;«, artikelsamling om moderne kommunikation, 2005. Redaktør af »Kilder til Grønlands og Færøernes historie«, 2004; »The Right To National Self-Determination«, 2004 og »Hv&iacute;tab&oacute;k« (Hvidbogen) om muligheden for oprettelse af en færøsk stat, 1999. Har udgivet »Pipar \& Salt. Alt var betri fyrr«, dvd, 2018; »Pipar \& Salt-live«, dvd, 2011; »E Elski Førjar live«, dvd, 2008; »E Elski Førjar«, tv-serie på dvd og cd, 2008; »Pipar \& Salt Classic«, tv-serie på dvd, 2007; »Pipar \& Salt 2005« dvd, 2005; »Pipar \& Salt« vhs-bånd, 2000 og »Pipar \& Salt«, cd, 1998.

\end{cvletter}
\end{document}