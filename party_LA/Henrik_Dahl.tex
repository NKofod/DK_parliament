%!TEX TS-program = xelatex
%!TEX encoding = UTF-8 Unicode
\documentclass[11pt, a4paper]{awesome-cv}
\geometry{left=1.4cm, top=.8cm, right=1.4cm, bottom=1.8cm, footskip=.5cm}
\fontdir[fonts/]
\colorlet{awesome}{LA-colour}
\setbool{acvSectionColorHighlight}{true}
\renewcommand{\acvHeaderSocialSep}{\quad\textbar\quad}
\recipient{}{}
\name{Henrik}{Dahl}
\mobile{+45 3337 4912}
\email{henrik.dahlft.dk}
\position{Medlem af Folketinget{\enskip\cdotp\enskip}Liberal Alliance}
\address{}
\photo[circle,noedge,left]{"./party_LA/Henrik_Dahl_profile.jpg"}
\letterdate{\today}
\lettertitle{Henrik Dahl - Blå Bog}
\letteropening{}
\letterclosing{}
\letterenclosure[Attached]{Stemme Statistik}
\begin{document}
\makecvheader[R]
\makecvfooter{\today}{\lettertitle{Henrik Dahl - Blå Bog}}{}
\makelettertitle
\begin{cvletter}
\lettersection{Baggrund}
Johan Henrik Dahl, foslashdt 20. februar 1960 i VejlbyRisskov, soslashn af sognepraeligst Paul Dahl og lektor Inger Marie Dahl.nbspGift med Christina Yoon Soslashttrup Dahl.

\lettersection{Uddannelse}
\begin{itemize}
\item Ph.d., Copenhagen Business School,19901993.
\item Cand.scient.soc. sociolog, Københavns Universitet,19801987.
\item Sprogofficer russisk, København,19781980.
\item MA in Communications, University of Pennsylvania, Philadelphia, USA, fra 1985 til 1986 og fra 1987 til 1988,.
\end{itemize}
\lettersection{Parlamentarisk Karriere}
\subsection*{Ordførerskaber}
\begin{itemize}
\item Udenrigsordfører
\end{itemize}
\subsection*{Parlamentariske Tillidsposter}
\begin{itemize}
\item Uddannelse og forskningsordfører, børne og undervisningsordfører,  miljøordfører, udlændinge, integrations og indfødsretsordfører, sundheds og ældreordfører, kirkeordfører, forsvarsordfører, kulturordfører og fødevare og fiskeriordfører20192022.
\item Udenrigsordførerfra 2016.
\item Næstformand for Uddannelses og Forskningsudvalget20152019.
\end{itemize}
\subsection*{Folketinget}
\subsubsection*{Medlemsperioder}
\begin{itemize}
\item Folketingsmedlem for Liberal Alliance i Sydjyllands Storkreds fra 18. juni 2015.
\item Folketingsmedlem for Liberal Alliance i Sydjyllands Storkredsfra 18. juni 2015.
\end{itemize}
\subsubsection*{Kandidaturer}
\begin{itemize}
\item Kandidat for Liberal Alliance i alle opstillingskredse i Sydjyllands Storkredsfra 2014.
\end{itemize}
\lettersection{Erhvervserfaring}
\begin{itemize}
\item Forfatter og foredragsholder, København,20092015.
\item Anmelder, Weekendavisen,20092014.
\item Udviklingschef, Niras AS, Allerød,20082009.
\item Studievært, DR P1, København,20042005.
\item Adjungeret professor, Copenhagen Business School,20032008.
\item Medindehaver, Explora AS, København,19992008.
\item Forskningschef, AC Nielsen AIM, København,19941998.
\item Adjunkt, Roskilde Universitetscenter,19931994.
\item Adjunktvikar, Copenhagen Business School,19901993.
\item Medieforsker, Danmarks Radio,19881990.
\end{itemize}
\lettersection{Publikationer}
Forfatter til raquoDen sociale konstruktion af uvirkelighedenlaquo, 2020, romanen raquoNTlaquo, 2013, raquoSpildte kraeligfterlaquo, 2011, raquoDen usynlige verdenlaquo, 2008, raquoMindernes landlaquo, 2005, raquoDen kronologiske uskyldlaquo, 1998, og raquoHvis din nabo var en billaquo, 1997.nbspMedforfatter til raquoSandheden kort  Christiansborg fra A til Aringlaquo, 2018, raquoKrigeren, borgeren og taberenlaquo, 2006, raquoEpostlerlaquo, 2003, raquoDet ny systemskiftelaquo, 2001, raquoBorgerlige ord efter revolutionenlaquo, 1999, og raquoMarketing og semiotiklaquo, 1993.nbspTillige forfatter til en lang raeligkke bidrag til antologier og dagblade fra 1995.

\end{cvletter}
\end{document}