%!TEX TS-program = xelatex
%!TEX encoding = UTF-8 Unicode
\documentclass[11pt, a4paper]{awesome-cv}
\geometry{left=1.4cm, top=.8cm, right=1.4cm, bottom=1.8cm, footskip=.5cm}
\fontdir[fonts/]
\colorlet{awesome}{SF-colour}
\setbool{acvSectionColorHighlight}{true}
\renewcommand{\acvHeaderSocialSep}{\quad\textbar\quad}
\recipient{}{}
\name{Kirsten Normann}{Andersen}
\mobile{+45 3337 4466}
\email{kirsten.normann.andersen@ft.dk}
\position{Medlem af Folketinget{\enskip\cdotp\enskip}Socialistisk Folkeparti}
\address{}
\photo[circle,noedge,left]{"./party_SF/Kirsten Normann_Andersen_profile.jpg"}
\letterdate{\today}
\lettertitle{Kirsten Normann Andersen - Blå Bog}
\letteropening{}
\letterclosing{}
\letterenclosure[Attached]{Stemme Statistik}
\begin{document}
\makecvheader[R]
\makecvfooter{\today}{\lettertitle{Kirsten Normann Andersen - Blå Bog}}{}
\makelettertitle
\begin{cvletter}
\lettersection{Baggrund}
Kirsten Normann Andersen, født 28. april 1962 i Aarhus.

\lettersection{Uddannelse}
\begin{itemize}
\item Uddannet sygehjælper, Frederiksberg, 1985-1985.
\end{itemize}
\lettersection{Parlamentarisk Karriere}
\subsection*{Ordførerskaber}
\begin{itemize}
\item Ældreordfører
\item Sundhedsordfører
\item Ordfører for afbureaukratisering
\end{itemize}
\subsection*{Parlamentariske Tillidsposter}
\begin{itemize}
\item Boligordfører 2019-2020.
\item Formand for Social- og Indenrigsudvalget fra 2019.
\end{itemize}
\subsection*{Folketinget}
\subsubsection*{Medlemsperioder}
\begin{itemize}
\item Folketingsmedlem for Socialistisk Folkeparti i Østjyllands Storkreds fra 8. august 2016.
\end{itemize}
\subsubsection*{Kandidaturer}
\begin{itemize}
\item Kandidat for Socialistisk Folkeparti i Århus Østkredsen fra 2016.
\item Kandidat for Socialistisk Folkeparti i Skanderborgkredsen 2014-2016.
\end{itemize}
\lettersection{Erhvervserfaring}
\begin{itemize}
\item Afdelingsformand, FOA Århus, fra 2003.
\item Sektornæstformand, FOA Århus, 1996-2003.
\item Sygehjælper, Aarhus, fra 1986.
\end{itemize}
\end{cvletter}
\end{document}