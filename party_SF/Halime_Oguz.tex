%!TEX TS-program = xelatex
%!TEX encoding = UTF-8 Unicode
\documentclass[11pt, a4paper]{awesome-cv}
\geometry{left=1.4cm, top=.8cm, right=1.4cm, bottom=1.8cm, footskip=.5cm}
\fontdir[fonts/]
\colorlet{awesome}{SF-colour}
\setbool{acvSectionColorHighlight}{true}
\renewcommand{\acvHeaderSocialSep}{\quad\textbar\quad}
\recipient{}{}
\name{Halime}{Oguz}
\mobile{+45 3337 4408}
\email{halime.oguz@ft.dk}
\position{Medlem af Folketinget{\enskip\cdotp\enskip}Socialistisk Folkeparti}
\address{}
\photo[circle,noedge,left]{"./party_Socialistisk Folkeparti/Halime_Oguz_profile.jpg"}
\letterdate{\today}
\lettertitle{Halime Oguz - Blå Bog}
\letteropening{}
\letterclosing{}
\letterenclosure[Attached]{Stemme Statistik}
\begin{document}
\makecvheader[R]
\makecvfooter{\today}{\lettertitle{Halime Oguz - Blå Bog}}{}
\makelettertitle
\begin{cvletter}
\lettersection{Baggrund}
Halime Oguz, født 15. juni 1970 i Tyrkiet, datter af tolk Mustafa Oguz og hjemmegående husmor Sultan Oguz. 

\lettersection{Uddannelse}
\begin{itemize}
\item Cand.mag. i mellemøststudier, Syddansk Universitet, 2006-2012.
\item Ba i litteraturvidenskab, Syddansk Universitet, 2000-2006.
\end{itemize}
\lettersection{Parlamentarisk Karriere}
\subsection*{Ordførerskaber}
\begin{itemize}
\item Boligordfører
\item Integrationsordfører
\end{itemize}
\subsection*{Parlamentariske Tillidsposter}
\begin{itemize}
\item Ordfører for international udvikling EU, indfødsret og hjemløse 2019-2020.
\end{itemize}
\subsection*{Folketinget}
\subsubsection*{Medlemsperioder}
\begin{itemize}
\item Folketingsmedlem for Socialistisk Folkeparti i Københavns Storkreds fra 5. juni 2019.
\end{itemize}
\subsubsection*{Kandidaturer}
\begin{itemize}
\item Kandidat for Socialistisk Folkeparti i Nørrebrokredsen fra 2015.
\end{itemize}
\lettersection{Erhvervserfaring}
\begin{itemize}
\item Klummeskribent, Berlingske Tidende, fra 2016.
\item Integrationsvejleder, Københavns Kommune, 2013-2018.
\item Tolk, Odense Kommune, 2010-2012.
\item Tolk, diverse steder på Fyn, 2003-2012.
\end{itemize}
\end{cvletter}
\end{document}