%!TEX TS-program = xelatex
%!TEX encoding = UTF-8 Unicode
\documentclass[11pt, a4paper]{awesome-cv}
\geometry{left=1.4cm, top=.8cm, right=1.4cm, bottom=1.8cm, footskip=.5cm}
\fontdir[fonts/]
\colorlet{awesome}{SF-colour}
\setbool{acvSectionColorHighlight}{true}
\renewcommand{\acvHeaderSocialSep}{\quad\textbar\quad}
\recipient{}{}
\name{Lisbeth}{Bech-Nielsen}
\mobile{+45 3337 4428}
\email{lisbeth.poulsen@ft.dk}
\position{Medlem af Folketinget{\enskip\cdotp\enskip}Socialistisk Folkeparti}
\address{}
\photo[circle,noedge,left]{"./party_SF/Lisbeth_Bech-Nielsen_profile.jpg"}
\letterdate{\today}
\lettertitle{Lisbeth Bech-Nielsen - Blå Bog}
\letteropening{}
\letterclosing{}
\letterenclosure[Attached]{Stemme Statistik}
\begin{document}
\makecvheader[R]
\makecvfooter{\today}{\lettertitle{Lisbeth Bech-Nielsen - Blå Bog}}{}
\makelettertitle
\begin{cvletter}
\lettersection{Baggrund}
Lisbeth Bech-Nielsen, født 1. december 1982 i Sønderborg, datter af driftsmester Per Bech Poulsen og regnskabsassistent Randi Bech Poulsen.

\lettersection{Uddannelse}
\begin{itemize}
\item Kandidat i udvikling og internationale forhold, Aalborg Universitet, 2006-2009.
\item BSc i politik og administration, Aalborg Universitet, 2003-2006.
\item Sproglig studentereksamen, Sønderborg Statsskole, 1998-2001.
\end{itemize}
\lettersection{Parlamentarisk Karriere}
\subsection*{Ordførerskaber}
\begin{itemize}
\item Erhvervsordfører
\item Finansordfører
\item It-ordfører
\end{itemize}
\subsection*{Folketinget}
\subsubsection*{Medlemsperioder}
\begin{itemize}
\item Folketingsmedlem for Socialistisk Folkeparti i Nordjyllands Storkreds fra 15. september 2011.
\end{itemize}
\subsubsection*{Kandidaturer}
\begin{itemize}
\item Kandidat for Socialistisk Folkeparti i Aalborg Østkredsen fra 2008.
\end{itemize}
\lettersection{Erhvervserfaring}
\begin{itemize}
\item Projektkoordinator, SEA, Aalborg Universitet, 2010-2011.
\item Samfundsfagslærer på hf, VUC Nordjylland, 2009-2009.
\end{itemize}
\end{cvletter}
\end{document}