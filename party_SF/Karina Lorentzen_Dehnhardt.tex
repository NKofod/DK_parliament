%!TEX TS-program = xelatex
%!TEX encoding = UTF-8 Unicode
\documentclass[11pt, a4paper]{awesome-cv}
\geometry{left=1.4cm, top=.8cm, right=1.4cm, bottom=1.8cm, footskip=.5cm}
\fontdir[fonts/]
\colorlet{awesome}{SF-colour}
\setbool{acvSectionColorHighlight}{true}
\renewcommand{\acvHeaderSocialSep}{\quad\textbar\quad}
\recipient{}{}
\name{Karina Lorentzen}{Dehnhardt}
\mobile{+45 6162 3045}
\email{karina.dehnhardt@ft.dk}
\position{Medlem af Folketinget{\enskip\cdotp\enskip}Socialistisk Folkeparti}
\address{}
\photo[circle,noedge,left]{"./party_Socialistisk Folkeparti/Karina Lorentzen_Dehnhardt_profile.jpg"}
\letterdate{\today}
\lettertitle{Karina Lorentzen Dehnhardt - Blå Bog}
\letteropening{}
\letterclosing{}
\letterenclosure[Attached]{Stemme Statistik}
\begin{document}
\makecvheader[R]
\makecvfooter{\today}{\lettertitle{Karina Lorentzen Dehnhardt - Blå Bog}}{}
\makelettertitle
\begin{cvletter}
\lettersection{Baggrund}
Karina Lorentzen Dehnhardt, født 26. oktober 1973 i Kolding, datter af pensionist Hans Per Lorentzen og teknisk servicemedarbejder Jette Lorentzen. Gift med Brian Dehnhardt. Parret har datteren Clara.

\lettersection{Uddannelse}
\begin{itemize}
\item Master i professionel kommunikation, Roskilde Universitet, 2016-2018.
\item Læsevejleder, CVU Lillebælt, 2007-2009.
\item Lærer, Haderslev, 1998-2002.
\item Akademiøkonom, Esbjerg, 1995-1997.
\end{itemize}
\lettersection{Parlamentarisk Karriere}
\subsection*{Ordførerskaber}
\begin{itemize}
\item Landdistrikts- og øordfører
\item Retsordfører
\end{itemize}
\subsection*{Parlamentariske Tillidsposter}
\begin{itemize}
\item It-ordfører 2020-2020.
\item Formand for Udvalget for Landdistrikter og Øer fra 2019.
\item Gruppeformand for Socialistisk Folkepartis folketingsgruppe 2014-2014.
\item Formand for Retsudvalget 2012-2015.
\item Næstformand for Retsudvalget 2010-2011.
\end{itemize}
\subsection*{Folketinget}
\subsubsection*{Medlemsperioder}
\begin{itemize}
\item Folketingsmedlem for Socialistisk Folkeparti i Sydjyllands Storkreds fra 5. juni 2019.
\item Folketingsmedlem for Socialistisk Folkeparti i Sydjyllands Storkreds 13. november 2007 - 18. juni 2015.
\end{itemize}
\subsubsection*{Kandidaturer}
\begin{itemize}
\item Kandidat for Socialistisk Folkeparti i Fredericiakredsen fra 2018.
\item Kandidat for Socialistisk Folkeparti i Vejle Nordkredsen fra 2018.
\item Kandidat for Socialistisk Folkeparti i Vejle Sydkredsen fra 2018.
\item Kandidat for Socialistisk Folkeparti i Kolding Nordkredsen fra 2014.
\item Kandidat for Socialistisk Folkeparti i Kolding Sydkredsen fra 2008.
\item Kandidat for Socialistisk Folkeparti i Esbjerg Omegnskredsen 2008-2010.
\item Kandidat for Socialistisk Folkeparti i Vejle Sydkredsen 2008-2010.
\item Kandidat for Socialistisk Folkeparti i Kolding Nordkredsen 2006-2008.
\end{itemize}
\lettersection{Erhvervserfaring}
\begin{itemize}
\item Kommunikationskonsulent, Dansk Fjernvarme, 2016-2019.
\item Presserådgiver, Børne- og Ungeforvaltningen, Odense Kommune, 2015-2016.
\item Lærer, Ålykkeskolen, Kolding, 2003-2007.
\item Lærer, Salbrovadskolen, Assens, 2003-2004.
\item Restaurantmedarbejder, Monarch, Kolding, Harte Nord, 2002-2007.
\end{itemize}
\end{cvletter}
\end{document}