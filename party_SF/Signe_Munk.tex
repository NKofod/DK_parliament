%!TEX TS-program = xelatex
%!TEX encoding = UTF-8 Unicode
\documentclass[11pt, a4paper]{awesome-cv}
\geometry{left=1.4cm, top=.8cm, right=1.4cm, bottom=1.8cm, footskip=.5cm}
\fontdir[fonts/]
\colorlet{awesome}{SF-colour}
\setbool{acvSectionColorHighlight}{true}
\renewcommand{\acvHeaderSocialSep}{\quad\textbar\quad}
\recipient{}{}
\name{Signe}{Munk}
\mobile{+45 6162 4090}
\email{signe.munk@ft.dk}
\position{Medlem af Folketinget{\enskip\cdotp\enskip}Socialistisk Folkeparti}
\address{}
\photo[circle,noedge,left]{"./Signe_Munk_profile.jpg"}
\letterdate{\today}
\lettertitle{Signe Munk - Blå Bog}
\letteropening{}
\letterclosing{}
\letterenclosure[Attached]{Stemme Statistik}
\begin{document}
\makecvheader[R]
\makecvfooter{\today}{\lettertitle{Signe Munk - Blå Bog}}{}
\makelettertitle
\begin{cvletter}
\lettersection{Baggrund}
Signe Munk, født 10. marts 1990 i Odense, datter af overlæge Stig Munk og sygeplejerske Lis Munk.

\lettersection{Uddannelse}
\begin{itemize}
\item Bachelor i Statskundskab, Aarhus Universitet, 2010-2013.
\item Student fra alment gymnasium, Viborg Katedralskole, 2006-2009.
\item Studerer til sygeplejerske (pt. orlov), Københavns Professionshøjskole, fra 2016.
\end{itemize}
\lettersection{Parlamentarisk Karriere}
\subsection*{Ordførerskaber}
\begin{itemize}
\item Energiordfører
\item Klimaordfører
\item Forsyningsordfører
\end{itemize}
\subsection*{Parlamentariske Tillidsposter}
\begin{itemize}
\item Velfærdsuddannelsesordfører 2019-2020.
\end{itemize}
\subsection*{Folketinget}
\subsubsection*{Medlemsperioder}
\begin{itemize}
\item Folketingsmedlem for Socialistisk Folkeparti i Vestjyllands Storkreds fra 5. juni 2019.
\end{itemize}
\subsubsection*{Kandidaturer}
\begin{itemize}
\item Kandidat for Socialistisk Folkeparti i alle opstillingskredse i Vestjyllands Storkreds fra 2018.
\end{itemize}
\lettersection{Erhvervserfaring}
\begin{itemize}
\item Kampagnemedarbejder, Socialistisk Folkeparti, Silkeborg og Viborg, 2013-2014.
\item Ufaglært hjemmehjælper, Viborg Kommune, 2008-2010.
\end{itemize}
\end{cvletter}
\end{document}