%!TEX TS-program = xelatex
%!TEX encoding = UTF-8 Unicode
\documentclass[11pt, a4paper]{awesome-cv}
\geometry{left=1.4cm, top=.8cm, right=1.4cm, bottom=1.8cm, footskip=.5cm}
\fontdir[fonts/]
\colorlet{awesome}{S-colour}
\setbool{acvSectionColorHighlight}{true}
\renewcommand{\acvHeaderSocialSep}{\quad\textbar\quad}
\recipient{}{}
\name{Pernille}{Rosenkrantz-Theil}
\mobile{+45 3392 5000}
\email{ministeruvm.dk}
\position{Børne- og undervisningsminister{\enskip\cdotp\enskip}Socialdemokratiet}
\address{}
\photo[circle,noedge,left]{"./party_S/Pernille_Rosenkrantz-Theil_profile.jpg"}
\letterdate{\today}
\lettertitle{Pernille Rosenkrantz-Theil - Blå Bog}
\letteropening{}
\letterclosing{}
\letterenclosure[Attached]{Stemme Statistik}
\begin{document}
\makecvheader[R]
\makecvfooter{\today}{\lettertitle{Pernille Rosenkrantz-Theil - Blå Bog}}{}
\makelettertitle
\begin{cvletter}
\lettersection{Baggrund}
Pernille RosenkrantzTheil, født 17. januar 1977 i Skælskør, datter af skoleleder Jørgen RosenkrantzTheil og psykoterapeut Linda RosenkrantzTheil.

\lettersection{Uddannelse}
\begin{itemize}
\item Bachelor i statskundskab, Københavns Universitet,19982003.
\end{itemize}
\lettersection{Parlamentarisk Karriere}
\subsection*{Ministerposter}
\begin{itemize}
\item Børne og undervisningsministerfra 27. juni 2019.
\end{itemize}
\subsection*{Parlamentariske Tillidsposter}
\begin{itemize}
\item Børneordfører20172019.
\item Formand for Undervisningsudvalget20152019.
\item Handicapordfører20142015.
\item Socialordfører20142019.
\item Formand for Beskæftigelsesudvalget20132015.
\item Klima og energiordfører20112014.
\end{itemize}
\subsection*{Folketinget}
\subsubsection*{Medlemsperioder}
\begin{itemize}
\item Folketingsmedlem for Socialdemokratiet i Københavns Storkreds fra 5. juni 2019.
\item Folketingsmedlem for Socialdemokratiet i Københavns Storkredsfra 5. juni 2019.
\item Folketingsmedlem for Socialdemokratiet i Sjællands Storkreds18. juni 2015  5. juni 2019.
\item Folketingsmedlem for Socialdemokratiet i Fyns Storkreds15. september 2011  18. juni 2015.
\item Folketingsmedlem for Enhedslisten i Østre Storkreds8. februar 2005  13. november 2007.
\item Folketingsmedlem for Enhedslisten i Fyns Amtskreds20. november 2001  8. februar 2005.
\end{itemize}
\subsubsection*{Kandidaturer}
\begin{itemize}
\item Kandidat for Socialdemokratiet i Østerbrokredsenfra 2016.
\item Kandidat for Socialdemokratiet i Vordingborgkredsen20132016.
\item Kandidat for Socialdemokratiet i Odense Østkredsen20092013.
\item Kandidat for Enhedslisten i Brønshøjkredsen20032006.
\item Kandidat for Enhedslisten i Odense Østkredsen20012003.
\item Kandidat for Enhedslisten i Otterupkredsen20012003.
\item Kandidat for Enhedslisten i Nørrebrokredsen19962001.
\end{itemize}
\lettersection{Erhvervserfaring}
\begin{itemize}
\item Ledelseskonsulent, 3Fs formandssekretariat,20092011.
\item Konsulent og rådgiver, FOA,20082008.
\item Underviser, Krogerup Højskole,20062006.
\item Kampagneleder for Operation Dagsværk, København,19981998.
\item Koordinator for den fri ungdomsuddannelse, Det frie Gymnasium,19961997.
\item Hjemmehjælper, Søllerød Kommune,19951995.
\item Folkekirkens Nødhjælp, 1999 og 2001,.
\end{itemize}
\lettersection{Publikationer}
Forfatter til raquoHvilket velfaeligrdssamfundlaquo, 2019. Medforfatter til raquoDet betaler sig at investere i mennesker  en bog om sociale investeringer, tidlig indsats, finansministeriets regnemodeller amp SOslashMlaquo, 2018 og raquoNed og op med stresslaquo, 2010. Har bidraget til raquoFra kamp til kultur  20 smagsdommere skyder med skarptlaquo, 2004 og raquoEn dollar om dagenlaquo, 2001.

\end{cvletter}
\end{document}