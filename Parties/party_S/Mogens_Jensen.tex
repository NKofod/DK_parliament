%!TEX TS-program = xelatex
%!TEX encoding = UTF-8 Unicode
\documentclass[11pt, a4paper]{awesome-cv}
\geometry{left=1.4cm, top=.8cm, right=1.4cm, bottom=1.8cm, footskip=.5cm}
\fontdir[fonts/]
\colorlet{awesome}{S-colour}
\setbool{acvSectionColorHighlight}{true}
\renewcommand{\acvHeaderSocialSep}{\quad\textbar\quad}
\recipient{}{}
\name{Mogens}{Jensen}
\mobile{+45 3337 5500}
\email{mogens.jensenft.dk}
\position{Fhv. minister{\enskip\cdotp\enskip}Socialdemokratiet}
\address{}
\photo[circle,noedge,left]{"./party_S/Mogens_Jensen_profile.jpg"}
\letterdate{\today}
\lettertitle{Mogens Jensen - Blå Bog}
\letteropening{}
\letterclosing{}
\letterenclosure[Attached]{Stemme Statistik}
\begin{document}
\makecvheader[R]
\makecvfooter{\today}{\lettertitle{Mogens Jensen - Blå Bog}}{}
\makelettertitle
\begin{cvletter}
\lettersection{Baggrund}
Mogens Jensen, født 31. oktober 1963 i Nykøbing Mors, søn af ekspedient Harry Jensen og hjemmehjælper Ebba Møller Jensen.

\lettersection{Uddannelse}
\begin{itemize}
\item Fagbevægelsens lederuddannelse, LOSkolen, Helsingør,19971999.
\item Student, Morsø Gymnasium,19791982.
\item Nykøbing Mors Folke og Realskole,19701979.
\end{itemize}
\lettersection{Parlamentarisk Karriere}
\subsection*{Ministerposter}
\begin{itemize}
\item Minister for fødevarer, fiskeri og ligestilling og minister for nordisk samarbejde27. juni 2019  19. november 2020.
\item Handels og udviklingsminister3. februar 2014  28. juni 2015.
\end{itemize}
\subsection*{Ordførerskaber}
\begin{itemize}
\item Forsvarsordfører
\end{itemize}
\subsection*{Parlamentariske Tillidsposter}
\begin{itemize}
\item Formand for Transportudvalgetfra 2021.
\item Næstformand for Kulturudvalgetfra 2020.
\item Næstformand for Retsudvalget20152019.
\item Næstformand for Socialdemokratietfra 2012.
\item Formand for den socialdemokratiske folketingsgruppe20112012.
\item Tidligere formand og næstformand for delegationen til Europarådets parlamentariske forsamling PACE.
\item Tidligere ordfører for Grønland, Færøerne og Nordisk Råd.
\item Tidligere ordfører for kultur, medier og idræt.
\end{itemize}
\subsection*{Folketinget}
\subsubsection*{Medlemsperioder}
\begin{itemize}
\item Folketingsmedlem for Socialdemokratiet i Vestjyllands Storkreds fra 13. november 2007.
\item Folketingsmedlem for Socialdemokratiet i Vestjyllands Storkredsfra 13. november 2007.
\item Folketingsmedlem for Socialdemokratiet i Ringkøbing Amtskreds8. februar 2005  13. november 2007.
\end{itemize}
\subsubsection*{Kandidaturer}
\begin{itemize}
\item Kandidat for Socialdemokratiet i Herning Sydkredsenfra 2007.
\item Kandidat for Socialdemokratiet i Herningkredsen20032006.
\end{itemize}
\lettersection{Erhvervserfaring}
\begin{itemize}
\item Konsulent, LO,19872005.
\item Ulandskonsulent, Arbejdernes Oplysningsforbund,19861987.
\item Kulturkonsulent, Arbejdernes Oplysningsforbund,19851986.
\item Uddannelsessekretær, Danmarks Socialdemokratiske Ungdom,19821985.
\end{itemize}
\lettersection{Publikationer}
Medforfatter til raquoDe rejste sig trodsigt i vrimlenlaquo, 2021. Har bl.a. skrevet debatoplaeligggene raquoEn samordnet ungdomsuddannelselaquo, 1985, og raquoMenneskene skal blomstrelaquo, 1987. Bidrag til LOs 100aringrsjubilaeligumsbog raquoI takt med tidenlaquo, 1998. Diverse film og teaterproduktioner.

\end{cvletter}
\end{document}