%!TEX TS-program = xelatex
%!TEX encoding = UTF-8 Unicode
\documentclass[11pt, a4paper]{awesome-cv}
\geometry{left=1.4cm, top=.8cm, right=1.4cm, bottom=1.8cm, footskip=.5cm}
\fontdir[fonts/]
\colorlet{awesome}{V-colour}
\setbool{acvSectionColorHighlight}{true}
\renewcommand{\acvHeaderSocialSep}{\quad\textbar\quad}
\recipient{}{}
\name{Karen}{Ellemann}
\mobile{+45 3337 4502}
\email{karen.ellemannft.dk}
\position{Medlem af Folketinget{\enskip\cdotp\enskip}Venstre}
\address{}
\photo[circle,noedge,left]{"./party_V/Karen_Ellemann_profile.jpg"}
\letterdate{\today}
\lettertitle{Karen Ellemann - Blå Bog}
\letteropening{}
\letterclosing{}
\letterenclosure[Attached]{Stemme Statistik}
\begin{document}
\makecvheader[R]
\makecvfooter{\today}{\lettertitle{Karen Ellemann - Blå Bog}}{}
\makelettertitle
\begin{cvletter}
\lettersection{Baggrund}
Karen Ellemann Kloch, foslashdt 26. august 1969 i Charlottenlund, datter af fhv. udenrigsminister Uffe EllemannJensen og sekretaeligr Hanne EllemannJensen.nbspGift med Kresten Kloch. Mor til to boslashrn, foslashdt i 1998 og 1999.

\lettersection{Uddannelse}
\begin{itemize}
\item Uddannet lærer, N. Zahles Seminarium,20022004.
\item Student, Holte Gymnasium,19861989.
\end{itemize}
\lettersection{Parlamentarisk Karriere}
\subsection*{Ministerposter}
\begin{itemize}
\item Minister for fiskeri og ligestilling og minister for nordisk samarbejde7. august 2017  2. maj 2018.
\item Minister for ligestilling og minister for nordisk samarbejde28. november 2016  7. august 2017.
\item Social og indenrigsminister28. juni 2015  28. november 2016.
\item Miljøminister og minister for nordisk samarbejde23. februar 2010  3. oktober 2011.
\item Indenrigs og socialminister7. april 2009  23. februar 2010.
\end{itemize}
\subsection*{Ordførerskaber}
\begin{itemize}
\item Udviklingsordfører
\end{itemize}
\subsection*{Parlamentariske Tillidsposter}
\begin{itemize}
\item Formand for Folketingets Tværpolitiske netværk for seksuel og reproduktiv sundhed og rettighederfra 2020.
\item Næstformand for Udvalget for Forretningsordenenfra 2019.
\item Udviklingsordførerfra 2019.
\item Formand for Venstres folketingsgruppe20182019.
\item Socialordfører20142015.
\item Folkeskoleordfører, ungdomsuddannelsesordfører20112014.
\item Medlem af Børne og Undervisningsudvalget, Socialudvalget, Ligestillingsudvalget, Udvalget for Udlændinge og Integrationspolitik og Nordisk Råd20112015.
\item Ordfører for familiepolitik og udlændinge og integration20072009.
\item Medlem af Socialudvalget, Udenrigsudvalget og Udvalget for Udlændinge og Integrationspolitik20072009.
\end{itemize}
\subsection*{Folketinget}
\subsubsection*{Medlemsperioder}
\begin{itemize}
\item Folketingsmedlem for Venstre i Københavns Omegns Storkreds fra 13. november 2007.
\end{itemize}
\subsubsection*{Kandidaturer}
\begin{itemize}
\item Kandidat for Venstre i Gentoftekredsenfra 2020.
\item Kandidat for Venstre i Brøndbykredsen20072020.
\end{itemize}
\subsection*{Folketingets Præsidium}
\begin{itemize}
\item Medlem af Folketingets Præsidiumfra 13. august 2019.
\end{itemize}
\lettersection{Erhvervserfaring}
\begin{itemize}
\item Lærer, Rungsted Skole,20032007.
\item Selvstændig journalist,19992002.
\item Daglig leder, Dagmar Teatret,19961999.
\item Administrationschef, Brinkmann Kommunikation,19951996.
\item Daglig leder, Scandinavian Models,19931995.
\item Konsulent, The Voice,19901993.
\end{itemize}
\end{cvletter}
\end{document}