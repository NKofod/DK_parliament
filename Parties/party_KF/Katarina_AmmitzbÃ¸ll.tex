%!TEX TS-program = xelatex
%!TEX encoding = UTF-8 Unicode
\documentclass[11pt, a4paper]{awesome-cv}
\geometry{left=1.4cm, top=.8cm, right=1.4cm, bottom=1.8cm, footskip=.5cm}
\fontdir[fonts/]
\colorlet{awesome}{KF-colour}
\setbool{acvSectionColorHighlight}{true}
\renewcommand{\acvHeaderSocialSep}{\quad\textbar\quad}
\recipient{}{}
\name{Katarina}{Ammitzbøll}
\mobile{+45 3337 4214}
\email{katarina.ammitzbollft.dk}
\position{Medlem af Folketinget{\enskip\cdotp\enskip}Det Konservative Folkeparti}
\address{}
\photo[circle,noedge,left]{"./party_KF/Katarina_Ammitzbøll_profile.jpg"}
\letterdate{\today}
\lettertitle{Katarina Ammitzbøll - Blå Bog}
\letteropening{}
\letterclosing{}
\letterenclosure[Attached]{Stemme Statistik}
\begin{document}
\makecvheader[R]
\makecvfooter{\today}{\lettertitle{Katarina Ammitzbøll - Blå Bog}}{}
\makelettertitle
\begin{cvletter}
\lettersection{Baggrund}
KatarinanbspAmmitzboslashll, foslashdt 15. juli 1969 i Charlottenlund, datter af civilingenioslashr Kurt Ammitzboslashll og cand.jur. Elisabeth Lawrence.nbspGift.

\lettersection{Uddannelse}
\begin{itemize}
\item Certifikat i lederskab, Harvard Business School, Monaco og AVT Business School, København,20092010.
\item LLM Master in  international development law, Warwick University, Storbritannien,20042005.
\item Cand.scient.soc offentlig driftsøkonomi og international udvikling, Roskilde Universitet,19891997.
\end{itemize}
\lettersection{Parlamentarisk Karriere}
\subsection*{Ordførerskaber}
\begin{itemize}
\item Forsyningsordfører
\item Energiordfører
\item EUordfører
\item Itordfører
\item Menneskerettighedsordfører
\item Telekommunikationsordfører
\item Udviklingsordfører
\end{itemize}
\subsection*{Parlamentariske Tillidsposter}
\begin{itemize}
\item Formand for Folketingets 2030 netværk om FNs Verdensmålfra 2021.
\item Menneskerettighedsordførerfra 2021.
\item Energi og forsyningsordførerfra 2021.
\item IT og telekommunikationsordførerfra 2021.
\item Ungdomsuddannelsesordfører20212021.
\item EUordførerfra 2019.
\item Ordfører for international udviklingfra 2019.
\item Tidligere forskningsordfører og uddannelsesordfører.
\end{itemize}
\subsection*{Folketinget}
\subsubsection*{Medlemsperioder}
\begin{itemize}
\item Folketingsmedlem for Det Konservative Folkeparti i Københavns Storkreds fra 5. juni 2019.
\end{itemize}
\subsubsection*{Kandidaturer}
\begin{itemize}
\item Kandidat for Det Konservative Folkeparti i Falkonerkredsenfra 2018.
\item Kandidat for Det Konservative Folkeparti i Aarhus Østkredsen20142015.
\end{itemize}
\lettersection{Erhvervserfaring}
\begin{itemize}
\item Senior Manager, Strategy  Shared Value Sustainability, A.P. Møller Mærsk,20172019.
\item Manager, Risk Advisory, Deloitte,20162017.
\item Senior Manager, Peace Nexus Foundation, Schweiz,20122015.
\item Ministerråd, Danmarks faste Mission ved FN, Genève,20112012.
\item Seniorrådgiver, Udenrigsministeriet,20072011.
\item Vicedirektør, Control Risks, London,20062007.
\item Seniorrådgiver, NATO Allied Command Transformation ACT,20062006.
\item Forskningskonsulent, Norwegian Institute of International Affairs NUPI,20062006.
\item Demokratiseringsekspert, European Union, Electoral Commission of Afghanistan,20052005.
\item Vicedirektør, FNs Udviklingsprogram, Afghanistan,20022004.
\item Seniorrådgiver, African Centre for the Constructive Resolution of Disputes, Sydafrika, Burundi og Mozambique,20022002.
\item Særlig rådgiver, FNs Udviklingsprogram nedrustning af mindre våben, New York,20022002.
\item Seniorrådgiver og koordinator, UN Department of Peacekeeping Operations, Østtimor,20002001.
\item Manager for vidensdeling governance, FNs Udviklingsprogram, New York,19992000.
\item Policy Specialist JPO, FNs Udviklingsprogram, Egypten,19971999.
\end{itemize}
\lettersection{Publikationer}
Medforfatter til raquoProgramming for catalytic effects in peacebuilding  a guidelaquo, PeaceNexus, 2012,nbsp raquoFirst steps in postconflict state building a UNDPUSAID studylaquo, United States Agency for International Development, 2007, raquoMaximum or minimum Policy options for democratisation initiatives in UN peace operationslaquo, Norwegian Institute of International Affairs, 2007. Har bidraget til raquoInternational crisis management squaring the circlelaquo, National Defence Academy, Austrian Ministry of Defence and Sports and Geneva Centre for Security Policy, 2011, raquoUnintended consequences of peace building operationslaquo, United Nations University Press, 2006, raquoAfghanistan  a development democracylaquo, Swedish Committee for Afghanistan, 2005.

\end{cvletter}
\end{document}