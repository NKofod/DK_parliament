%!TEX TS-program = xelatex
%!TEX encoding = UTF-8 Unicode
\documentclass[11pt, a4paper]{awesome-cv}
\geometry{left=1.4cm, top=.8cm, right=1.4cm, bottom=1.8cm, footskip=.5cm}
\fontdir[fonts/]
\colorlet{awesome}{DF-colour}
\setbool{acvSectionColorHighlight}{true}
\renewcommand{\acvHeaderSocialSep}{\quad\textbar\quad}
\recipient{}{}
\name{Pia}{Kjærsgaard}
\mobile{+45 3337 5107}
\email{dfpksekr@ft.dk}
\position{Fhv. formand for Folketinget{\enskip\cdotp\enskip}Dansk Folkeparti}
\address{}
\photo[circle,noedge,left]{"./Pia_Kjærsgaard_profile.jpg"}
\letterdate{\today}
\lettertitle{Pia Kjærsgaard - Blå Bog}
\letteropening{}
\letterclosing{}
\letterenclosure[Attached]{Stemme Statistik}
\begin{document}
\makecvheader[R]
\makecvfooter{\today}{\lettertitle{Pia Kjærsgaard - Blå Bog}}{}
\makelettertitle
\begin{cvletter}
\lettersection{Baggrund}
Pia Merete Kjærsgaard, født 23. februar 1947 i København, datter af farvehandler Poul Kjærsgaard og husmoder Inge Munch Jensen. Gift med Henrik Thorup, statsrevisor.

\lettersection{Uddannelse}
\begin{itemize}
\item Købmandsskolen, København, 1963-1965.
\item Folkeskolen, 1954-1963.
\end{itemize}
\lettersection{Parlamentarisk Karriere}
\subsection*{Ordførerskaber}
\begin{itemize}
\item Værdiordfører
\item Dyrevelfærdsordfører
\item Udlændingeordfører
\end{itemize}
\subsection*{Parlamentariske Tillidsposter}
\begin{itemize}
\item Værdiordfører fra 2020.
\item Dyrevelfærd fra 2019.
\item Næstformand for Udlændinge- og Integrationsudvalget fra 2019.
\item Udlændingeordfører fra 2019.
\item Medlem af Udvalget for Forretningsordenen (formand 2015-2019) fra 2015.
\item Delegeret ved FN’s Generalforsamling i New York i 1995, 2000, 2003, 2006, 2011 2012-2012.
\item Værdiordfører 2012-2015.
\item Medlem af Nordisk Råd 1998-2000.
\item Medlem af Forsvarskommissionen 1997-1998.
\item Formand for Dansk Folkeparti 1995-2012.
\item Medstifter af Dansk Folkeparti 1995-1995.
\item Medlem af Nordisk Råd 1990-1994.
\item Politisk leder af Fremskridtspartiet 1985-1994.
\item Tidligere stedfortræder i Dansk Interparlamentarisk Gruppes bestyrelse.
\item Tidligere medlem af Det Udenrigspolitiske Nævn, Udvalget vedrørende Efterretningstjenesterne, Udenrigsudvalget og OSCE’s Parlamentariske Forsamling.
\end{itemize}
\subsection*{Folketinget}
\subsubsection*{Medlemsperioder}
\begin{itemize}
\item Folketingsmedlem for Dansk Folkeparti i Københavns Omegns Storkreds fra 18. juni 2015.
\item Folketingsmedlem for Dansk Folkeparti i Sjællands Storkreds 13. november 2007 - 18. juni 2015.
\item Folketingsmedlem for Dansk Folkeparti i Københavns Amtskreds 11. marts 1998 - 13. november 2007.
\item Folketingsmedlem for Dansk Folkeparti i Fyns Amtskreds 6. oktober 1995 - 11. marts 1998.
\item Folketingsmedlem for Fremskridtspartiet i Fyns Amtskreds 8. september 1987 - 5. oktober 1995.
\item Folketingsmedlem for Fremskridtspartiet i Københavns Amtskreds 10. januar 1984 - 8. september 1987.
\end{itemize}
\subsubsection*{Kandidaturer}
\begin{itemize}
\item Kandidat for Dansk Folkeparti i Ballerupkredsen fra 2013.
\item Kandidat for Dansk Folkeparti i Kalundborgkredsen 2007-2013.
\item Kandidat for Dansk Folkeparti i Glostrupkredsen 1997-2006.
\item Kandidat for Dansk Folkeparti i Hellerupkredsen 1997-2000.
\item Kandidat for Dansk Folkeparti i Gentoftekredsen 1997-2000.
\item Kandidat for Fremskridtspartiet i Middelfartkredsen 1984-1995.
\item Kandidat for Fremskridtspartiet i Hvidovrekredsen 1983-1984.
\item Kandidat for Fremskridtspartiet i Ballerupkredsen 1981-1984.
\item Kandidat for Fremskridtspartiet i Gladsaxekredsen 1981-1984.
\item Kandidat for Fremskridtspartiet i Ryvangskredsen 1979-1981.
\end{itemize}
\subsection*{Folketingets Præsidium}
\begin{itemize}
\item Medlem af Folketingets Præsidium fra 2. oktober 2012.
\end{itemize}
\lettersection{Erhvervserfaring}
\begin{itemize}
\item Hjemmehjælper, 1978-1984.
\item Ansat på kontor inden for forsikrings- og reklamevirksomhed, 1963-1967.
\end{itemize}
\lettersection{Publikationer}
Har skrevet »... men udsigten er god - midtvejdserindringer«, 1998, og selvbiografien »Fordi jeg var nødt til det«, sammen med journalist Jette Meier Carlsen, 2013. Medforfatter til »Digteren og partiformanden« sammen med Henrik Nordbrandt, 2006. Redaktør af »Jul på Borgen II«, 2012, »Jul på Borgen III«, 2016, »Jul på Borgen IV«, 2017, »Jul på Borgen V«, 2018, »Jul på Borgen VI«, 2019 og »Jul på Borgen VII«, 2020.

\end{cvletter}
\end{document}