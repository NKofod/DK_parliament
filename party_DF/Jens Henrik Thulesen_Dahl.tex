%!TEX TS-program = xelatex
%!TEX encoding = UTF-8 Unicode
\documentclass[11pt, a4paper]{awesome-cv}
\geometry{left=1.4cm, top=.8cm, right=1.4cm, bottom=1.8cm, footskip=.5cm}
\fontdir[fonts/]
\colorlet{awesome}{DF-colour}
\setbool{acvSectionColorHighlight}{true}
\renewcommand{\acvHeaderSocialSep}{\quad\textbar\quad}
\recipient{}{}
\name{Jens Henrik Thulesen}{Dahl}
\mobile{+45 3337 5126}
\email{jens.henrik.thulesen.dahl@ft.dk}
\position{Medlem af Folketinget{\enskip\cdotp\enskip}Dansk Folkeparti}
\address{}
\photo[circle,noedge,left]{"./party_DF/Jens Henrik Thulesen_Dahl_profile.jpg"}
\letterdate{\today}
\lettertitle{Jens Henrik Thulesen Dahl - Blå Bog}
\letteropening{}
\letterclosing{}
\letterenclosure[Attached]{Stemme Statistik}
\begin{document}
\makecvheader[R]
\makecvfooter{\today}{\lettertitle{Jens Henrik Thulesen Dahl - Blå Bog}}{}
\makelettertitle
\begin{cvletter}
\lettersection{Baggrund}
Jens Henrik Winther Thulesen Dahl, født 20. juli 1961 i Brædstrup, søn af afdøde overlærer Anders Thulesen Dahl og pensioneret overlærer Inge Margrethe Simoni Dahl. Gift med Helle Winther Dahl. Har tre sønner.

\lettersection{Uddannelse}
\begin{itemize}
\item Diplomuddannelse i ledelse, VIA University College, 2009-2009.
\item Voksenpædagogisk grundkursus, 1996-1996.
\item Civilingeniør, Aalborg Universitet, 1985-1985.
\item Nysproglig student, Rosborg Gymnasium, 1980-1980.
\end{itemize}
\lettersection{Parlamentarisk Karriere}
\subsection*{Ordførerskaber}
\begin{itemize}
\item Uddannelsesordfører
\item Erhvervsuddannelsesordfører
\item Forskningsordfører
\item Kommunalordfører
\end{itemize}
\subsection*{Parlamentariske Tillidsposter}
\begin{itemize}
\item Erhvervsuddannelsesordfører og kommunalordfører fra 2019.
\item Næstformand for Undervisningsudvalget 2016-2019.
\item Næstformand for Børne- og Undervisningsudvalget 2015-2016.
\item Forebyggelsesordfører 2011-2015.
\item Uddannelses- og forskningsordfører fra 2011.
\end{itemize}
\subsection*{Folketinget}
\subsubsection*{Medlemsperioder}
\begin{itemize}
\item Folketingsmedlem for Dansk Folkeparti i Fyns Storkreds fra 15. september 2011.
\end{itemize}
\subsubsection*{Kandidaturer}
\begin{itemize}
\item Kandidat for Dansk Folkeparti i Assenskredsen fra 2010.
\end{itemize}
\lettersection{Erhvervserfaring}
\begin{itemize}
\item Konsulentopgaver, eget konsulentfirma, fra 2008.
\item Arbejdsmiljøkonsulent, hr-afdelingen, borgmesterens afdeling, Aarhus Kommune, 2008-2011.
\item Undervisning i indeklima og støj, Ergo- og Fysioterapiskolen, Aarhus, 2001-2011.
\item Souschef, Aarhus Kommunes BST, 2000-2006.
\item Konstitueret leder , Aarhus Kommunes BST, 1999-1999.
\item Arbejdsmiljøkonsulent, Udviklingshuset, Aarhus Kommune (tidligere Aarhus Kommunes BST), 1994-2007.
\item Projektingeniør og VKO-konsulent, Carl Bro A/S, 1989-1994.
\item Ekstern undervisningsassistent, Institut for Bygningsteknik, Aalborg Universitet, 1986-1991.
\item Fagingeniør i VVS-afdelingen, rådgivende ingeniørfirma Nellemann, Aalborg, 1985-1989.
\end{itemize}
\end{cvletter}
\end{document}