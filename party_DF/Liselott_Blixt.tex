%!TEX TS-program = xelatex
%!TEX encoding = UTF-8 Unicode
\documentclass[11pt, a4paper]{awesome-cv}
\geometry{left=1.4cm, top=.8cm, right=1.4cm, bottom=1.8cm, footskip=.5cm}
\fontdir[fonts/]
\colorlet{awesome}{DF-colour}
\setbool{acvSectionColorHighlight}{true}
\renewcommand{\acvHeaderSocialSep}{\quad\textbar\quad}
\recipient{}{}
\name{Liselott}{Blixt}
\mobile{+45 3337 5120}
\email{liselott.blixt@ft.dk}
\position{Medlem af Folketinget{\enskip\cdotp\enskip}Dansk Folkeparti}
\address{}
\photo[circle,noedge,left]{"./party_DF/Liselott_Blixt_profile.jpg"}
\letterdate{\today}
\lettertitle{Liselott Blixt - Blå Bog}
\letteropening{}
\letterclosing{}
\letterenclosure[Attached]{Stemme Statistik}
\begin{document}
\makecvheader[R]
\makecvfooter{\today}{\lettertitle{Liselott Blixt - Blå Bog}}{}
\makelettertitle
\begin{cvletter}
\lettersection{Baggrund}
Pia Liselott Blixt, født 22. februar 1965 i Lund i Sverige, datter af gårdejer Jørgen Kjeld Jørgensen og afdelingsleder i ISS Ulrika Eleonora Blixt Jørgensen. Har to børn, Daniel Christoffer, født 1991, og Christine Ulrika, født 1995.

\lettersection{Uddannelse}
\begin{itemize}
\item Social- og sundhedsassistent, Social- og sundhedsskolen, Greve, 2000-2002.
\item Buntmagerlærling, Lollands Buntmageri, 1981-1986.
\end{itemize}
\lettersection{Parlamentarisk Karriere}
\subsection*{Ordførerskaber}
\begin{itemize}
\item Etisk Råd-ordfører
\item Ligestillingsordfører
\item Ordfører vedr. nordisk samarbejde
\item Psykiatriordfører
\item Sundhedsordfører
\end{itemize}
\subsection*{Parlamentariske Tillidsposter}
\begin{itemize}
\item Ligestillingsordfører fra 2020.
\item Medlem af Dansk Interparlamentarisk Gruppes bestyrelse fra 2019.
\item Næstformand for Tilsynet i henhold til grundlovens § 71 fra 2019.
\item Ordfører vedr. nordisk samarbejde fra 2019.
\item Formand for Sundheds- og Ældreudvalget 2015-2019.
\item Medlem af Nordisk Råd fra 2012.
\item Formand for Tilsynet i henhold til grundlovens § 71 2011-2015.
\item Medlem af §71-tilsynet fra 2011.
\item Psykiatriordfører fra 2007.
\item Sundhedsordfører fra 2007.
\end{itemize}
\subsection*{Folketinget}
\subsubsection*{Medlemsperioder}
\begin{itemize}
\item Folketingsmedlem for Dansk Folkeparti i Sjællands Storkreds fra 13. november 2007.
\end{itemize}
\subsubsection*{Kandidaturer}
\begin{itemize}
\item Kandidat for Dansk Folkeparti i Køgekredsen fra 2012.
\item Kandidat for Dansk Folkeparti i Grevekredsen 2011-2012.
\item Kandidat for Dansk Folkeparti i Faxekredsen 2007-2010.
\end{itemize}
\lettersection{Erhvervserfaring}
\begin{itemize}
\item Social- og sundhedsassistent, vikarbureau, 2005-2007.
\item Social- og sundhedsassistent, Glostrup Sygehus, 2002-2005.
\item Selvstændig buntmager, Greve, fra 1995.
\item Hjemmehjælper, Greve Kommune, 1993-2002.
\item Hjemmehjælper, Københavns Kommune, 1989-1991.
\item Selvstændig buntmager, Rødby, 1986-1989.
\item Taxichauffør, Rødbyhavn, 1986-1988.
\end{itemize}
\lettersection{Publikationer}
Har skrevet »Gåsedrengen fra Skåne«, 2021.

\end{cvletter}
\end{document}