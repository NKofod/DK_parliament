%!TEX TS-program = xelatex
%!TEX encoding = UTF-8 Unicode
\documentclass[11pt, a4paper]{awesome-cv}
\geometry{left=1.4cm, top=.8cm, right=1.4cm, bottom=1.8cm, footskip=.5cm}
\fontdir[fonts/]
\colorlet{awesome}{DF-colour}
\setbool{acvSectionColorHighlight}{true}
\renewcommand{\acvHeaderSocialSep}{\quad\textbar\quad}
\recipient{}{}
\name{Morten}{Messerschmidt}
\mobile{+45 3337 5500}
\email{morten.messerschmidt@ft.dk}
\position{Medlem af Folketinget{\enskip\cdotp\enskip}Dansk Folkeparti}
\address{}
\photo[circle,noedge,left]{"./party_DF/Morten_Messerschmidt_profile.jpg"}
\letterdate{\today}
\lettertitle{Morten Messerschmidt - Blå Bog}
\letteropening{}
\letterclosing{}
\letterenclosure[Attached]{Stemme Statistik}
\begin{document}
\makecvheader[R]
\makecvfooter{\today}{\lettertitle{Morten Messerschmidt - Blå Bog}}{}
\makelettertitle
\begin{cvletter}
\lettersection{Baggrund}
Morten Messerschmidt, født 13. november 1980 i Frederikssund, søn af struktør Carsten Christoffersen og byrådsmedlem Inge Messerschmidt. Samboende med Dot Wessman.

\lettersection{Uddannelse}
\begin{itemize}
\item Cand.jur., Københavns Universitet, 2009-2009.
\item Student, Sankt Annæ Gymnasium, 1996-1999.
\end{itemize}
\lettersection{Parlamentarisk Karriere}
\subsection*{Ordførerskaber}
\begin{itemize}
\item Europarådsordfører
\item Energiordfører
\item EU-ordfører
\item Klimaordfører
\item Kulturordfører
\end{itemize}
\subsection*{Parlamentariske Tillidsposter}
\begin{itemize}
\item Kulturordfører fra 2020.
\item Politisk næstformand for Dansk Folkeparti fra 2020.
\item EU-ordfører, europarådsordfører, energi- og klimaordfører fra 2019.
\item Medlem af Europa-Parlamentet for Dansk Folkeparti 2009-2019.
\item Energiordfører og klimaordfører 2007-2009.
\item EU-ordfører 2005-2009.
\end{itemize}
\subsection*{Folketinget}
\subsubsection*{Medlemsperioder}
\begin{itemize}
\item Folketingsmedlem for Dansk Folkeparti i Nordsjællands Storkreds fra 5. juni 2019.
\item Folketingsmedlem for Dansk Folkeparti i Østjyllands Storkreds 13. november 2007 - 18. juni 2009.
\item Folketingsmedlem for Dansk Folkeparti i Århus Amtskreds 31. maj 2007 - 13. november 2007.
\item Folketingsmedlem for Uden for folketingsgrupperne i Århus Amtskreds 3. maj 2007 - 30. maj 2007.
\item Folketingsmedlem for Dansk Folkeparti i Århus Amtskreds 8. februar 2005 - 2. maj 2007.
\end{itemize}
\subsubsection*{Kandidaturer}
\begin{itemize}
\item Kandidat for Dansk Folkeparti i Fredensborgkredsen fra 2018.
\item Kandidat for Dansk Folkeparti i Djurskredsen 2007-2009.
\item Kandidat for Dansk Folkeparti i Silkeborgkredsen 2004-2006.
\end{itemize}
\lettersection{Erhvervserfaring}
\begin{itemize}
\item Juridisk konsulent, Stefansens Forlystelser, fra 2009.
\end{itemize}
\lettersection{Publikationer}
Har skrevet »Farvel til folkestyret: hvordan EU ødelægger frihed, folk og folkestyre - og hvad vi kan gøre ved det«, 2020 og »Intet over og intet ved siden af ... ‒ EU-Domstolen og dens aktivisme«, 2013. Medforfatter til »Overlad det trygt til Bruxelles &ndash; Debat om EU «, 2015.

\end{cvletter}
\end{document}