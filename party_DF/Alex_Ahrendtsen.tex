%!TEX TS-program = xelatex
%!TEX encoding = UTF-8 Unicode
\documentclass[11pt, a4paper]{awesome-cv}
\geometry{left=1.4cm, top=.8cm, right=1.4cm, bottom=1.8cm, footskip=.5cm}
\fontdir[fonts/]
\colorlet{awesome}{DF-colour}
\setbool{acvSectionColorHighlight}{true}
\renewcommand{\acvHeaderSocialSep}{\quad\textbar\quad}
\recipient{}{}
\name{Alex}{Ahrendtsen}
\mobile{+45 3337 5105}
\email{alex.ahrendtsenft.dk}
\position{Medlem af Folketinget{\enskip\cdotp\enskip}Dansk Folkeparti}
\address{}
\photo[circle,noedge,left]{"./party_DF/Alex_Ahrendtsen_profile.jpg"}
\letterdate{\today}
\lettertitle{Alex Ahrendtsen - Blå Bog}
\letteropening{}
\letterclosing{}
\letterenclosure[Attached]{Stemme Statistik}
\begin{document}
\makecvheader[R]
\makecvfooter{\today}{\lettertitle{Alex Ahrendtsen - Blå Bog}}{}
\makelettertitle
\begin{cvletter}
\lettersection{Baggrund}
Alex Ahrendtsen, født 14. februar 1967 i Kolding, søn af pensioneret entreprenørformand og tømrer Henning Ahrendtsen og pensioneret korrespondent Marianne Ahrendtsen.

\lettersection{Uddannelse}
\begin{itemize}
\item Cand.mag. i dansk, litteratur, religion og oldgræsk, Odense Universitet,19891996.
\end{itemize}
\lettersection{Parlamentarisk Karriere}
\subsection*{Ordførerskaber}
\begin{itemize}
\item Bygningsordfører
\item Europarådsordfører
\item Uddannelsesordfører
\item Børneordfører
\item Boligordfører
\item Erhvervsuddannelsesordfører
\item EUordfører
\item Færdselsordfører
\item Færøerneordfører
\item Folkeskoleordfører
\item Forskningsordfører
\item Forsvarsordfører
\item Grønlandsordfører
\item Gymnasieordfører
\item Transportordfører
\item Undervisningsordfører
\end{itemize}
\subsection*{Parlamentariske Tillidsposter}
\begin{itemize}
\item Børneordførerfra 2022.
\item Erhvervsuddannelsesordførerfra 2022.
\item Europarådsordførerfra 2022.
\item Færøerneordførernefra 2022.
\item Færdselsordførerfra 2022.
\item Forskningsordførerfra 2022.
\item Forsvarsordførerfra 2022.
\item Grønlandsordførerfra 2022.
\item Gymnasieordførerfra 2022.
\item Transportordførerfra 2022.
\item Uddannelsesordførerfra 2022.
\item Undervisningsordførerfra 2022.
\item EUordførerfra 2021.
\item Bolig og bygningsordførerfra 2020.
\item Ordfører for udviklingsbistand20202022.
\item Idrætsordfører20192020.
\item Medieordfører20192020.
\item Folkeskoleordførerfra 2011.
\item Kulturordfører20112020.
\end{itemize}
\subsection*{Folketinget}
\subsubsection*{Medlemsperioder}
\begin{itemize}
\item Folketingsmedlem for Dansk Folkeparti i Fyns Storkreds fra 15. september 2011.
\item Folketingsmedlem for Dansk Folkeparti i Fyns Storkredsfra 15. september 2011.
\end{itemize}
\subsubsection*{Kandidaturer}
\begin{itemize}
\item Kandidat for Dansk Folkeparti i Middelfartkredsenfra 2022.
\item Kandidat for Dansk Folkeparti i Odense Vestkredsenfra 2022.
\item Kandidat for Dansk Folkeparti i Odense Sydkredsenfra 2007.
\item Kandidat for Dansk Folkeparti i Svendborgkredsen20012007.
\end{itemize}
\lettersection{Erhvervserfaring}
\begin{itemize}
\item Forlægger, Trykkefrihedsselskabets Bibliotek, Odense,20072014.
\item Forlægger, Lysias,20032020.
\item Forfatter på forskellige forlag,fra 1997.
\item Receptionist, underviser, redaktør, tekstforfatter, produktionsmedhjælper og selvstændig,19962007.
\end{itemize}
\lettersection{Publikationer}
Forfatter til raquoNaringr danskere boslashjer af ndash Islamiseringen af Odenselaquo, 2009, raquoDanmark ndash Fortaeligllinger ved kornmodlaquo, 2003, raquoDen danske ligevaeliggtlaquo, 2002, raquoMartin A. Hansen og Indre Missionlaquo, 1997 og raquoIndsigt eller fordom ndash missionen i litteraturen fra Aakjaeligr til Hoslasheglaquo, 1997.

\end{cvletter}
\end{document}