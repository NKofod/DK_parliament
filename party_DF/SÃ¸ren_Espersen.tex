%!TEX TS-program = xelatex
%!TEX encoding = UTF-8 Unicode
\documentclass[11pt, a4paper]{awesome-cv}
\geometry{left=1.4cm, top=.8cm, right=1.4cm, bottom=1.8cm, footskip=.5cm}
\fontdir[fonts/]
\colorlet{awesome}{DF-colour}
\setbool{acvSectionColorHighlight}{true}
\renewcommand{\acvHeaderSocialSep}{\quad\textbar\quad}
\recipient{}{}
\name{Søren}{Espersen}
\mobile{+45 3337 5124}
\email{soren.espersen@ft.dk}
\position{Fhv. medlem af Folketingets Præsidium{\enskip\cdotp\enskip}Dansk Folkeparti}
\address{}
\photo[circle,noedge,left]{"./Søren_Espersen_profile.jpg"}
\letterdate{\today}
\lettertitle{Søren Espersen - Blå Bog}
\letteropening{}
\letterclosing{}
\letterenclosure[Attached]{Stemme Statistik}
\begin{document}
\makecvheader[R]
\makecvfooter{\today}{\lettertitle{Søren Espersen - Blå Bog}}{}
\makelettertitle
\begin{cvletter}
\lettersection{Baggrund}
Søren Espersen, født 20. juli 1953 i Svenstrup, Himmerland, søn af Peder Christian Espersen og Inger Espersen, født Lund. Gift med Yvette Kim Fay Espersen. Parret har børnene Ditte, Anders, Robin, Susie og Peter og børnebørnene Daniel, Mai, Josefine, Mathilde og Estrid.



\lettersection{Uddannelse}
\begin{itemize}
\item Journalist, Danmarks Journalisthøjskole, 1976-1980.
\item Teologistudier, Københavns Universitet, 1974-1976.
\item Værnepligt, Søværnet, 1973-1974.
\item Studentereksamen, Hasseris Gymnasium, 1970-1973.
\item Realeksamen, Svenstrup Skole, 1967-1970.
\end{itemize}
\lettersection{Parlamentarisk Karriere}
\subsection*{Ordførerskaber}
\begin{itemize}
\item Færøerneordfører
\item Forsvarsordfører
\item Grønlandsordfører
\item Grundlovsordfører
\item Udenrigsordfører
\end{itemize}
\subsection*{Parlamentariske Tillidsposter}
\begin{itemize}
\item Forsvarsordfører fra 2020.
\item Formand for Det Udenrigspolitiske Nævn 2015-2019.
\item Politisk næstformand for Dansk Folkeparti 2012-2020.
\item Medlem af Forsvarskommissionen 2007-2009.
\item Næstformand for Færøudvalget 2007-2011.
\item Medlem af Dansk Folkepartis gruppebestyrelse, gruppesekretær fra 2006.
\item Færøerneordfører og grønlandsordfører fra 2005.
\item Grundlovsordfører, udenrigsordfører og ordfører for Rigsfællesskabet fra 2005.
\item Medlem af Nordisk Råd 2005-2006.
\item Medlem af OSCE's Parlamentariske Forsamling 2005-2006.
\item Medlem af den grønlandsk-danske selvstyrekommission 2005-2008.
\item  
\end{itemize}
\subsection*{Folketinget}
\subsubsection*{Medlemsperioder}
\begin{itemize}
\item Folketingsmedlem for Dansk Folkeparti i Sjællands Storkreds fra 18. juni 2015.
\item Folketingsmedlem for Dansk Folkeparti i Københavns Omegns Storkreds 13. november 2007 - 18. juni 2015.
\item Folketingsmedlem for Dansk Folkeparti i Københavns Amtskreds 8. februar 2005 - 13. november 2007.
\end{itemize}
\subsubsection*{Kandidaturer}
\begin{itemize}
\item Kandidat for Dansk Folkeparti i Kalundborgkredsen fra 2013.
\item Kandidat for Dansk Folkeparti i Ballerupkredsen 2007-2013.
\item Kandidat for Dansk Folkeparti i Hvidovrekredsen 2004-2006.
\item Kandidat for Dansk Folkeparti i Sundbykredsen 1997-2004.
\item Kandidat for Fremskridtspartiet i Præstøkredsen 1992-1995.
\end{itemize}
\subsection*{Folketingets Præsidium}
\begin{itemize}
\item Medlem af Folketingets Præsidium 28. november 2007 - 2. oktober 2012.
\end{itemize}
\lettersection{Erhvervserfaring}
\begin{itemize}
\item Pressechef, Dansk Folkeparti, 1997-2005.
\item Journalist, Billed-Bladet, 1992-1996.
\item Freelancejournalist, London, 1989-1992.
\item Journalist, BT, 1980-1988.
\item Journalist, Dagbladet, Ringsted, 1980-1980.
\end{itemize}
\lettersection{Publikationer}
Har skrevet »Israels selvstændighedskrig &ndash; og de danske frivillige«, 2007, »Valdemar Rørdam ‒ Nationalskjald og Landsforræder«, 2003 og »Danmarks fremtid &ndash; dit land, dit valg&hellip;« sammen med U. Dahlerup, K. Thulesen Dahl og A. Skjødt, 2001.  Ansvarshavende redaktør af Dansk Folkeblad 1997-2005.

\end{cvletter}
\end{document}