%!TEX TS-program = xelatex
%!TEX encoding = UTF-8 Unicode
\documentclass[11pt, a4paper]{awesome-cv}
\geometry{left=1.4cm, top=.8cm, right=1.4cm, bottom=1.8cm, footskip=.5cm}
\fontdir[fonts/]
\colorlet{awesome}{IA-colour}
\setbool{acvSectionColorHighlight}{true}
\renewcommand{\acvHeaderSocialSep}{\quad\textbar\quad}
\recipient{}{}
\name{Aaja Chemnitz}{Larsen}
\mobile{+45 3337 5454}
\email{aaja.larsen@ft.dk}
\position{Medlem af Folketinget{\enskip\cdotp\enskip}Inuit Ataqatigiit}
\address{}
\photo[circle,noedge,left]{"./Aaja Chemnitz_Larsen_profile.jpg"}
\letterdate{\today}
\lettertitle{Aaja Chemnitz Larsen - Blå Bog}
\letteropening{}
\letterclosing{}
\letterenclosure[Attached]{Stemme Statistik}
\begin{document}
\makecvheader[R]
\makecvfooter{\today}{\lettertitle{Aaja Chemnitz Larsen - Blå Bog}}{}
\makelettertitle
\begin{cvletter}
\lettersection{Baggrund}
Aaja Chemnitz Arnatsiaq Larsen, født 2. december 1977 i Nuuk, datter af Jørgen Schmidt Chemnitz og Jette Larsen. Samlevende med Michael Driefer.

\lettersection{Uddannelse}
\begin{itemize}
\item Executive Education Leadership Program, Harvard Kennedy School, 2020-2020.
\item Executive Management, INSEAD, 2011-2011.
\item Cand.scient.adm., Ilisimatusarfik, 1998-2004.
\end{itemize}
\lettersection{Parlamentarisk Karriere}
\subsection*{Parlamentariske Tillidsposter}
\begin{itemize}
\item Næstformand for Grønlandsudvalget fra 2017.
\item Medlem af Grønlands Landsting 2014-2018.
\end{itemize}
\subsection*{Folketinget}
\subsubsection*{Medlemsperioder}
\begin{itemize}
\item Folketingsmedlem for Inuit Ataqatigiit i Grønland fra 18. juni 2015.
\end{itemize}
\subsubsection*{Kandidaturer}
\begin{itemize}
\item Kandidat for Inuit Ataqatigiit i Grønland fra 2015.
\end{itemize}
\lettersection{Erhvervserfaring}
\begin{itemize}
\item Børnetalsmand, Grønlands Selvstyre, 2012-2015.
\item Direktør, Velfærdsforvaltningen, Kommuneqarfik Sermersooq, 2009-2012.
\item Afdelingschef, Nuup Kommunea, 2007-2009.
\item Associate Expert, FN, New York, 2006-2007.
\item Fuldmægtig, Bestyrelsessekretariatet, Grønlands Hjemmestyre, 2004-2006.
\end{itemize}
\end{cvletter}
\end{document}